% !TEX root=../root.tex

In addition to the hardware results presented in the following section, we
present simulation results to show the effectiveness of the proposed estimation
algorithm.
A multirotor UAV is simulated using the dynamics presented in~\secref{sec:BLAH}.
Additionally, a landing vehicle is simulated using the dynamics presented
in~\secref{sec:BLAH} using the parameters found
in~\tabref{table:BLAH}. As the estimator depends on
tracking and estimating the positions of visual landmarks that are rigidly
attached to the landing vehicle, landmark positions on the landing vehicle
are simulated as follows.
Positions for each landmark in the goal frame are
randomly sampled from the
uniform distribution between $\vect{r}_{min}$ and $\vect{r}_{max}$. A specified probability determines
whether the landmark is removed and a new landmark is generated or if the landmark
remains at each time step.

The necessary sensor measurements for the estimator are simulated using the true
state of the simulation. These sensor measurements include accelerometer,
gyroscope, global position of the UAV, global attitude 
of the UAV, relative position and attitude from the fiducial landing marker, and
landmark locations in the camera frame. All measurements are
corrupted with white gaussian noise according to the noise parameters
in~\tabref{tab:BLAH}. In addition, the IMU measurements contain a random
additive walk defied by the parameters in~\tabref{tab:BLAH}.

Somewhere in here that the $q_b2c = blah$ and $p_b2c = [0.25, -0.2, 0.4]$.

The results for a simulation where the estimator uses ten visual landmarks and
zero visual landmarks are seen in~\ref{fig:BLAH} and~\ref{fig:BLAH}
respectively. We do not include the plots of the UAV states,
$\vect{x}_{\text{UAV}}$ as it is well known that these states are easily
estimated with the given measurements.




\begin{table}[h!]
  \begin{center}
    \caption{Simulation Unicycle Motion Model Parameters.}
    \label{tab:sim_unicycle}
    \begin{tabular}{l|r}
      \textbf{Parameter} & \textbf{Value} \\
      \hline
      $\omega$ & 5 rad/s \\
      v & 5 m/s \\
      landmark $x_{\text{min}}$ & -2 m \\
      landmark $x_{\text{max}}$ & 2 m \\
      landmark $y_{\text{min}}$ & -2 m \\
      landmark $y_{\text{max}}$ & 2 m \\
      landmark $z_{\text{min}}$ & -1 m \\
      landmark $z_{\text{max}}$ & 1 m \\
      landmark disappear prob. & 1\% \\
    \end{tabular}
  \end{center}
\end{table}

\begin{table}[h!]
  \begin{center}
    \caption{Simulation Sensor Noise Characteristics.}
    \label{tab:sim_noise}
    \begin{tabular}{l|l|l}
      \textbf{Measurement} & \textbf{Std Dev} & \textbf{Rate} \\
      \hline
      accelerometer & 0.2 $m/s^2$ & 250 Hz \\
      walk & 0.05 $m/s^2$  \\
      init & 0.01 $m/s^2$ \\
      gyro & 0.1 $rad/s$ & 250 Hz \\
      walk & 0.01 $rad/s$  \\
      init & 0.01 $rad/s$ \\
      UAV global position & 0.01 $m$ & 10 Hz \\
      UAV global attitude & 0.001 $rad$ & 10 Hz \\
      Fiducial marker position & 0.1 $m$ & 30 Hz \\
      Fiducial marker attitude & 0.1 $rad$ & 30 Hz \\
      Landmark image point & 2.0 pixels & 30 Hz \\
    \end{tabular}
  \end{center}
\end{table}

% !TEX root=./root.tex

\section{Estimator using normal depth}
\subsection{Goal State Dynamics}
We assume that the landing vehicle moves with a constant velocity and a constant
angular velocity. The dynamics of the goal states are expressed as
\begin{align}
  \dot{\vect{p}}_{g/v}^{v} &= \left(R_{v}^{g}\right)^\transpose
  \vect{v}_{g/I}^{g} - \left(R_{v}^{b}\right)^\transpose \vect{v}_{b/I}^{b} \\
  \dot{\vect{v}}_{g/I}^{g} &= \vect{0} + \eta_{v} \\
  \dot{\theta}_{I}^{g} &= \omega_{g/I}^g \\
  \dot{\omega}_{g/I}^{g} &= 0 + \eta_{\omega} \\
  \dot{\vect r}_{i}^{b} &= \vect{0}.
\end{align}.

Note that the rotation from the vehicle frame to the goal frame is given by
\begin{equation}
  R_v^g =
  \begin{bmatrix}
    \ctheta & \stheta \\
    -\stheta & \ctheta
  \end{bmatrix}.
\end{equation}

\subsection{Motion Model Jacobians}
The full state jacobian is given by
\begin{equation}
  A =
  \begin{bmatrix}
    \frac{\partial \dot{\x}_{\text{UAV}}}{\partial \x_{\text{UAV}}} &
    \frac{\partial \dot{\x}_{\text{UAV}}}{\partial \x_{\text{Goal}}} \\
    \frac{\partial \dot{\x}_{\text{Goal}}}{\partial \x_{\text{UAV}}} &
    \frac{\partial \dot{\x}_{\text{Goal}}}{\partial \x_{\text{Goal}}} 
  \end{bmatrix}.
\end{equation}
We have defined the first term, $\frac{\partial \dot{\x}_{\text{UAV}}}{\partial
\x_{\text{UAV}}}$ above in the previous section. We also note that
\begin{equation}
  \frac{\partial \dot{\x}_{\text{UAV}}}{\partial \x_{\text{Goal}}} = \vect{0}
\end{equation}
as the UAV dynamics do not depend on the landing vehicle dynamics. We define the
remaining terms here below.

\subsubsection{Jacobian w.r.t. UAV State}
\begin{equation}
  \frac{\partial \dot{\x}_{\text{Goal}}}{\partial \x_{\text{UAV}}}
  =
  \begin{bmatrix}
    \vect{0} & \frac{\partial \dot{\vect{p}}_{\text{Goal}}}{\partial
      \vect{\theta}_{\text{UAV}}} & \frac{\partial
      \dot{\vect{p}}_{\text{Goal}}}{\partial \vect{v}_{\text{UAV}}} & 0 \\
    \vect{0} & \vect{0} & \vect{0} & 0 \\
    \vect{0} & \vect{0} & \vect{0} & 0 \\
    \vect{0} & \vect{0} & \vect{0} & 0 \\
  \end{bmatrix}
\end{equation}
\begin{align}
    \frac{\partial \dot{\vect{p}}_{\text{Goal}}}{\partial
      \vect{\theta}_{\text{UAV}}}
      &=
      - \frac{\partial}{\partial \vect{\theta}_{\text{UAV}}}
      \left(R_{v}^{b}\right)^\transpose \vect{v}_{b/I}^{b}
      \\
    \frac{\partial \dot{\vect{p}}_{\text{Goal}}}{\partial \vect{v}_{\text{UAV}}}
      &=
      - \left(R_{v}^{b}\right)^\transpose
      \\
\end{align}

\subsubsection{Jacobian w.r.t. Goal State}
\begin{equation}
  \frac{\partial \dot{\x}_{\text{Goal}}}{\partial \x_{\text{Goal}}}
  =
  \begin{bmatrix}
    \vect{0} & 0 & \frac{\partial \dot{\vect{p}}_{\text{Goal}}}{\partial
      \vect{v}_{\text{Goal}}} & \frac{\partial
      \dot{\vect{p}}_{\text{Goal}}}{\partial \theta_{\text{Goal}}} & 0 & \vect{0} & 0 \\
    \vect{0} & 0 & \vect{0} & 0 & 0 & \vect{0} & 0 \\
    \vect{0} & 0 & \vect{0} & 0 & \frac{\partial
      \dot{\theta}_{\text{Goal}}}{\partial \omega_{Goal}} & \vect{0} & 0 \\
    \vect{0} & 0 & \vect{0} & 0 & 0 & \vect{0} & 0 \\
    \vect{0} & 0 & \vect{0} & 0 & 0 & \vect{0} & 0
  \end{bmatrix}
\end{equation}
\begin{align}
    \frac{\partial \dot{\vect{p}}_{\text{Goal}}}{\partial
      \vect{v}_{\text{Goal}}}
      &=
      \left(R_{v}^{g}\right)^\transpose
      \\
    \frac{\partial \dot{\vect{p}}_{\text{Goal}}}{\partial \theta_{\text{Goal}}}
      &=
      \frac{\partial}{\partial \theta_{\text{Goal}}} \left(R_{v}^{g}\right)^\transpose \vect{v}_{g/I}^{g}
      \\
    \frac{\partial \dot{\theta}_{\text{Goal}}}{\partial \omega_{Goal}}
      &=
      1
\end{align}

\subsection{Goal ArUco Measurement}
\subsubsection{Measurement model}
In the case that and ArUco marker is placed as the landing pad, and therefore
defines the goal frame, we get a measurement of relative transfrom from the
camera to the ArUco frame. If we assume that there is a known rotation offset
between the ArUco frame and the goal frame, the measurements can be expressed as
\begin{equation}
  h \left( \x \right) = 
  \begin{bmatrix}
    \vect{p}_{a/c}^c \\
    \vect{q}_c^a,
  \end{bmatrix}
\end{equation}
where $F^a$ is the ArUco frame.
We can expand this measurement model using our estimated state
\begin{align}
  \hat{h} \left( \x \right) &=
  \begin{bmatrix}
    \hat{\vect{p}}_{a/c}^c \\
    \hat{\vect{q}}_c^a,
  \end{bmatrix} \\
  \hat{\vect{p}}_{a/c}^c  &= R_b^c \left( \hat{R}_I^b \hat{\vect{p}}_{g/v}^v -
    \vect{p}_{c/v}^b \right) \\
  \hat{\vect{q}}_{c}^a  &= R_g^a \hat{R}_I^g \hat{R}_b^I R_c^b
\end{align}

\subsubsection{Measurement Jacobians}
\begin{align}
  \frac{\partial \hat{\vect{p}}_{a/c}^c}{\partial \phi} =& R_b^c \frac{\partial
  R_I^b}{\partial \phi} \hat{\vect{p}}_{g/v}^v \\
  \frac{\partial \hat{\vect{p}}_{a/c}^c}{\partial \theta} =& R_b^c \frac{\partial
  R_I^b}{\partial \theta} \hat{\vect{p}}_{g/v}^v \\
  \frac{\partial \hat{\vect{p}}_{a/c}^c}{\partial \psi} =& R_b^c \frac{\partial
  R_I^b}{\partial \psi} \hat{\vect{p}}_{g/v}^v \\
    \frac{\partial \hat{\vect{p}}_{a/c}^c}{\partial \vect{p}_{g/v}^v} =& R_b^c \hat{R}_I^b
\end{align}

\subsection{Landmark Pixel Measurement}
We assume that landmark locations that are rigidly attached to the landing
vehicle are tracked in the camera image. For now we will just have a fixed
number of landmarks that are tracked for the entire duration of the flight, even
when they leave the FOV of the camera and reenter. In real life, however, the
landmarks must be dynamically added and removed from the estimated states vector
as they leave the FOV of the camera and more are acquired.

\subsubsection{Measurement Model}
The measurement is a pixel location of each landmark in the camera image
\begin{equation}
  \vect{z} =
  \begin{bmatrix}
    p_x & p_y
  \end{bmatrix}^\transpose.
\end{equation}
We can expand this measurements using our estimated states
\begin{align}
  \vect{z} &=
  \begin{bmatrix}
    p_x & p_y
  \end{bmatrix}^\transpose \\
  \vect{z} &=
  \begin{bmatrix}
    f_x \frac{\e_1 \vect{p}_{i/c}^c}{\e_3 \vect{p}_{i/c}^c} + c_x \\
    f_y \frac{\e_2 \vect{p}_{i/c}^c}{\e_3 \vect{p}_{i/c}^c} + c_y
  \end{bmatrix},
\end{align}
where $f_x$, $f_y$, $c_x$, and $c_y$ are constant parameters of the camera. We
can then express the position of the landmark $i$ with respect to the camera in the
camera frame as
\begin{align}
  \vect{p}_{i/c}^c &= R_b^c R_v^b \left(\vect{p}_{i/v}^v -
    \vect{p}_{c/v}^v\right) \\
    \vect{p}_{i/c}^c &= R_b^c \left(R_v^b \vect{p}_{i/v}^v -
    \vect{p}_{c/b}^b\right)
\end{align}
This is actually a little bit weird because of the state parameters, but
\begin{equation}
  \vect{p}_{i/v}^v = R_g^v \vect{p}_{i/g}^g + \vect{p}_{g/v}^v
  %\begin{bmatrix}
    %\e_1^\transpose \left( R_g^v \vect{p}_{i/g}^g + \vect{p}_{g/v}^v \right) \\
    %\e_2^\transpose \left( R_g^v \vect{p}_{i/g}^g + \vect{p}_{g/v}^v \right) \\
    %\left( \e_3^\transpose \vect{p}_{i/g}^g + \vect{p}_{g/v}^v \right)
  %\end{bmatrix}.
\end{equation}

\subsubsection{Measurement Jacobian}
For the purpose of deriving the jacobians, we will just look at the jacobians
with respect to $p_x$, the pixel location along the $x$ axis of the camera
frame. From above we have that
\begin{align}
  p_x &= f_x \frac{\e_1^\transpose \vect{p}_{i/c}^c}{\e_3^\transpose \vect{p}_{i/c}^c} + c_x \\ 
  p_x &= f_x \frac{\e_1^\transpose R_b^c \left(R_v^b \vect{p}_{i/v}^v -
      \vect{p}_{c/b}^b\right) }{\e_3^\transpose R_b^c \left(R_v^b \vect{p}_{i/v}^v -
  \vect{p}_{c/b}^b\right) } + c_x 
\end{align}
Note that all values are constants except for $\vect{p}_{i/v}^v$ and $R_v^b$.
The individual parts of the jacobian are given here:
\begin{equation}
  \frac{\partial p_x}{\partial \vect{p}_{g/v}^v} =
  \frac{f_x \e_1 R_b^c R_v^b}
    {\left(\e_3 R_b^c \left(R_v^b \vect{p}_{i/v}^v -
    \vect{p}_{c/b}^b\right)\right)}
    - \frac{\left(\e_3 R_b^c R_v^b \right) f_x \left(\e_1 R_b^c \left(R_v^b \vect{p}_{i/v}^v -
        \vect{p}_{c/b}^b\right)\right)} {\left(\e_3 R_b^c \left(R_v^b \vect{p}_{i/v}^v -
  \vect{p}_{c/b}^b\right)\right)^2}
\end{equation}
The derivatives for the euler angles are given by
\begin{equation}
  \frac{\partial p_x}{\partial \phi} =
  \frac{f_x \e_1 R_b^c \frac{\partial R_v^b}{\partial \phi} \vect{p}_{i/v}^v}
    {\left(\e_3 R_b^c \left(R_v^b \vect{p}_{i/v}^v -
    \vect{p}_{c/b}^b\right)\right)}
    - \frac{\left(\e_3 R_b^c \frac{\partial R_v^b}{\partial \phi} \vect{p}_{i/v}^v \right) f_x \left(\e_1 R_b^c \left(R_v^b \vect{p}_{i/v}^v -
        \vect{p}_{c/b}^b\right)\right)} {\left(\e_3 R_b^c \left(R_v^b \vect{p}_{i/v}^v -
  \vect{p}_{c/b}^b\right)\right)^2}
\end{equation}
where the jacobians for the other angles are similar, just substituting in.
The derivative for the goal angle, $\theta_g$ is given by using
\begin{equation}
  \frac{\partial \vect{p}_{i/v}^v}{\partial \theta_g} =
  \frac{\partial R_g^v}{\partial \theta_g} \vect{p}_{i/g}^g
  %\begin{bmatrix}
    %\e_1^\transpose \frac{\partial R_g^v}{\partial \theta_g} \vect{p}_{i/g}^g \\
    %\e_2^\transpose \frac{\partial R_g^v}{\partial \theta_g} \vect{p}_{i/g}^g \\
    %0
  %\end{bmatrix}
\end{equation}
in the jacobian given by
\begin{equation}
  \frac{\partial p_x}{\partial \theta_g} =
  \frac{f_x \e_1 R_b^c R_v^b \frac{\partial \vect{p}_{i/v}^v}{\partial \theta_g}}
    {\left(\e_3 R_b^c \left(R_v^b \vect{p}_{i/v}^v -
    \vect{p}_{c/b}^b\right)\right)}
    - \frac{\left(\e_3 R_b^c R_v^b \frac{\partial \vect{p}_{i/v}^v}{\partial \theta_g} \right) f_x \left(\e_1 R_b^c \left(R_v^b \vect{p}_{i/v}^v -
        \vect{p}_{c/b}^b\right)\right)} {\left(\e_3 R_b^c \left(R_v^b \vect{p}_{i/v}^v -
  \vect{p}_{c/b}^b\right)\right)^2}
\end{equation}
Similarly, the derivative for the landmark offset, $\vect{r}_i$ or
$\vect{p}_{i/g}^g$ is given by using
\begin{equation}
  \frac{\partial \vect{p}_{i/v}^v}{\partial \vect{p}_{i/g}^g} = R_g^v 
  %\begin{bmatrix}
    %\e_1^\transpose R_g^v \\
    %\e_2^\transpose R_g^v \\
    %\e_3^\transpose
  %\end{bmatrix}
\end{equation}
in the jacobian given by
\begin{equation}
  \frac{\partial p_x}{\partial \vect{p}_{i/g}^g} =
  \frac{f_x \e_1 R_b^c R_v^b \frac{\partial \vect{p}_{i/v}^v}{\partial \vect{p}_{i/g}^g}}
    {\left(\e_3 R_b^c \left(R_v^b \vect{p}_{i/v}^v -
    \vect{p}_{c/b}^b\right)\right)}
    - \frac{\left(\e_3 R_b^c R_v^b \frac{\partial \vect{p}_{i/v}^v}{\partial \vect{p}_{i/g}^g} \right) f_x \left(\e_1 R_b^c \left(R_v^b \vect{p}_{i/v}^v -
        \vect{p}_{c/b}^b\right)\right)} {\left(\e_3 R_b^c \left(R_v^b \vect{p}_{i/v}^v -
  \vect{p}_{c/b}^b\right)\right)^2}
\end{equation}
The jacobians for the y pixel measurement, $p_y$ are similar to the ones derived
above, just changing $f_y$ in place of $f_x$ and $\e_2$ instead of $\e_1$.

\subsection{Landmark Initialization}
Each time a new landmark is acquired and added to the estimated vector, we
should initialize its location to something intelligent. Theoretically, we could
just initialize the landmark's location to zero with a large enough associated
covariance. However, by initializing the landmark to the location based on its
initial pixel measurement, we can initialize the landmark with a smaller
covariance.

When we first acquire a landmark, the only information we have about it is a
measurement of its pixel location in the image frame. This measurement, denoted
by
\begin{equation}
  h \left( \x \right) = 
  \begin{bmatrix}
    p_x & p_y
  \end{bmatrix}^\transpose
\end{equation}
can be used to deduce the landmark's relative position vector that is estimated.
The easiest way to see this is by first creating a virtual image plane. This
virtual image plane projects the landmark pixel location into the image plane as
if the camera were perfectly aligned with the inertial, vehicle frame. We can
derive this using the pin hole camera model where
\begin{align}
  \begin{bmatrix}
    p_x \\ p_y \\ 1
  \end{bmatrix} &= K
  \begin{bmatrix}
    X^c / Z^c \\
    Y^c / Z^c \\
    1
  \end{bmatrix} \\
  \begin{bmatrix}
    X^c / Z^c \\
    Y^c / Z^c \\
    1
  \end{bmatrix}
   &= K^{-1}
  \begin{bmatrix}
    p_x \\ p_y \\ 1
  \end{bmatrix} \\
  \begin{bmatrix}
    X^c / Z^c \\
    Y^c / Z^c \\
    1
  \end{bmatrix}^v
   &= R_b^v R_c^b K^{-1}
  \begin{bmatrix}
    p_x \\ p_y \\ 1
  \end{bmatrix} \\
\end{align}
The resulting vector, $
  \begin{bmatrix}
    X^c / Z^c \\
    Y^c / Z^c \\
    1
  \end{bmatrix}^v$
  is the vector $\vect{p}_{i/c}^v$ up to a scale factor. The vector can be scaled by assuming that the altitude of the landmark is equal to the
altitude of the goal. To get the expected altitude, we solve for
\begin{align}
  \e_3^\transpose \vect{p}_{g/c}^v &= \e_3^\transpose \vect{p}_{g/b}^v - \e_3^\transpose \vect{p}_{c/b}^v \\
  \e_3^\transpose \vect{p}_{g/c}^v &= \e_3^\transpose \vect{p}_{g/b}^v -
  \e_3^\transpose R_b^I \vect{p}_{c/b}^b
  %\e_3^\transpose \vect{p}_{g/c}^v &= \frac{1}{\rho_g} -
  %\e_3^\transpose R_b^I \vect{p}_{c/b}^b \\
\end{align}

This gives us the vector, $\vect{p}_{i/c}^v$.
With this vector, we can then reach the estimated state vector,
$\vect{p}_{i/g}^g$ with the following
\begin{align}
  \vect{p}_{i/v}^v &= \vect{p}_{i/c}^v + R_b^I \vect{p}_{c/b}^b \\
  \vect{p}_{i/g}^g &= R_v^g \left( \vect{p}_{i/v}^v - \vect{p}_{g/v}^v \right).
\end{align}


% !TEX root=./root.tex

% The propsed estimator estimates both the state of
% the UAV and the state of the target vehicle as well as the positions of visual
% landmarks on the target vehicle.
% in the same Error-State Kalman
% filter~\cite{sola2017quaternion}.
% Though
% This work focuses on the estimation of the state of a landing target
% vehicle. However, we also estimate the state of the UAV to properly account for the
% uncertainty in the UAV states that appear in the motion model of the target
% vehicle and the measurement
% models used. states upon which many of the measurement models depend.
This work focuses on the estimation of the state of the landing target vehicle.
However, as several of the measurement models employed depend on the state of the
UAV, we also estimate the state of the UAV in the same filter to properly
account for the uncertainty in its state. We, therefore, estimate the position,
attitude, and velocity of the UAV together with the position, attitude, velocity,
and angular rate of the target vehicle and the positions of visual features on
the target vehicle. We also estimate bias states for the
accelerometer and gyroscope sensors that are used as inputs to the filter.
% to properly
% account for the uncertainty, as several of the 
We express the state of the estimated system as the tuple
\begin{equation}
  \hat{\x} =
  \begin{pmatrix}
    \hat{\x}_{\text{UAV}}, \hat{\x}_{\text{Goal}}, \hat{\x}_{\text{Features}}
  \end{pmatrix}
\end{equation}
with the components defined as
\begin{align}
  \hat{\vect{x}}_{\text{UAV}} &=
  \begin{pmatrix}
    \hat{\vect{p}}_{b/I}^{I}, \hat{\vect{q}}_I^{b}, \hat{\vect{v}}_{b/I}^b,
    \hat{\vect{\beta}}_a,
    \hat{\vect{\beta}}_{\omega}
  \end{pmatrix}
    \in \mathbb{R}^3 \times S^3 \times \mathbb{R}^3 \times \mathbb{R}^3 \times
    \mathbb{R}^3  \\
  % \x_{\text{UAV}} &=
  % \begin{bmatrix}
    % \vect{p}_{b/I}^I &
    % \phi & \theta & \psi &
    % \vect{v}_{b/I}^b &
    % \mu & \vect{\beta}_a & \vect{\beta}_\omega
  % \end{bmatrix}^\transpose \\
    \hat{\x}_{\text{Goal}} & =
    \begin{pmatrix}
      \hat{\vect{p}}_{g/b}^{v}, \hat{\vect{v}}_{g/I}^{g}, \hat{\psi}_{I}^{g},
      \hat{\omega}_{g/I}^{g}
    \end{pmatrix}
    \in \mathbb{R}^3 \times \mathbb{R}^2 \times \mathbb{R}^1 \times \mathbb{R}^1
    \\
    \hat{\x}_{\text{Features}} & =
    \begin{pmatrix}
      \hat{\vect{r}}_{1/g}^{g}, \dots \hat{\vect{r}}_{n/g}^{g}
    \end{pmatrix}
    \in \mathbb{R}^3 \times \dots \mathbb{R}^3.
\end{align}
The inputs to the estimated system are given by
\begin{equation}
  \vect{u} = \begin{pmatrix} \bar{\vect{a}}_{b/I}^b, \bar{\vect{\omega}}_{b/I}^b \end{pmatrix} \in
        \mathbb{R}^3 \times \mathbb{R}^3,
\end{equation}
which are directly measured from an inertial measurement unit mounted on the UAV.
% Though
% this work focuses on the estimation of the state of the landing target
% vehicle, we also estimate the state of the UAV to properly account for the
% uncertainty in these states upon which the motion model for
% $\x_{\text{Goal}}$ and many of the measurement models depend.

% The estimated states associated with the UAV, $\hat{\vect{x}}_{\text{UAV}}$,
% contain the traditionally estimated states of position, attitude, and velocity
% in addition to bias states for the accelerometer and gyroscope sensors.

% The estimated states associated with the target vehicle,
% $\hat{\vect{x}}_{\text{Goal}}$, contain the position, attitude, velocity, and
% angular velocity of the target vehicle.
We note that the
estimated position of the target vehicle, $\hat{\vect{p}}_{g/b}^v$, is relative to the position of the
UAV.
We formulate this state relatively,
as this relative state is observable even with poor estimates of the UAV's global
position, $\hat{\vect{p}}_{g/I}^I$, due to the relative information provided
by the measurements as described in~\secref{sec:measurement_models}.
% Note that $\hat{\vect{x}}_{\text{UAV}}$ contains the same states mentioned previously
% in~\secref{sec:UAV_dynamics} with the addition of $\hat{\vect{\beta}}_a$ and
% $\hat{\vect{\beta}}_\omega$, the estimated bias vectors for the acclerometer and
% gyroscope sensors. On the
% other hand, $\hat{\vect{x}}_{\text{Goal}}$ varies 
% from the target vehicle states mentioned in~\secref{sec:landing_veh_dynamics}
% by containing $\hat{\vect{p}}_{g/b}^v$ instead of $\hat{\vect{p}}_{b/I}^I$.
% % as well as $\hat{\vect{r}}_{1/g}^{g} \dots \hat{\vect{r}}_{n/g}^{g}$.
% We estimate the relative state, $\hat{\vect{p}}_{g/b}^v$, instead
% of the global state, $\hat{\vect{p}}_{g/I}^I$, as the relative state is observable
% even with poor estimates of the UAV's global position, $\hat{\vect{p}}_{b/I}^I$.

The estimated vectors $\hat{\vect{r}}_{1/g}^{g}, \dots \hat{\vect{r}}_{n/g}^{g}$ represent the
locations of visual features $1, \dots n$ with respect to the goal frame and
expressed in the goal frame. As we assume these features are rigidly attached to the
target vehicle, these vectors remain constant as the vehicle moves and rotates.
We show in the experiments described in~\secref{sec:est_paper_simulation}
and~\secref{sec:est_paper_hardware} that the
addition of only ten visual features to the estimated state significantly
improves the estimates of the state of the target vehicle
% $\hat{\x}_{\text{Goal}}$
while the fiducial landing marker is not detected. 

We also note
that the estimated state is of dynamic size. As visual features are
detected they are added to the state vector until a maximum size of the state
vector is reached. As the visual features leave the field of view of the camera,
or are otherwise no longer tracked, they are removed from the estimated state
vector to make room for new visual features to be added.

% In the following
% subsections, we define the error-state of the estimated system,  thdescribe the dynamic equations used to model the state of the
% estimated system, the
% initialization of certain states, and the specific
% measurement models used to update the filter.

% As mentioned in~\secref{sec:intro??} the estimation of these
% visual landmarks allows the estimator to maintain accurate and consistent
% estimates of the target vehicle even while the fiducial landing marker is not
% detected for long periods of time.

\subsection{Error-State Definition}

As the estimated state is not a vector, but rather a tuple of Lie groups, we
employ the error-state Kalman filter (ESKF) as described in~\cite{koch2017relative}.
% The error-state of the estimated system is defined by
We define the error-state of the estimated system as
\begin{align}
  \tilde{\vect{x}} =&
  \left[ \begin{matrix}
    \tilde{\vect{p}}_{b/I}^{I}, \tilde{\vect{\theta}}_I^{b}, \tilde{\vect{v}}_{b/I}^b,
    \tilde{\vect{\beta}}_a,
    \tilde{\vect{\beta}}_{\omega},
    \tilde{\vect{p}}_{g/b}^{v}, \tilde{\vect{v}}_{g/I}^{g}, \tilde{\psi}_{I}^{g},
    \tilde{\omega}_{g/I}^{g},
      \tilde{\vect{r}}_{1/g}^{g}, \dots \tilde{\vect{r}}_{n/g}^{g}
  \end{matrix} \right]
  \in \mathbb{R}^{22 + 3n}
\end{align}
with the error-state components related to the vector states, $\x_{\vect{v}}$, defined with
the vector subtraction operator as
\begin{equation}
\tilde{\x}_{\vect{v}} \triangleq \x_{\vect{v}} - \hat{\x}_{\vect{v}}
\label{eq:est_paper_vector_error_state}
\end{equation}
such that
\begin{equation}
  \tilde{\vect{p}}_{b/I}^I = \vect{p}_{b/I}^I - \hat{\vect{p}}_{b/I}^I.
  \label{eq:est_paper_uav_pos_err_state}
\end{equation}
We reiterate that we treat $\psi_I^g \in \mathbb{R}^1$
% as a vector space
such that
\begin{equation}
  \tilde{\psi}_I^g = \psi_I^g - \hat{\psi}_I^g.
  \label{eq:2d_att_err_state}
\end{equation}

We follow~\cite{koch2017relative}, defining the error-state of the quaternion
state,
$\q_I^b$,
as the minimal representation
\begin{equation}
  \tilde{\vect{\theta}}_I^b \triangleq \log_{\q} \left( \left( \hat{\q}_I^b \right)^{-1}
  \otimes \q_I^b \right)
  \label{eq:quat_error_state}
\end{equation}
which implies
\begin{equation}
  \q_I^b  = \hat{\q}_I^b \otimes \exp_{\q} \left( \tilde{\vect{\theta}}_I^b
  \right).
  \label{eq:quat_true_state}
\end{equation}
As rotation matrices concatenate in the order opposite to
quaternions,~\eqref{eq:quat_true_state} can also be expressed as
\begin{equation}
  R_I^b  = R \left( \exp_{\q} \left( \tilde{\vect{\theta}}_I^b \right) \right)
  \hat{R}_I^b.
  \label{eq:rot_true_state}
\end{equation}
% Similarly, we define the error-state of the 2D rotation matrix state, $R_I^g$,
% as
% \begin{equation}
  % \tilde{\psi}_I^g = \log_R \left(  
  % R_I^g \left( \hat{R}_I^g \right)^\transpose \right)
  % \label{eq:2d_rot_err_state}
% \end{equation}

To derive the error-state dynamics and the measurement residual Jacobians in the
following sections, we
use an approximation for~\eqref{eq:rot_true_state} developed by first expanding
the
quaternion exponential using~\eqref{eq:est_paper_quaternion_exp_approx} as
$\tilde{\vect{\theta}}_I^b$ is assumed to be small
\begin{equation}
  R_I^b  \approx R \left( \begin{bmatrix}
      1 \\
    \frac{1}{2} \tilde{\vect{\theta}}_I^b
  \end{bmatrix}\right)
  \hat{R}_I^b.
\end{equation}
We then employ~\eqref{eq:est_paper_R_from_q}, neglecting higher-order terms, to
yield the approximation
\begin{equation}
  R_I^b  \approx 
  \left( I - \skewmat{\tilde{\vect{\theta}}_I^b} \right)
  \hat{R}_I^b.
  \label{eq:est_paper_Rapprox}
\end{equation}
It can similarly be shown that
\begin{equation}
  \left( R_I^b \right)^\transpose  \approx 
  \left( \hat{R}_I^b \right)^\transpose
  \left( I + \skewmat{\tilde{\vect{\theta}}_I^b} \right).
  \label{eq:est_paper_RTapprox}
\end{equation}




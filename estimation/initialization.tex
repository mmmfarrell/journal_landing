% !TEX root=./root.tex

\subsection{State Initialization}
% Unlike $\hat{\vect{x}}_{\text{UAV}}$ which is initialized based on an initial
% guess of the true state, the estimated states associated with the landing
% vehicle, $\hat{\vect{x}}_{\text{Goal}}$, and the estimated states associated
% with the visual features, $\hat{\vect{x}}_{\text{Features}}$, are initialized
% upon receiving the first measurements corresponding measurements.

While the states associated with the UAV and sensor biases,
$\hat{\x}_{\text{UAV}}$, are initialized at startup,
the states associated with the target vehicle, $\hat{\x}_{\text{Goal}}$ are
initialized upon receiving the first measurement from
the detection of the fiducial marker given by
% We assume that we can measure the
% relative translation and rotation from the camera to the fiducial marker
% at each detection. This measurement,
\begin{equation}
  \bar{\vect{z}} =
  \begin{bmatrix}
    \bar{\vect{p}}_{g/c}^c & \bar{R}_c^g
  \end{bmatrix}.
  \label{eq:fiducial_meas}
\end{equation}
We use this measurement to initialize the estimates of the states with
% is used to initialize the goal states with
\begin{align}
  \hat{\vect{p}}_{g/b}^v &= \left( \hat{R}_I^b \right)^\transpose \left( \left ( R_b^c
  \right)^\transpose \bar{\vect{p}}_{g/c}^c + \vect{p}_{c/b}^b
\right)   \\
      \hat{\vect{v}}_{g/I}^I &= \vect{0} \\
      \hat{\psi_I^g} &= \text{yaw} \left( \bar{R}_c^g R_b^c \hat{R}_I^b \right)
      \\
      \hat{\omega_{g/I}^g} &= 0.
  % quat::Quatd q_I2g_meas = x().q * q_b2c_ * z.q_c2a * q_a2g;
  % const double yaw_meas = q_I2g_meas.euler()(2);
\end{align}
% where $R_b^c$ and $\vect{p}_{c/b}^b$ are assumed to be known constants and
% $yaw()$ is a function that extracts the yaw Euler angle from a rotation matrix.

% The linear and angular velocity of the landing vehicle are initialized to
% zero with large enough covariances for the specific use case.

Similar to the manner in which the target vehicle states are intialized,
the visual feature states are initialized based on the first corresponding measurement received.
% each time a new
% landmark is added to the estimated vector, we initialize its state based on the
% first measurement received.
As we only receive a measurement of the pixel
location of the visual feature in the camera image,
\begin{equation}
  \bar{\vect{z}} = \begin{bmatrix} \bar{p}_x & \bar{p}_y \end{bmatrix}
  \label{eq:pixel_meas},
\end{equation}
the estimated state, $\hat{\vect{r}}_{i/g}^g$, is not entirely observed.
Therefore, to initialize
the feature state, we assume the feature lies in the $xy$ plane of the
goal frame, initializing the $z$ component of $\hat{\vect{r}}_{i/g}^g$ to zero.
% with a large enough covariance for the specific landing vehicle.
To compute the
$x$ and $y$ components of $\hat{\vect{r}}_{i/g}^g$,
% we start with the pinhole camera
% model.
% We assume that the pixel measurements in~\eqref{eq:pixel_meas} have been
% rectified to compensate for lens distortion such that
% \begin{equation}
  % \begin{bmatrix}
    % p_x \\ p_y \\ 1
  % \end{bmatrix} = \frac{1}{\e_3^\transpose \vect{p}_{i/c}^c} K \vect{p}_{i/c}^c \\
% \end{equation}
% where $K$ is the camera intrisic matrix.
we start with~\eqref{eq:pinhole_camera}
% If we invert $K$ and multiply by both
% sides, we get
% \begin{equation}
 % \frac{1}{\e_3^\transpose \vect{p}_{i/c}^c} \vect{p}_{i/c}^c
  % =
  % K^{-1} \begin{bmatrix}
    % p_x \\ p_y \\ 1
  % \end{bmatrix}
% \end{equation}
which can be rotated and solved to yield
\begin{equation}
 \vect{p}_{i/c}^v
  =
  \left( \e_3^\transpose \vect{p}_{i/c}^c \right) \left( R_I^b
  \right)^\transpose \left( R_b^c \right)^\transpose K^{-1} \begin{bmatrix}
    \bar{p}_x \\ \bar{p}_y \\ 1
  \end{bmatrix}.
\end{equation}
As $\e_3^\transpose \vect{p}_{i/c}^c$ is unknown, we define 
\begin{equation}
  \scaled{\vect{p}}_{i/c}^v
  % =
  \triangleq
   \left( R_I^b
  \right)^\transpose \left( R_b^c \right)^\transpose K^{-1} \begin{bmatrix}
    \bar{p}_x \\ \bar{p}_y \\ 1
  \end{bmatrix},
\end{equation}
which is equivalent to $\vect{p}_{i/c}^v$ up to a scale factor. As previously
mentioned, we estimate this scale factor by assuming $\e_3^\transpose
\vect{r}_{i/g}^g = 0$ such that
\begin{equation}
  \e_3^\transpose \hat{\vect{p}}_{i/c}^v = \e_3^\transpose \hat{\vect{p}}_{g/b}^v -
  \e_3^\transpose \left( \hat{R}_I^b \right)^\transpose \vect{p}_{c/b}^b.
\end{equation}
We therefore initialize
\begin{align}
    \hat{\vect{r}}_{i/g}^g &= \hat{R}_I^g \left( \hat{\vect{p}}_{i/b}^v -
    \hat{\vect{p}}_{g/b}^v \right)
\end{align}
where
% This is used to solve for $\hat{\vect{r}}_{i/g}^g$ with
\begin{align}
  % \hat{\vect{p}}_{i/c}^v &= \frac{\e_3^\transpose
  % \hat{\vect{p}}_{i/c}^v}{\e_3^\transpose \scaled{\vect{p}}_{i/c}^v} \scaled{\vect{p}}_{i/c}^v
  % \\
    \hat{\vect{p}}_{i/b}^v &= \frac{\e_3^\transpose
  \hat{\vect{p}}_{i/c}^v}{\e_3^\transpose \scaled{\vect{p}}_{i/c}^v}
  \scaled{\vect{p}}_{i/c}^v+ \left( \hat{R}_I^b \right)^\transpose
  \vect{p}_{c/b}^b.
    % \hat{\vect{p}}_{i/b}^v &= \hat{\vect{p}}_{i/c}^v + \left( R_I^b \right)^\transpose \vect{p}_{c/b}^b \\
    % \hat{\vect{r}}_{i/g}^g &= \hat{R}_I^g \left( \hat{\vect{p}}_{i/b}^v -
    % \hat{\vect{p}}_{g/b}^v \right).
\end{align}

% \begin{align}
  % \begin{bmatrix}
    % p_x \\ p_y \\ 1
  % \end{bmatrix} &= \frac{1}{\e_3^\transpose \vect{p}_{i/c}^c} K \vect{p}_{i/c}^c \\
  % \begin{bmatrix}
    % X^c / Z^c \\
    % Y^c / Z^c \\
    % 1
  % \end{bmatrix}
   % &= K^{-1}
  % \begin{bmatrix}
    % p_x \\ p_y \\ 1
  % \end{bmatrix} \\
  % \begin{bmatrix}
    % X^c / Z^c \\
    % Y^c / Z^c \\
    % 1
  % \end{bmatrix}^v
   % &= R_b^v R_c^b K^{-1}
  % \begin{bmatrix}
    % p_x \\ p_y \\ 1
  % \end{bmatrix} \\
% \end{align}
% The resulting vector, $
  % \begin{bmatrix}
    % X^c / Z^c \\
    % Y^c / Z^c \\
    % 1
  % \end{bmatrix}^v$
  % is the vector $\vect{p}_{i/c}^v$ up to a scale factor. The vector can be scaled by assuming that the altitude of the landmark is equal to the
% altitude of the goal. To get the expected altitude, we solve for
% \begin{align}
  % \e_3^\transpose \vect{p}_{g/c}^v &= \e_3^\transpose \vect{p}_{g/b}^v - \e_3^\transpose \vect{p}_{c/b}^v \\
  % \e_3^\transpose \vect{p}_{g/c}^v &= \e_3^\transpose \vect{p}_{g/b}^v -
  % \e_3^\transpose R_b^I \vect{p}_{c/b}^b \\
  % \e_3^\transpose \vect{p}_{g/c}^v &= \frac{1}{\rho_g} -
  % \e_3^\transpose R_b^I \vect{p}_{c/b}^b \\
% \end{align}

% This gives us the vector, $\vect{p}_{i/c}^v$.
% With this vector, we can then reach the estimated state vector,
% $\vect{p}_{i/g}^g$ with the following
% \begin{align}
  % \vect{p}_{i/v}^v &= \vect{p}_{i/c}^v + R_b^I \vect{p}_{c/b}^b \\
  % \vect{p}_{i/g}^g &= R_v^g \left( \vect{p}_{i/v}^v - \vect{p}_{g/v}^v \right).
% \end{align}


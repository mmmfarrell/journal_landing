% !TEX root=./root.tex

\subsection{Measurement Models}

\subsubsection{Global UAV Position Measurement}
We assume to have some measurement of the position of the UAV with respect to
the inertial
frame. This measurement may come from a sensor such as GPS or a motion capture system.
The measurement model and its estimate can
be written as
\begin{align}
  h_{\text{pos}} \left( \x \right) &= \vect{p}_{b/I}^I +
  \vect{\eta}_{\text{pos}} \\
  h_{\text{pos}} \left( \hat{\x} \right) &= \hat{\vect{p}}_{b/I}^I.
\end{align}
For a given measurement of position, $\vect{z}_{\text{pos}}$, the residual is
\begin{equation*}
  \vect{r}_{\text{pos}} = \vect{z}_{\text{pos}} - h_{\text{pos}} \left( \hat{\x}
  \right),
\end{equation*}
which for the error-state Kalman filter is modeled as
\begin{align*}
  \vect{r}_{\text{pos}} &=  h_{\text{pos}} \left( \x \right) - h_{\text{pos}} \left( \hat{\x}
  \right), \\
                        &= \vect{p}_{b/I}^I + \eta_{\text{pos}} -
                        \hat{\vect{p}}_{b/I}^I \\
                        &= \tilde{\vect{p}}_{b/I}^I + \eta_{\text{pos}}.
\end{align*}
This results in the measurement jacobian
\begin{equation*}
  H_{\text{pos}} =
  \begin{bmatrix}
    I_{3 \times 3} & \vect{0} & \vect{0} & \dots & \vect{0}
  \end{bmatrix}.
\end{equation*}

% and the non-zero component of the jacobian of the measurement model as
% \begin{equation}
  % \frac{\partial \hat{\vect{p}}_{b/I}^I}{\partial \vect{p}_{b/I}^I} = I_{3
  % \times 3}.
% \end{equation}

\subsubsection{Global UAV Attitude Measurement}
Similar to the position measurement above, we assume to have some measurement of
the attitude of the UAV with respect to the inertial frame. This measurement may
come from an attitude and heading reference system (AHRS) or from a motion
capture system. 
The measurement model and its estimate can
be written as
\begin{align}
  h_{\text{att}} \left( \x \right) &= \vect{q}_{I}^b \otimes \exp_{\q} \left(
  \vect{\eta}_{\text{att}} \right) \\
    h_{\text{att}} \left( \hat{\x} \right) &= \hat{\vect{q}}_{I}^b.
\end{align}
For a given measurement of attitude, $\vect{z}_{\text{att}}$, the residual is
given by
\begin{equation*}
  \vect{r}_{\text{att}} = \log_{\q} \left(  h_{\text{att}} \left(
  \hat{\x} \right)^{-1} \otimes \vect{z}_{\text{att}} \right)
  % \vect{r}_{\text{att}} = \vect{z}_{\text{att}} - h_{\text{att}} \left( \hat{\x}
\end{equation*}
which is modeled and simplified using~\eqref{eq:quat_true_state}
\begin{align*}
  \vect{r}_{\text{att}} &= \log_{\q} \left(  h_{\text{att}} \left(
  \hat{\x} \right)^{-1} \otimes h_{\text{att}} \left( \x \right) \right) \\
                        &= \log_{\q} \left(  \left(
  \hat{\vect{q}}_{I}^b \right)^{-1} \otimes \vect{q}_{I}^b \otimes \exp_{\q} \left(
  \vect{\eta}_{\text{att}} \right)\right) \\
                        &= \log_{\q} \left(  \left(
                        \hat{\vect{q}}_{I}^b \right)^{-1} \otimes
                        \hat{\vect{q}}_{I}^b \otimes \exp_{\q} \left(
                      \tilde{\vect{\theta}}_I^b \right) \right)
                          + \vect{\eta}_{\text{att}}  \\
                        &= \tilde{\vect{\theta}}_I^b
                          + \vect{\eta}_{\text{att}}. 
\end{align*}
This results in the measurement jacobian
\begin{equation*}
  H_{\text{pos}} =
  \begin{bmatrix}
    \vect{0} & I_{3 \times 3} & \vect{0} & \dots & \vect{0}
  \end{bmatrix}.
\end{equation*}

\subsubsection{Fiducial Translation Measurement}
We assume that a known fiducial marker serves as the desired landing position
for the multirotor UAV on the landing vehicle. The goal frame is therefore
located at the center of the fiducial marker. This means that every detection of the fiducial
marker yields a measurement of the relative translation and rotation from the
camera frame to the goal frame as noted in~\eqref{eq:fiducial_meas}.
The measurement model and its estimate
therefore can be written as
\begin{align*}
  h_{\text{ft}} \left( \x \right) &=
  \vect{p}_{g/c}^c + \vect{\eta}_{\text{ft}} \\
  &= R_b^c \left( R_I^b \vect{p}_{g/b}^v -
  \vect{p}_{c/b}^b \right) + \vect{\eta}_{\text{ft}} \\
  h_{\text{ft}} \left( \hat{\x} \right) &=
    \hat{\vect{p}}_{g/c}^c \\
  &= R_b^c \left( \hat{R}_I^b \hat{\vect{p}}_{g/b}^v -
    \vect{p}_{c/b}^b \right). 
  % \hat{\vect{q}}_{c}^a  &= R_g^a \hat{R}_I^g \hat{R}_b^I R_c^b
\end{align*}
For a given measurement of the relative translation to the fiducial marker,
$\vect{z}_{\text{ft}}$, the residual is given by
\begin{equation*}
  \vect{r}_{\text{ft}} = \vect{z}_{\text{ft}} - h_{\text{ft}} \left( \hat{\x}
  \right),
\end{equation*}
which is modeled as
\begin{align}
  \vect{r}_{\text{ft}} &=  h_{\text{ft}} \left( \x \right) - h_{\text{ft}} \left( \hat{\x}
  \right), \\
                       &= R_b^c \left( R_I^b \vect{p}_{g/b}^v -
                         \vect{p}_{c/b}^b \right)  +\eta_{\text{rt}} - R_b^c \left( \hat{R}_I^b \hat{\vect{p}}_{g/b}^v -
    \vect{p}_{c/b}^b \right)  \\
                       &= R_b^c R_I^b \vect{p}_{g/b}^v 
                          - R_b^c \hat{R}_I^b \hat{\vect{p}}_{g/b}^v +
                          \eta_{\text{rt}}.
                          \label{eq:rft_1}
\end{align}
We note that from~\eqref{eq:quat_true_state},
\begin{equation*}
  R_I^b = R \left( \exp_{\q} \left( \tilde{\vect{\theta}}_I^b \right) \right)
  \hat{R}_I^b,
\end{equation*}
which can be simplifed using~\eqref{eq:quaternion_exp_approx}
and~\eqref{eq:R_from_q} to show that
\begin{equation}
  R_I^b \approx \left( I - \skewmat{\tilde{\vect{\theta}}_I^b }\right) \hat{R}_I^b.
  \label{eq:expand_R}
\end{equation}
We can similarly show that 
\begin{equation}
  \left( R_I^b \right)^\transpose \approx \left( \hat{R}_I^b \right)^\transpose
  \left( I + \skewmat{\tilde{\vect{\theta}}_I^b }\right).
\end{equation}
Using~\eqref{eq:expand_R}, we can expand~\eqref{eq:rft_1} and ignore
second-order terms to yield
\begin{align*}
   \vect{r}_{\text{ft}} &\approx R_b^c \left( I - \skewmat{\tilde{\vect{\theta}}_I^b }\right) \hat{R}_I^b \left( \hat{\vect{p}}_{g/b}^v + \tilde{\vect{p}}_{g/b}^v \right) 
                          - R_b^c \hat{R}_I^b \hat{\vect{p}}_{g/b}^v +
                          \eta_{\text{rt}} \\
  &\approx R_b^c \hat{R}_I^b \tilde{\vect{p}}_{g/b}^v -
  R_b^c \skewmat{\tilde{\vect{\theta}}_I^b } \hat{R}_I^b \hat{\vect{p}}_{g/b}^v 
      + \eta_{\text{rt}} \\
&= R_b^c \hat{R}_I^b \tilde{\vect{p}}_{g/b}^v + R_b^c \skewmat{\hat{R}_I^b \hat{\vect{p}}_{g/b}^v} \tilde{\vect{\theta}}_I^b 
      + \eta_{\text{rt}}.
\end{align*}
This results in the measurement jacobian
\begin{equation*}
  H_{\text{pos}} =
  \begin{bmatrix}
    \vect{0} & R_b^c \skewmat{ \hat{R}_I^b \hat{\vect{p}}_{g/b}^v } & \vect{0} &
    \vect{0} & \vect{0} & R_b^c \hat{R}_I^b & \vect{0} & \dots & \vect{0}
  \end{bmatrix}.
\end{equation*}


% The non-zero components of the jacobian of the measurement model used for the
% Kalman filter update are

% \begin{align}
  % \frac{\partial \hat{\vect{p}}_{g/c}^c}{\partial \vect{q}_I^b} =& R_b^c 
  % \hat{R}_I^b \skewmat{\hat{\vect{p}}_{g/b}^v} \\
    % \frac{\partial \hat{\vect{p}}_{g/c}^c}{\partial \vect{p}_{g/b}^v} =& R_b^c
    % \hat{R}_I^b.
% \end{align}

\subsubsection{Fidual Rotation Measurement}
We use the relative rotation measurement that results from a detection of the
fiducial marker
to create a pseudo measurement of the orientation of the goal frame. This pseudo
measurement is created with
\begin{equation}
  \bar{\theta}_I^g = yaw \left( \bar{R}_c^g R_b^c \hat{R}_I^b \right).
\end{equation}
The measurement model and its estimate can
be written as
\begin{align}
  h_{\text{fr}} \left( \x \right) &= \theta_I^g + \eta_{\text{fr}} \\
  h_{\text{fr}} \left( \hat{\x} \right) &= \hat{\theta}_I^g.
\end{align}
For a given measurement of the rotation of the fiducial marker, $z_{\text{fr}}$,
the residual is given by
\begin{equation}
  r_{\text{fr}} = z_{\text{fr}} - h_{\text{fr}} \left( \hat{\x} \right),
\end{equation}
which is modeled as
\begin{align}
  r_{\text{fr}} &= h_{\text{fr}} \left( \x \right) - h_{\text{fr}} \left( \hat{\x} \right) \\
                &= \theta_I^g + \eta_{\text{fr}} - \hat{\theta}_I^g \\
                &= \tilde{\theta}_I^g + \eta_{\text{fr}}.
\end{align}
This results in the measurement jacobian
\begin{equation*}
  H_{\text{pos}} =
  \begin{bmatrix}
    \vect{0} & \vect{0} & \vect{0} &
    \vect{0} & \vect{0} & \vect{0} & \vect{0} & 1 & \vect{0} & \dots & \vect{0}
  \end{bmatrix}.
\end{equation*}

% and the non-zero jacobians of the measurement model as
% \begin{equation}
  % \frac{\partial \hat{\theta}_I^g}{\partial \theta_I^g} = 1.
% \end{equation}

\subsubsection{Landmark Pixel Measurement}
As described in~\eqref{eq:pixel_meas}, the estimator receives measurements of
the location of each visual landmark in the camera image. We assume that the
pixel measurements received have already been corrected for lens distortion.
Using the pinhole camera model, we therefore express the pixel locations as
\begin{align}
  % \hat{h} &=
  % \begin{bmatrix}
    % \hat{p}_x & \hat{p}_y
  % \end{bmatrix}^\transpose \\
  \begin{bmatrix}
    p_x \\ p_y \\ 1
  \end{bmatrix} &= \frac{1}{\e_3^\transpose \vect{p}_{i/c}^c} K
  \vect{p}_{i/c}^c,
  % \vect{z} &=
  % \begin{bmatrix}
    % f_x \frac{\e_1 \vect{p}_{i/c}^c}{\e_3 \vect{p}_{i/c}^c} + c_x \\
    % f_y \frac{\e_2 \vect{p}_{i/c}^c}{\e_3 \vect{p}_{i/c}^c} + c_y
  % \end{bmatrix},
\end{align}
where K is the camera intrisic matrix and 
\begin{align}
  \vect{p}_{i/c}^c = R_b^c \left( R_I^b \left( R_I^g \right)^\transpose
  \vect{r}_{i/g}^g + R_I^b \vect{p}_{g/b}^v - \vect{p}_{c/b}^b \right).
  \label{eq:p_i_c_c}
\end{align}
The measurement model and its estimate can be written as
\begin{align}
  h_{\text{pix}} \left( \x \right)
  &= \begin{bmatrix} p_x & p_y \end{bmatrix}^\transpose \\
  &= \frac{1}{\e_3^\transpose \vect{p}_{i/c}^c} I_{2 \times 3} K
  \vect{p}_{i/c}^c + \vect{\eta}_{\text{pix}} \\
  h_{\text{pix}} \left( \hat{\x} \right)
  &= \begin{bmatrix} \hat{p}_x & \hat{p}_y \end{bmatrix}^\transpose \\
  &= \frac{1}{\e_3^\transpose \hat{\vect{p}}_{i/c}^c} I_{2 \times 3} K
  \hat{\vect{p}}_{i/c}^c
\end{align}
where
\begin{align}
  \hat{\vect{p}}_{i/c}^c = R_b^c \left( \hat{R}_I^b \left( \hat{R}_I^g \right)^\transpose
  \hat{\vect{r}}_{i/g}^g + \hat{R}_I^b \hat{\vect{p}}_{g/b}^v - \vect{p}_{c/b}^b
\right).
\end{align}
For a given measurement of the pixel location of a landmark,
$\vect{z}_{\text{pix}}$, the residual is given by
\begin{equation}
  \vect{r}_{\text{pix}} = \vect{z}_{\text{pix}} - h_{\text{pix}} \left( \hat{\x}
    \right),
\end{equation}
which is modeled as
\begin{align}
  \vect{r}_{\text{pix}} &= h_{\text{pix}} \left( \x \right) - h_{\text{pix}} \left( \hat{\x}
    \right) \\
  &= \frac{1}{\e_3^\transpose \vect{p}_{i/c}^c} I_{2 \times 3} K
  \vect{p}_{i/c}^c + \vect{\eta}_{\text{pix}} - \frac{1}{\e_3^\transpose \hat{\vect{p}}_{i/c}^c} I_{2 \times 3} K
  \hat{\vect{p}}_{i/c}^c.
\end{align}
To compute the measurement jacobian, we first expand~\eqref{eq:p_i_c_c}
\MF{This measurement jac is very crazy and the derivation may not be 100~\% correct}
\begin{align}
  \vect{p}_{i/c}^c &= R_b^c \left( \left( I - \skewmat{\tilde{\vect{\theta}}_I^b
    } \right) \hat{R}_I^b \left( I + \skewmat{\tilde{\vect{\theta}}_I^g }
  \right) \left( \hat{R}_I^g \right)^\transpose
\left( \hat{\vect{r}}_{i/g}^g  + \tilde{\vect{r}}_{i/g}^g \right) + \left( I - \skewmat{\tilde{\vect{\theta}}_I^b
} \right) \hat{R}_I^b \left( \hat{\vect{p}}_{g/b}^v + \tilde{\vect{p}}_{g/b}^v \right) - \vect{p}_{c/b}^b \right).
\end{align}
By simplifying and ignoring secord-order terms,
\begin{align*}
  \vect{p}_{i/c}^c &= R_b^c \left( \hat{R}_I^b \left( \hat{R}_I^g \right)^\transpose
  \hat{\vect{r}}_{i/g}^g + \hat{R}_I^b \hat{\vect{p}}_{g/b}^v - \vect{p}_{c/b}^b
  - \skewmat{\tilde{\vect{\theta}}_I^b} \hat{R}_I^b \left( \hat{R}_I^g
    \right)^\transpose \hat{\vect{r}}_{i/g}^g + \hat{R}_I^b
    \skewmat{\tilde{\vect{\theta}}_I^g} \left( \hat{R}_I^g
  \right)^\transpose \hat{\vect{r}}_{i/g}^g +
\hat{R}_I^b \left( \hat{R}_I^g
    \right)^\transpose \tilde{\vect{r}}_{i/g}^g \right. \\
                   &- \left. \skewmat{\tilde{\vect{\theta}}_I^b} \hat{R}_I^b \hat{\vect{p}}_{g/b}^v
    + \hat{R}_I^b \tilde{\vect{p}}_{g/b}^v
   \right). \\
    &= \hat{\vect{p}}_{i/c}^c
    + R_b^c \left( 
  - \skewmat{\tilde{\vect{\theta}}_I^b} \hat{R}_I^b \left( \hat{R}_I^g
    \right)^\transpose \hat{\vect{r}}_{i/g}^g + \hat{R}_I^b
    \skewmat{\tilde{\vect{\theta}}_I^g} \left( \hat{R}_I^g
  \right)^\transpose \hat{\vect{r}}_{i/g}^g +
\hat{R}_I^b \left( \hat{R}_I^g
    \right)^\transpose \tilde{\vect{r}}_{i/g}^g
                   - \skewmat{\tilde{\vect{\theta}}_I^b} \hat{R}_I^b \hat{\vect{p}}_{g/b}^v
    + \hat{R}_I^b \tilde{\vect{p}}_{g/b}^v
   \right), 
\end{align*}
which we define as
\begin{equation*}
  \vect{p}_{i/c}^c = \hat{\vect{p}}_{i/c}^c + \tilde{\vect{p}}_{i/c}^c .
\end{equation*}
For simplificty we note that 
\begin{equation}
  H_{\text{pix}} = \frac{\partial}{\partial \tilde{\x}} \vect{r}_{\text{pix}}
\end{equation}
and that
\begin{align*}
  \frac{\partial}{\partial \tilde{\x}} \left( \frac{1}{\e_3^\transpose \hat{\vect{p}}_{i/c}^c} I_{2 \times 3} K
  \hat{\vect{p}}_{i/c}^c \right) &= \vect{0}
\\
  \frac{\partial}{\partial \tilde{\x}} \vect{\eta}_{\text{pix}} &= \vect{0}
\end{align*}
such that
\begin{equation}
  H_{\text{pix}} = \frac{\partial}{\partial \tilde{\x}}
 \left( \frac{1}{\e_3^\transpose \vect{p}_{i/c}^c} I_{2 \times 3} K
 \vect{p}_{i/c}^c \right).
\end{equation}
We expand this using the chain rule to show that
\begin{align}
  H_{\text{pix}} &= \frac{\partial}{\partial \tilde{\x}}
 \left( \frac{1}{\e_3^\transpose \vect{p}_{i/c}^c} I_{2 \times 3} K
 \vect{p}_{i/c}^c \right) \\
  &= 
 \frac{1}{\e_3^\transpose \vect{p}_{i/c}^c} I_{2 \times 3} K
 \frac{\partial}{\partial \tilde{\x}} \vect{p}_{i/c}^c 
 - \frac{\e_3^\transpose \frac{\partial}{\partial \tilde{\x}} 
 \vect{p}_{i/c}^c}{\left( \e_3^\transpose
 \vect{p}_{i/c}^c \right)^2 } I_{2 \times 3} K
  \vect{p}_{i/c}^c. 
\end{align}
Using the equation above,
\begin{align*}
  \frac{\partial}{\partial \tilde{\vect{\theta}}_I^b } \vect{p}_{i/c}^c
  &=
  R_b^c \left( \skewmat{} \right) \\
  \frac{\partial}{\partial \tilde{\vect{p}}_{g/b}^v} \vect{p}_{i/c}^c
  &=
  R_b^c \hat{R}_I^b \\
  \frac{\partial}{\partial \tilde{\theta}_I^g} \vect{p}_{i/c}^c
  &=
  0 \\
  \frac{\partial}{\partial \tilde{\vect{r}}_{i/c}^c} \vect{p}_{i/c}^c
  &=
  0 \\
\end{align*}


% where $f_x$, $f_y$, $c_x$, and $c_y$ are constant parameters of the camera. We
% can then express the position of the landmark $i$ with respect to the camera in the
% camera frame as
% \begin{align}
  % \vect{p}_{i/c}^c &= R_b^c R_v^b \left(\vect{p}_{i/v}^v -
    % \vect{p}_{c/v}^v\right) \\
    % \vect{p}_{i/c}^c &= R_b^c \left(R_v^b \vect{p}_{i/v}^v -
    % \vect{p}_{c/b}^b\right)
% \end{align}
% This is actually a little bit weird because of the state parameters, but
% \begin{equation}
  % \vect{p}_{i/v}^v =
  % \begin{bmatrix}
    % \e_1^\transpose \left( R_g^v \vect{p}_{i/g}^g + \vect{p}_{g/v}^v \right) \\
    % \e_2^\transpose \left( R_g^v \vect{p}_{i/g}^g + \vect{p}_{g/v}^v \right) \\
    % \e_3^\transpose \left( \vect{p}_{i/g}^g - \vect{p}_{b/I}^I \right)
  % \end{bmatrix}.
% \end{equation}

% Let us derive the measurement jacobians.
% \begin{align}
  % \frac{\partial}{\partial \x} \hat{h} &= \frac{\e_3^\transpose \hat{\vect{p}}_{i/c}^c
  % \frac{\partial}{\partial \x} K \hat{\vect{p}}_{i/c}^c - K \hat{\vect{p}}_{i/c}^c
  % \frac{\partial}{\partial \x} \e_3^\transpose \hat{\vect{p}}_{i/c}^c}{\left(
% \e_3^\transpose \hat{\vect{p}}_{i/c}^c \right)^2 } \\
  % \frac{\partial}{\partial \x} \hat{h} &= \frac{\e_3^\transpose \hat{\vect{p}}_{i/c}^c
  % K \frac{\partial}{\partial \x} \hat{\vect{p}}_{i/c}^c - K \hat{\vect{p}}_{i/c}^c
  % \e_3^\transpose \frac{\partial}{\partial \x} \hat{\vect{p}}_{i/c}^c}{\left(
% \e_3^\transpose \hat{\vect{p}}_{i/c}^c \right)^2 }
% \end{align}
% \begin{align}
  % \frac{\partial}{\partial \vect{p}_{g/b}^v} \hat{\vect{p}}_{i/c}^c &=
  % R_b^c \hat{R}_I^b \\
  % \frac{\partial}{\partial \vect{q}_{I}^b} \hat{\vect{p}}_{i/c}^c &=
  % -R_b^c \hat{R}_I^b \skewmat{ \left( R_I^g \right)^\transpose
  % \hat{\vect{r}}_{i/g}^g + \hat{\vect{p}}_{g/b}^v } \\
  % \frac{\partial}{\partial \theta_I^g} \hat{\vect{p}}_{i/c}^c &=
  % R_b^c \hat{R}_I^b \\
% \end{align}

% % \subsubsection{Measurement Jacobian}
% For the purpose of deriving the jacobians, we will just look at the jacobians
% with respect to $p_x$, the pixel location along the $x$ axis of the camera
% frame. From above we have that
% \begin{align}
  % p_x &= f_x \frac{\e_1^\transpose \vect{p}_{i/c}^c}{\e_3^\transpose \vect{p}_{i/c}^c} + c_x \\ 
  % p_x &= f_x \frac{\e_1^\transpose R_b^c \left(R_v^b \vect{p}_{i/v}^v -
      % \vect{p}_{c/b}^b\right) }{\e_3^\transpose R_b^c \left(R_v^b \vect{p}_{i/v}^v -
  % \vect{p}_{c/b}^b\right) } + c_x 
% \end{align}
% Note that all values are constants except for $\vect{p}_{i/v}^v$ and $R_v^b$.
% The individual parts of the jacobian are given here:
% \begin{equation}
  % \frac{\partial p_x}{\partial \vect{p}_{g/v}^v} =
  % \frac{f_x \e_1 R_b^c R_v^b}
    % {\left(\e_3 R_b^c \left(R_v^b \vect{p}_{i/v}^v -
    % \vect{p}_{c/b}^b\right)\right)}
    % - \frac{\left(\e_3 R_b^c R_v^b \right) f_x \left(\e_1 R_b^c \left(R_v^b \vect{p}_{i/v}^v -
        % \vect{p}_{c/b}^b\right)\right)} {\left(\e_3 R_b^c \left(R_v^b \vect{p}_{i/v}^v -
  % \vect{p}_{c/b}^b\right)\right)^2}
% \end{equation}
% where only the first two (of three) entries of the vector are used. The third
% entry is used for the jacobian w.r.t. $\vect{p}_{b/I}^I(2)$ as
% \begin{equation}
  % \frac{\partial p_x}{\partial \vect{p}_{b/I}^I(2)} = -\frac{\partial p_x}{\partial
  % \vect{p}_{g/v}^v}\left(2\right).
% \end{equation}
% The derivatives for the attitude representation are less straight forward. From
% "Micro Lie Theory" we use the fact that the jacobian of the rotation action can
% be shown to be
% \begin{equation}
  % J_R^{R \cdot v} = -R \skewmat{v}.
% \end{equation}
% \begin{equation}
  % \frac{\partial p_x}{\partial \q} =
  % \frac{-f_x \e_1 R_b^c R_v^b \skewmat{\vect{p}_{i/v}^v}}
    % {\left(\e_3 R_b^c \left(R_v^b \vect{p}_{i/v}^v -
    % \vect{p}_{c/b}^b\right)\right)}
    % - \frac{\left(-\e_3 R_b^c R_v^b \skewmat{\vect{p}_{i/v}^v} \right) f_x \left(\e_1 R_b^c \left(R_v^b \vect{p}_{i/v}^v -
        % \vect{p}_{c/b}^b\right)\right)} {\left(\e_3 R_b^c \left(R_v^b \vect{p}_{i/v}^v -
  % \vect{p}_{c/b}^b\right)\right)^2}
% \end{equation}
% where the jacobians for the other angles are similar, just substituting in.
% The derivative for the goal angle, $\theta_g$ is given by using
% \begin{equation}
  % \frac{\partial \vect{p}_{i/v}^v}{\partial \theta_g} =
  % \begin{bmatrix}
    % \e_1^\transpose \frac{\partial R_g^v}{\partial \theta_g} \vect{p}_{i/g}^g \\
    % \e_2^\transpose \frac{\partial R_g^v}{\partial \theta_g} \vect{p}_{i/g}^g \\
    % 0
  % \end{bmatrix}
% \end{equation}
% in the jacobian given by
% \begin{equation}
  % \frac{\partial p_x}{\partial \theta_g} =
  % \frac{f_x \e_1 R_b^c R_v^b \frac{\partial \vect{p}_{i/v}^v}{\partial \theta_g}}
    % {\left(\e_3 R_b^c \left(R_v^b \vect{p}_{i/v}^v -
    % \vect{p}_{c/b}^b\right)\right)}
    % - \frac{\left(\e_3 R_b^c R_v^b \frac{\partial \vect{p}_{i/v}^v}{\partial \theta_g} \right) f_x \left(\e_1 R_b^c \left(R_v^b \vect{p}_{i/v}^v -
        % \vect{p}_{c/b}^b\right)\right)} {\left(\e_3 R_b^c \left(R_v^b \vect{p}_{i/v}^v -
  % \vect{p}_{c/b}^b\right)\right)^2}
% \end{equation}
% Similarly, the derivative for the landmark offset, $\vect{r}_i$ or
% $\vect{p}_{i/g}^g$ is given by using
% \begin{equation}
  % \frac{\partial \vect{p}_{i/v}^v}{\partial \vect{p}_{i/g}^g} =
  % \begin{bmatrix}
    % \e_1^\transpose R_g^v \\
    % \e_2^\transpose R_g^v \\
    % \e_3^\transpose
  % \end{bmatrix}
% \end{equation}
% in the jacobian given by
% \begin{equation}
  % \frac{\partial p_x}{\partial \vect{p}_{i/g}^g} =
  % \frac{f_x \e_1 R_b^c R_v^b \frac{\partial \vect{p}_{i/v}^v}{\partial \vect{p}_{i/g}^g}}
    % {\left(\e_3 R_b^c \left(R_v^b \vect{p}_{i/v}^v -
    % \vect{p}_{c/b}^b\right)\right)}
    % - \frac{\left(\e_3 R_b^c R_v^b \frac{\partial \vect{p}_{i/v}^v}{\partial \vect{p}_{i/g}^g} \right) f_x \left(\e_1 R_b^c \left(R_v^b \vect{p}_{i/v}^v -
        % \vect{p}_{c/b}^b\right)\right)} {\left(\e_3 R_b^c \left(R_v^b \vect{p}_{i/v}^v -
  % \vect{p}_{c/b}^b\right)\right)^2}
% \end{equation}
% The jacobians for the y pixel measurement, $p_y$ are similar to the ones derived
% above, just changing $f_y$ in place of $f_x$ and $\e_2$ instead of $\e_1$.


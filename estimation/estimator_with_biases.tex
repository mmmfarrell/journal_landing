% !TEX root=./root.tex

As mentioned in~\secref{sec:TODO}, we estimate the state of the UAV and the
state of the landing vehicle in the same EKF.
The state of the combined system can be expressed as
\begin{equation}
  \x =
  \begin{bmatrix}
    \left(\x_{\text{UAV}}\right)^\transpose & \left(\x_{\text{Goal}}\right)^\transpose
  \end{bmatrix}^\transpose
\end{equation}
with the components defined as
\begin{align}
  \x_{\text{UAV}} &=
  \begin{bmatrix}
    \vect{p}_{b/I}^I &
    \phi & \theta & \psi &
    \vect{v}_{b/I}^b &
    \mu & \vect{\beta}_a & \vect{\beta}_\omega
  \end{bmatrix}^\transpose \\
  \x_{\text{Goal}} & =
    \begin{bmatrix}
      \vect{p}_{g/v}^{v} & \vect{v}_{g/I}^{g} & \theta_{I}^{g} &
      \omega_{g/I}^{g} &
      \vect{r}_{1/g}^{g} & \dots & \vect{r}_{n/g}^{g}
    \end{bmatrix}^{\transpose}.
\end{align}
The UAV states are self explanatory, covering position, attitude and velocity.
The Goal states, however, are a little more unique. They are described here:
\begin{itemize}
  \item $\vect{p}_{g/v}^{v}$ - The position of the goal frame w.r.t. vehicle
    frame, expressed in the vehicle frame. This position does not change as
    either the UAV or the goal rotates because the vehicle frame never rotates.
  \item $\rho$ - The inverse depth from the vehicle frame to the goal frame.
    Also note that rotation does not affect this state.
  \item $\vect{v}_{g/I}^{g}$ - The velocity of the goal frame w.r.t. inertial
    frame, expressed in the goal frame. The motion of the goal is most easily
    defined in the goal frame.
  \item $\theta_I^g$ - The angle that represents the rotation from the inertial
    frame to the goal frame. Note that this is only one value as the goal frame
    can only yaw while it is constrained to motion on the plane.
  \item $\omega_{g/I}^g$ - The angular rate of the goal frame w.r.t. inertial
    frame, expressed in the goal frame. This is the rate at which $\theta_I^g$
    changes.
  \item $\vect{r}_{i/g}^g$ - The location of landmark $i$ w.r.t. the goal frame,
    expressed in the goal frame. These landmarks are rigidly attached to the
    same landing vehicle as the goal frame, and therefore $\vect{r}_{i/g}^g$
    does not change, but is a constant.
\end{itemize}

\subsection{UAV State Dynamics}
We use common UAV dynamics with the addition of a linear drag term as explained
in~\cite{leishman2014accel}. The UAV dynamics are given by
\begin{align}
  \dot{\vect{p}}_{b/I}^I
  &=
  R_b^I \vect{v}_{b/I}^b 
  \\
  \begin{bmatrix}
    \dot{\phi} \\
    \dot{\theta} \\
    \dot{\psi}
  \end{bmatrix}
  &=
  \begin{bmatrix}
    1 & \sin\phi\tan\theta & \cos\phi\tan\theta \\
    0 & \cos\phi & -\sin\phi \\
    0 & \frac{\sin\phi}{\cos\theta} & \frac{\cos\phi}{\cos\theta}
  \end{bmatrix}
  \left( \bar{\vect{\omega}}_{b/I}^b - \vect{\beta}_\omega \right)
  \\
  \dot{\vect{v}}_{b/I}^b 
  &=
  R_I^b \vect{g}
  +
  \skewmat{\vect{v}_{b/I}^b}
  \left( \bar{\vect{\omega}}_{b/I}^b - \vect{\beta}_\omega \right)
  -
  \left( \bar{a}_z - \beta_{a_z} \right) \e_3
  -
  \begin{bmatrix}
    \mu u \\
    \mu v \\
    0
  \end{bmatrix}
  \\
  \dot{\mu} &= 0
  \\
  \dot{\vect{\beta}}_a &= \vect{0} + \vect{\eta}_{\beta_a}
  \\
  \dot{\vect{\beta}}_\omega &= \vect{0} + \vect{\eta}_{\beta_\omega}
\end{align}

\subsubsection{UAV State Jacobians}
\begin{equation}
  A =
  \begin{bmatrix}
    \vect{0} & \frac{\partial \dot{\vect{p}}}{\partial \vect{\theta}} &
    \frac{\partial \dot{\vect{p}}}{\partial \vect{v}} & \vect{0} & \vect{0} &
    \vect{0} \\
    \vect{0} & \frac{\partial \dot{\vect{\theta}}}{\partial \vect{\theta}} &
    \vect{0} & \vect{0} & \vect{0} &
    \frac{\partial \dot{\vect{\theta}}}{\partial \vect{\beta}_\omega} \\
    \vect{0} & \frac{\partial \dot{\vect{v}}}{\partial \vect{\theta}} &
    \frac{\partial \dot{\vect{v}}}{\partial \vect{v}} & \frac{\partial
    \dot{\vect{v}}}{\partial \mu} & 
    \frac{\partial \dot{\vect{v}}}{\partial \vect{\beta}_a} &
    \frac{\partial \dot{\vect{v}}}{\partial \vect{\beta}_\omega} \\
    \vect{0} & \vect{0} & \vect{0} & 0 & \vect{0} & \vect{0} \\
    \vect{0} & \vect{0} & \vect{0} & 0 & \vect{0} & \vect{0} \\
    \vect{0} & \vect{0} & \vect{0} & 0 & \vect{0} & \vect{0}
  \end{bmatrix}
\end{equation}
The partial jacobians are then given by:
\begin{align*}
  \dot{\vect{p}}_{b/I}^I &= R_b^I \vect{v}_{b/I}^b \\
  \frac{\partial \dot{\vect{p}}}{\partial \vect{\theta}} &= 
  \frac{\partial R_b^I}{\partial \vect{\theta}} \vect{v}_{b/I}^b \\
  \frac{\partial \dot{\vect{p}}}{\partial \vect{v}} &= R_b^I
\end{align*}

\begin{align*}
  \begin{bmatrix}
    \dot{\phi} \\
    \dot{\theta} \\
    \dot{\psi}
  \end{bmatrix}
  &=
  \begin{bmatrix}
    1 & \sin\phi\tan\theta & \cos\phi\tan\theta \\
    0 & \cos\phi & -\sin\phi \\
    0 & \frac{\sin\phi}{\cos\theta} & \frac{\cos\phi}{\cos\theta}
  \end{bmatrix}
  \left( \bar{\vect{\omega}}_{b/I}^b - \vect{\beta}_\omega \right) \\
  \frac{\partial \dot{\vect{\theta}}}{\partial \vect{\theta}} &= 
  \frac{\partial W_{\text{mat}}}{\partial \vect{\theta}} \left(
  \bar{\vect{\omega}}_{b/I}^b - \vect{\beta}_\omega \right) \\
  \frac{\partial \dot{\vect{\theta}}}{\partial \vect{\beta}_\omega} &= -
  \begin{bmatrix}
    1 & \sin\phi\tan\theta & \cos\phi\tan\theta \\
    0 & \cos\phi & -\sin\phi \\
    0 & \frac{\sin\phi}{\cos\theta} & \frac{\cos\phi}{\cos\theta}
  \end{bmatrix}
\end{align*}

\begin{align*}
  \dot{\vect{v}}_{b/I}^b 
  &=
  R_I^b \vect{g}
  +
  \skewmat{\vect{v}_{b/I}^b}
  \left( \bar{\vect{\omega}}_{b/I}^b - \vect{\beta}_\omega \right)
  -
  \left( \bar{a}_z - \beta_{a_z} \right) \e_3
  -
  \begin{bmatrix}
    \mu u \\
    \mu v \\
    0
  \end{bmatrix} \\
  \frac{\partial \dot{\vect{v}}}{\partial \vect{\theta}} &= 
  \frac{\partial R_I^b}{\partial \vect{\theta}} \vect{g} \\
  \frac{\partial \dot{\vect{v}}}{\partial \vect{v}} &= 
  -\skewmat{\bar{\vect{\omega}}_{b/I}^b - \vect{\beta}_\omega} -
  \begin{bmatrix}
    \mu & 0 & 0 \\
    0 & \mu & 0 \\
    0 & 0 & 0
  \end{bmatrix}\\
  \frac{\partial \dot{\vect{v}}}{\partial \mu} &= -
  \begin{bmatrix}
    u \\ v \\ 0
  \end{bmatrix}\\
  \\
  \frac{\partial \dot{\vect{v}}}{\partial \vect{\beta}_a} &= 
  \begin{bmatrix}
    0 & 0 & 0 \\
    0 & 0 & 0 \\
    0 & 0 & 1
  \end{bmatrix}\\
  \frac{\partial \dot{\vect{v}}}{\partial \vect{\beta}_\omega} &= 
  - \skewmat{\vect{v}_{b/I}^b}
\end{align*}

\subsubsection{Input Jacobians}
\begin{equation}
  B =
  \begin{bmatrix}
    0 & \vect{0} \\
    0 & \frac{\partial \dot{\vect{\theta}}}{\partial \vect{\omega}} \\
    \frac{\partial \dot{\vect{v}}}{\partial a_z} & \frac{\partial
      \dot{\vect{v}}}{\partial \vect{\omega}} \\
    0 & \vect{0} \\
    0 & \vect{0} \\
    0 & \vect{0}
  \end{bmatrix}
\end{equation}

\begin{align*}
  \frac{\partial \dot{\vect{\theta}}}{\partial \vect{\omega}} &=
  \begin{bmatrix}
    1 & \sin\phi\tan\theta & \cos\phi\tan\theta \\
    0 & \cos\phi & -\sin\phi \\
    0 & \frac{\sin\phi}{\cos\theta} & \frac{\cos\phi}{\cos\theta}
  \end{bmatrix} \\
  \frac{\partial \dot{\vect{v}}}{\partial a_z} &= 
  \begin{bmatrix}
    0 & 0 & -1
  \end{bmatrix}^\transpose \\
  \frac{\partial \dot{\vect{v}}}{\partial \vect{\omega}} &=
  \skewmat{\vect{v}_{b/I}^b}
\end{align*}

\subsection{Goal State Dynamics}
We assume that the landing vehicle moves with a constant velocity and a constant
angular velocity. The dynamics of the goal states are expressed as
\begin{align}
  \dot{\vect{p}}_{g/v}^{v} &= \left(R_{v}^{g}\right)^\transpose
  \vect{v}_{g/I}^{g} - I_{2 \times 3}\left(R_{v}^{b}\right)^\transpose \vect{v}_{b/I}^{b} \\
  \dot{\rho} &= \rho^{2} \vect{e}_{3}^\transpose \left(R_{v}^{b}\right)^\transpose \vect{v}_{b/I}^{b} \\
  \dot{\vect{v}}_{g/I}^{g} &= \vect{0} + \eta_{v} \\
  \dot{\theta}_{I}^{g} &= \omega_{g/I}^g \\
  \dot{\omega}_{g/I}^{g} &= 0 + \eta_{\omega} \\
  \dot{\vect r}_{i}^{b} &= \vect{0}.
\end{align}.

Note that the rotation from the vehicle frame to the goal frame is given by
\begin{equation}
  R_v^g =
  \begin{bmatrix}
    \ctheta & \stheta \\
    -\stheta & \ctheta
  \end{bmatrix}.
\end{equation}

\subsection{Motion Model Jacobians}
The full state jacobian is given by
\begin{equation}
  A =
  \begin{bmatrix}
    \frac{\partial \dot{\x}_{\text{UAV}}}{\partial \x_{\text{UAV}}} &
    \frac{\partial \dot{\x}_{\text{UAV}}}{\partial \x_{\text{Goal}}} \\
    \frac{\partial \dot{\x}_{\text{Goal}}}{\partial \x_{\text{UAV}}} &
    \frac{\partial \dot{\x}_{\text{Goal}}}{\partial \x_{\text{Goal}}} 
  \end{bmatrix}.
\end{equation}
We have defined the first term, $\frac{\partial \dot{\x}_{\text{UAV}}}{\partial
\x_{\text{UAV}}}$ above in the previous section. We also note that
\begin{equation}
  \frac{\partial \dot{\x}_{\text{UAV}}}{\partial \x_{\text{Goal}}} = \vect{0}
\end{equation}
as the UAV dynamics do not depend on the landing vehicle dynamics. We define the
remaining terms here below.

\subsubsection{Jacobian w.r.t. UAV State}
\begin{equation}
  \frac{\partial \dot{\x}_{\text{Goal}}}{\partial \x_{\text{UAV}}}
  =
  \begin{bmatrix}
    \vect{0} & \frac{\partial \dot{\vect{p}}_{\text{Goal}}}{\partial
      \vect{\theta}_{\text{UAV}}} & \frac{\partial
      \dot{\vect{p}}_{\text{Goal}}}{\partial \vect{v}_{\text{UAV}}} & 0 \\
    \vect{0} & \frac{\partial \dot{\rho}_{\text{Goal}}}{\partial
      \vect{\theta}_{\text{UAV}}} & \frac{\partial
      \dot{\rho}_{\text{Goal}}}{\partial \vect{v}_{\text{UAV}}} & 0 \\
      \vect{0} & 0 & \vect{0} & 0 \\
    \vect{0} & \vect{0} & \vect{0} & 0 \\
    \vect{0} & \vect{0} & \vect{0} & 0 \\
    \vect{0} & \vect{0} & \vect{0} & 0 \\
  \end{bmatrix}
\end{equation}
\begin{align}
    \frac{\partial \dot{\vect{p}}_{\text{Goal}}}{\partial
      \vect{\theta}_{\text{UAV}}}
      &=
      - I_{2 \times 3} \frac{\partial}{\partial \vect{\theta}_{\text{UAV}}}
      \left(R_{v}^{b}\right)^\transpose \vect{v}_{b/I}^{b}
      \\
    \frac{\partial \dot{\vect{p}}_{\text{Goal}}}{\partial \vect{v}_{\text{UAV}}}
      &=
      - I_{2 \times 3}\left(R_{v}^{b}\right)^\transpose
      \\
    \frac{\partial \dot{\rho}_{\text{Goal}}}{\partial
      \vect{\theta}_{\text{UAV}}}
      &=
      \rho^{2} \vect{e}_{3}^\transpose
        \frac{\partial}{\partial \vect{\theta}_{\text{UAV}}}
        \left(R_{v}^{b}\right)^\transpose \vect{v}_{b/I}^{b}
      \\
    \frac{\partial \dot{\rho}_{\text{Goal}}}{\partial \vect{v}_{\text{UAV}}}
      &=
      \rho^{2} \vect{e}_{3}^\transpose \left(R_{v}^{b}\right)^\transpose
      \\
\end{align}

\subsubsection{Jacobian w.r.t. Goal State}
\begin{equation}
  \frac{\partial \dot{\x}_{\text{Goal}}}{\partial \x_{\text{Goal}}}
  =
  \begin{bmatrix}
    \vect{0} & 0 & \frac{\partial \dot{\vect{p}}_{\text{Goal}}}{\partial
      \vect{v}_{\text{Goal}}} & \frac{\partial
      \dot{\vect{p}}_{\text{Goal}}}{\partial \theta_{\text{Goal}}} & 0 & \vect{0} & 0 \\
    \vect{0} & \frac{\partial \dot{\rho}}{\partial \rho} & \vect{0} & 0 & 0
             & \vect{0} & 0 \\
    \vect{0} & 0 & \vect{0} & 0 & 0 & \vect{0} & 0 \\
    \vect{0} & 0 & \vect{0} & 0 & \frac{\partial
      \dot{\theta}_{\text{Goal}}}{\partial \omega_{Goal}} & \vect{0} & 0 \\
    \vect{0} & 0 & \vect{0} & 0 & 0 & \vect{0} & 0 \\
    \vect{0} & 0 & \vect{0} & 0 & 0 & \vect{0} & 0
  \end{bmatrix}
\end{equation}
\begin{align}
    \frac{\partial \dot{\vect{p}}_{\text{Goal}}}{\partial
      \vect{v}_{\text{Goal}}}
      &=
      \left(R_{v}^{g}\right)^\transpose
      \\
    \frac{\partial \dot{\vect{p}}_{\text{Goal}}}{\partial \theta_{\text{Goal}}}
      &=
      \frac{\partial}{\partial \theta_{\text{Goal}}} \left(R_{v}^{g}\right)^\transpose \vect{v}_{g/I}^{g}
      \\
    \frac{\partial \dot{\rho}}{\partial \rho}
      &=
      2 \rho \vect{e}_{3}^\transpose \left(R_{v}^{b}\right)^\transpose \vect{v}_{b/I}^{b}
      \\
    \frac{\partial \dot{\theta}_{\text{Goal}}}{\partial \omega_{Goal}}
      &=
      1
\end{align}

\subsection{Accelerometer Measurement Update}
As in~\cite{leishman2014accel} we use the accelerometer measurements from the
$x$ and $y$ axes as a measurement update. The expected measurement due to drag
is expressed as
\begin{align}
  \bar{a}_x &= -\mu u + \beta_{a_x} \\
  \bar{a}_y &= -\mu v + \beta_{a_y}.
\end{align}
The measurement jacobian therefore is given by
\begin{align}
  \frac{\partial h_x}{\partial u} &= -\mu \\
  \frac{\partial h_x}{\partial \mu} &= -u \\
  \frac{\partial h_x}{\partial \beta_x} &= 1 \\
  \frac{\partial h_y}{\partial v} &= -\mu \\
  \frac{\partial h_y}{\partial \mu} &= -v \\
  \frac{\partial h_y}{\partial \beta_y} &= 1.
\end{align}

\subsection{Goal ArUco Measurement}
\subsubsection{Measurement model}
In the case that and ArUco marker is placed as the landing pad, and therefore
defines the goal frame, we get a measurement of relative transfrom from the
camera to the ArUco frame. If we assume that there is a known rotation offset
between the ArUco frame and the goal frame, the measurements can be expressed as
\begin{equation}
  h \left( \x \right) = 
  \begin{bmatrix}
    \vect{p}_{a/c}^c \\
    \vect{q}_c^a,
  \end{bmatrix}
\end{equation}
where $F^a$ is the ArUco frame.
We can expand this measurement model using our estimated state
\begin{align}
  \hat{h} \left( \x \right) &=
  \begin{bmatrix}
    \hat{\vect{p}}_{a/c}^c \\
    \hat{\vect{q}}_c^a,
  \end{bmatrix} \\
  \hat{\vect{p}}_{a/c}^c  &= R_b^c \left( \hat{R}_I^b \hat{\vect{p}}_{g/v}^v -
    \vect{p}_{c/v}^b \right) \\
  \hat{\vect{q}}_{c}^a  &= R_g^a \hat{R}_I^g \hat{R}_b^I R_c^b
\end{align}

\subsubsection{Measurement Jacobians}
\begin{align}
  \frac{\partial \hat{\vect{p}}_{a/c}^c}{\partial \phi} =& R_b^c \frac{\partial
  R_I^b}{\partial \phi} \hat{\vect{p}}_{g/v}^v \\
  \frac{\partial \hat{\vect{p}}_{a/c}^c}{\partial \theta} =& R_b^c \frac{\partial
  R_I^b}{\partial \theta} \hat{\vect{p}}_{g/v}^v \\
  \frac{\partial \hat{\vect{p}}_{a/c}^c}{\partial \psi} =& R_b^c \frac{\partial
  R_I^b}{\partial \psi} \hat{\vect{p}}_{g/v}^v \\
    \frac{\partial \hat{\vect{p}}_{a/c}^c}{\partial \vect{p}_{g/v}^v} =& R_b^c \hat{R}_I^b
\end{align}

\subsection{Goal Pixel Measurement}
\subsubsection{Measurement model}
We assume that the UAV is equiped with a downward facing camera that is rigidly
attached to the UAV. From this camera, we can get a measurement of the pixel
location of the goal landing location in the image frame
\begin{equation}
  \vect{z} =
  \begin{bmatrix}
    p_x & p_y
  \end{bmatrix}^\transpose.
\end{equation}
We can expand this measurements using our estimated states
\begin{align}
  \vect{z} &=
  \begin{bmatrix}
    p_x & p_y
  \end{bmatrix}^\transpose \\
  \vect{z} &=
  \begin{bmatrix}
    f_x \frac{\e_1 \vect{p}_{g/c}^c}{\e_3 \vect{p}_{g/c}^c} + c_x \\
    f_y \frac{\e_2 \vect{p}_{g/c}^c}{\e_3 \vect{p}_{g/c}^c} + c_y
  \end{bmatrix},
\end{align}
where $f_x$, $f_y$, $c_x$, and $c_y$ are constant parameters of the camera. We
can then express the position of the goal with respect to the camera in the
camera frame 
\begin{align}
  \vect{p}_{g/c}^c &= R_b^c R_v^b \left(\vect{p}_{g/v}^v -
    \vect{p}_{c/v}^v\right) \\
    \vect{p}_{g/c}^c &= R_b^c \left(R_v^b \vect{p}_{g/v}^v -
    \vect{p}_{c/b}^b\right)
\end{align}
where $R_b^c$ and $p_{c/b}^b$ are constant, known values and
\begin{equation}
  \vect{p}_{g/v}^v =
    \begin{bmatrix}
      \vect{p}_{g/v}^v(0) \\
      \vect{p}_{g/v}^v(1) \\
      1 / \rho
    \end{bmatrix}
\end{equation}
Therefore, the measurement model expanded looks like this
\begin{equation}
  \vect{z} =
  \begin{bmatrix}
    f_x \frac{\e_1 R_b^c \left(R_v^b \vect{p}_{g/v}^v - \vect{p}_{c/b}^b\right)}{\e_3 R_b^c \left(R_v^b \vect{p}_{g/v}^v - \vect{p}_{c/b}^b\right)} + c_x \\
    f_y \frac{\e_2 R_b^c \left(R_v^b \vect{p}_{g/v}^v - \vect{p}_{c/b}^b\right)}{\e_3 R_b^c \left(R_v^b \vect{p}_{g/v}^v - \vect{p}_{c/b}^b\right)} + c_y
  \end{bmatrix}.
\end{equation}

\subsubsection{Jacobian}
To derive  the jacobian of this measurement model, we first look at just the
first measurement, $p_x$ which can be written as
\begin{equation}
    p_x = f_x \left(\e_1 R_b^c \left(R_v^b \vect{p}_{g/v}^v -
      \vect{p}_{c/b}^b\right)\right) \left(\e_3 R_b^c \left(R_v^b \vect{p}_{g/v}^v -
      \vect{p}_{c/b}^b\right)\right)^{-1} + c_x.
\end{equation}
Note that all values are constants except for $\vect{p}_{g/v}^v$ and $R_v^b$.
The individual parts of the jacobian are given here:
\begin{equation}
  \frac{\partial p_x}{\partial \vect{p}_{g/v}^v} =
  \frac{f_x \e_1 R_b^c R_v^b}
    {\left(\e_3 R_b^c \left(R_v^b \vect{p}_{g/v}^v -
    \vect{p}_{c/b}^b\right)\right)}
    - \frac{\left(\e_3 R_b^c R_v^b \right) f_x \left(\e_1 R_b^c \left(R_v^b \vect{p}_{g/v}^v -
        \vect{p}_{c/b}^b\right)\right)} {\left(\e_3 R_b^c \left(R_v^b \vect{p}_{g/v}^v -
  \vect{p}_{c/b}^b\right)\right)^2}
\end{equation}
where only the first two (of three) entries of the vector are used. The third
entry is used for the jacobian w.r.t. $\rho$ as
\begin{equation}
  \frac{\partial p_x}{\partial \rho} = -\frac{1}{\rho^2} \frac{\partial p_x}{\partial
  \vect{p}_{g/v}^v}\left(2\right).
\end{equation}
The derivatives for the euler angles are given by
\begin{equation}
  \frac{\partial p_x}{\partial \phi} =
  \frac{f_x \e_1 R_b^c \frac{\partial R_v^b}{\partial \phi} \vect{p}_{g/v}^v}
    {\left(\e_3 R_b^c \left(R_v^b \vect{p}_{g/v}^v -
    \vect{p}_{c/b}^b\right)\right)}
    - \frac{\left(\e_3 R_b^c \frac{\partial R_v^b}{\partial \phi} \vect{p}_{g/v}^v \right) f_x \left(\e_1 R_b^c \left(R_v^b \vect{p}_{g/v}^v -
        \vect{p}_{c/b}^b\right)\right)} {\left(\e_3 R_b^c \left(R_v^b \vect{p}_{g/v}^v -
  \vect{p}_{c/b}^b\right)\right)^2}
\end{equation}
where the jacobians for the other angles are similar, just substituting in. The
jacobians for the y pixel measurement, $p_y$ are similar to the ones derived
above, just changing $f_y$ in place of $f_x$ and $\e_2$ instead of $\e_1$.


% Python code
%d1 = -np.matmul(E3, np.matmul(RBC, R_I_b)) * FX * (p_g_c_c[0] / p_g_c_c[2] /
        %p_g_c_c[2]) + FX * np.matmul(E1, np.matmul(RBC, R_I_b)) \
                %/ p_g_c_c[2]
%d2 = -np.matmul(E3, np.matmul(RBC, R_I_b)) * FY * (p_g_c_c[1] / p_g_c_c[2] /
        %p_g_c_c[2]) + FY * np.matmul(E2, np.matmul(RBC, R_I_b)) \
                %/ p_g_c_c[2]
%jac[0, 10:12] = d1[0:2]
%jac[1, 10:12] = d2[0:2]

%# d / d rho
%jac[0, 12] = -d1[2] / rho_g / rho_g
%jac[1, 12] = -d2[2] / rho_g / rho_g

%d1dphi = -np.matmul(E3, np.matmul(RBC, np.matmul(dRdPhi, p_g_v_v))) * FX * (p_g_c_c[0] / p_g_c_c[2] /
        %p_g_c_c[2]) + FX * np.matmul(E1, np.matmul(RBC, np.matmul(dRdPhi,
            %p_g_v_v))) / p_g_c_c[2]
%d1dtheta = -np.matmul(E3, np.matmul(RBC, np.matmul(dRdTheta, p_g_v_v))) * FX * (p_g_c_c[0] / p_g_c_c[2] /
        %p_g_c_c[2]) + FX * np.matmul(E1, np.matmul(RBC, np.matmul(dRdTheta,
            %p_g_v_v))) / p_g_c_c[2]
%d1dpsi = -np.matmul(E3, np.matmul(RBC, np.matmul(dRdPsi, p_g_v_v))) * FX * (p_g_c_c[0] / p_g_c_c[2] /
        %p_g_c_c[2]) + FX * np.matmul(E1, np.matmul(RBC, np.matmul(dRdPsi,
            %p_g_v_v))) / p_g_c_c[2]

%jac[0, 3] = d1dphi
%jac[0, 4] = d1dtheta
%jac[0, 5] = d1dpsi

\subsection{Goal Depth Measurement}
We assume that the camera also provides a measurement of the distance to the
goal. This is given in the form of the z coordinate of the goal location in the
camera frame.

\subsubsection{Measurement Model}
We can express the goal depth measurement as
\begin{equation}
  \vect{z} = \e_3^\transpose \vect{p}_{g/c}^c.
\end{equation}
We can expand this to be in terms of our estimated state as
\begin{align}
  \vect{z} &= \e_3^\transpose \vect{p}_{g/c}^c \\
  \vect{z} &= \e_3^\transpose R_b^c R_v^b \vect{p}_{g/c}^v \\
  %\vect{z} &= \e_3^\transpose R_b^c R_v^b \left( \vect{p}_{g/v}^v \right) \\
  \vect{z} &= \e_3 R_b^c \left(R_v^b \vect{p}_{g/v}^v -
    \vect{p}_{c/b}^b\right).
\end{align}

\subsubsection{Measurement Jacobian}
As seen in the measurement model above, the measurement is dependent on the UAV
attitude as well as the goal position. The jacobian terms for these parts are
given by
\begin{align}
  \frac{\partial \vect{z}}{\partial \phi} &= \e_3^\transpose R_b^c \frac{
    \partial R_v^b}{\partial \phi} \left( \vect{p}_{g/v}^v \right) \\
  \frac{\partial \vect{z}}{\partial \theta} &= \e_3^\transpose R_b^c \frac{
    \partial R_v^b}{\partial \theta} \left( \vect{p}_{g/v}^v \right) \\
  \frac{\partial \vect{z}}{\partial \psi} &= \e_3^\transpose R_b^c \frac{
    \partial R_v^b}{\partial \psi} \left( \vect{p}_{g/v}^v \right) \\
    \frac{\partial \vect{z}}{\partial \vect{p}_{g/v}^v} &= \e_3^\transpose R_b^c
    R_v^b \\
  \frac{\partial \vect{z}}{\partial \rho} &= -\frac{1}{\rho^2}
    \frac{\partial \vect{z}}{\partial \vect{p}_{g/v}^v}\left(2\right).
\end{align}

\subsection{Landmark Pixel Measurement}
We assume that landmark locations that are rigidly attached to the landing
vehicle are tracked in the camera image. For now we will just have a fixed
number of landmarks that are tracked for the entire duration of the flight, even
when they leave the FOV of the camera and reenter. In real life, however, the
landmarks must be dynamically added and removed from the estimated states vector
as they leave the FOV of the camera and more are acquired.

\subsubsection{Measurement Model}
The measurement is a pixel location of each landmark in the camera image
\begin{equation}
  \vect{z} =
  \begin{bmatrix}
    p_x & p_y
  \end{bmatrix}^\transpose.
\end{equation}
We can expand this measurements using our estimated states
\begin{align}
  \vect{z} &=
  \begin{bmatrix}
    p_x & p_y
  \end{bmatrix}^\transpose \\
  \vect{z} &=
  \begin{bmatrix}
    f_x \frac{\e_1 \vect{p}_{i/c}^c}{\e_3 \vect{p}_{i/c}^c} + c_x \\
    f_y \frac{\e_2 \vect{p}_{i/c}^c}{\e_3 \vect{p}_{i/c}^c} + c_y
  \end{bmatrix},
\end{align}
where $f_x$, $f_y$, $c_x$, and $c_y$ are constant parameters of the camera. We
can then express the position of the landmark $i$ with respect to the camera in the
camera frame as
\begin{align}
  \vect{p}_{i/c}^c &= R_b^c R_v^b \left(\vect{p}_{i/v}^v -
    \vect{p}_{c/v}^v\right) \\
    \vect{p}_{i/c}^c &= R_b^c \left(R_v^b \vect{p}_{i/v}^v -
    \vect{p}_{c/b}^b\right)
\end{align}
This is actually a little bit weird because of the state parameters, but
\begin{equation}
  \vect{p}_{i/v}^v =
  \begin{bmatrix}
    \e_1^\transpose \left( R_g^v \vect{p}_{i/g}^g + \vect{p}_{g/v}^v \right) \\
    \e_2^\transpose \left( R_g^v \vect{p}_{i/g}^g + \vect{p}_{g/v}^v \right) \\
    \left( \e_3^\transpose \vect{p}_{i/g}^g + 1 / \rho \right)
  \end{bmatrix}.
\end{equation}

\subsubsection{Measurement Jacobian}
For the purpose of deriving the jacobians, we will just look at the jacobians
with respect to $p_x$, the pixel location along the $x$ axis of the camera
frame. From above we have that
\begin{align}
  p_x &= f_x \frac{\e_1^\transpose \vect{p}_{i/c}^c}{\e_3^\transpose \vect{p}_{i/c}^c} + c_x \\ 
  p_x &= f_x \frac{\e_1^\transpose R_b^c \left(R_v^b \vect{p}_{i/v}^v -
      \vect{p}_{c/b}^b\right) }{\e_3^\transpose R_b^c \left(R_v^b \vect{p}_{i/v}^v -
  \vect{p}_{c/b}^b\right) } + c_x 
\end{align}
Note that all values are constants except for $\vect{p}_{i/v}^v$ and $R_v^b$.
The individual parts of the jacobian are given here:
\begin{equation}
  \frac{\partial p_x}{\partial \vect{p}_{g/v}^v} =
  \frac{f_x \e_1 R_b^c R_v^b}
    {\left(\e_3 R_b^c \left(R_v^b \vect{p}_{i/v}^v -
    \vect{p}_{c/b}^b\right)\right)}
    - \frac{\left(\e_3 R_b^c R_v^b \right) f_x \left(\e_1 R_b^c \left(R_v^b \vect{p}_{i/v}^v -
        \vect{p}_{c/b}^b\right)\right)} {\left(\e_3 R_b^c \left(R_v^b \vect{p}_{i/v}^v -
  \vect{p}_{c/b}^b\right)\right)^2}
\end{equation}
where only the first two (of three) entries of the vector are used. The third
entry is used for the jacobian w.r.t. $\rho$ as
\begin{equation}
  \frac{\partial p_x}{\partial \rho} = -\frac{1}{\rho^2} \frac{\partial p_x}{\partial
  \vect{p}_{g/v}^v}\left(2\right).
\end{equation}
The derivatives for the euler angles are given by
\begin{equation}
  \frac{\partial p_x}{\partial \phi} =
  \frac{f_x \e_1 R_b^c \frac{\partial R_v^b}{\partial \phi} \vect{p}_{i/v}^v}
    {\left(\e_3 R_b^c \left(R_v^b \vect{p}_{i/v}^v -
    \vect{p}_{c/b}^b\right)\right)}
    - \frac{\left(\e_3 R_b^c \frac{\partial R_v^b}{\partial \phi} \vect{p}_{i/v}^v \right) f_x \left(\e_1 R_b^c \left(R_v^b \vect{p}_{i/v}^v -
        \vect{p}_{c/b}^b\right)\right)} {\left(\e_3 R_b^c \left(R_v^b \vect{p}_{i/v}^v -
  \vect{p}_{c/b}^b\right)\right)^2}
\end{equation}
where the jacobians for the other angles are similar, just substituting in.
The derivative for the goal angle, $\theta_g$ is given by using
\begin{equation}
  \frac{\partial \vect{p}_{i/v}^v}{\partial \theta_g} =
  \begin{bmatrix}
    \e_1^\transpose \frac{\partial R_g^v}{\partial \theta_g} \vect{p}_{i/g}^g \\
    \e_2^\transpose \frac{\partial R_g^v}{\partial \theta_g} \vect{p}_{i/g}^g \\
    0
  \end{bmatrix}
\end{equation}
in the jacobian given by
\begin{equation}
  \frac{\partial p_x}{\partial \theta_g} =
  \frac{f_x \e_1 R_b^c R_v^b \frac{\partial \vect{p}_{i/v}^v}{\partial \theta_g}}
    {\left(\e_3 R_b^c \left(R_v^b \vect{p}_{i/v}^v -
    \vect{p}_{c/b}^b\right)\right)}
    - \frac{\left(\e_3 R_b^c R_v^b \frac{\partial \vect{p}_{i/v}^v}{\partial \theta_g} \right) f_x \left(\e_1 R_b^c \left(R_v^b \vect{p}_{i/v}^v -
        \vect{p}_{c/b}^b\right)\right)} {\left(\e_3 R_b^c \left(R_v^b \vect{p}_{i/v}^v -
  \vect{p}_{c/b}^b\right)\right)^2}
\end{equation}
Similarly, the derivative for the landmark offset, $\vect{r}_i$ or
$\vect{p}_{i/g}^g$ is given by using
\begin{equation}
  \frac{\partial \vect{p}_{i/v}^v}{\partial \vect{p}_{i/g}^g} =
  \begin{bmatrix}
    \e_1^\transpose R_g^v \\
    \e_2^\transpose R_g^v \\
    \e_3^\transpose
  \end{bmatrix}
\end{equation}
in the jacobian given by
\begin{equation}
  \frac{\partial p_x}{\partial \vect{p}_{i/g}^g} =
  \frac{f_x \e_1 R_b^c R_v^b \frac{\partial \vect{p}_{i/v}^v}{\partial \vect{p}_{i/g}^g}}
    {\left(\e_3 R_b^c \left(R_v^b \vect{p}_{i/v}^v -
    \vect{p}_{c/b}^b\right)\right)}
    - \frac{\left(\e_3 R_b^c R_v^b \frac{\partial \vect{p}_{i/v}^v}{\partial \vect{p}_{i/g}^g} \right) f_x \left(\e_1 R_b^c \left(R_v^b \vect{p}_{i/v}^v -
        \vect{p}_{c/b}^b\right)\right)} {\left(\e_3 R_b^c \left(R_v^b \vect{p}_{i/v}^v -
  \vect{p}_{c/b}^b\right)\right)^2}
\end{equation}
The jacobians for the y pixel measurement, $p_y$ are similar to the ones derived
above, just changing $f_y$ in place of $f_x$ and $\e_2$ instead of $\e_1$.

\subsection{Landmark Pixel Measurement 2.0}

\subsubsection{Measurement Model}
The measurement is a pixel location of each landmark in the camera image
\begin{equation}
  \vect{z} =
  \begin{bmatrix}
    p_x & p_y
  \end{bmatrix}^\transpose.
\end{equation}
We can expand this measurements using our estimated states
\begin{align}
  \vect{z} &=
  \begin{bmatrix}
    p_x & p_y
  \end{bmatrix}^\transpose \\
  \vect{z} &=
  \begin{bmatrix}
    f_x \frac{\e_1 \vect{p}_{i/c}^c}{\e_3 \vect{p}_{i/c}^c} + c_x \\
    f_y \frac{\e_2 \vect{p}_{i/c}^c}{\e_3 \vect{p}_{i/c}^c} + c_y
  \end{bmatrix},
\end{align}
where $f_x$, $f_y$, $c_x$, and $c_y$ are constant parameters of the camera. We
can then express the position of the landmark $i$ with respect to the camera in the
camera frame as
\begin{align}
  \vect{p}_{i/c}^c &= R_b^c R_v^b \left(\vect{p}_{i/v}^v -
    \vect{p}_{c/v}^v\right) \\
    \vect{p}_{i/c}^c &= R_b^c \left(R_v^b \vect{p}_{i/v}^v -
    \vect{p}_{c/b}^b\right)
\end{align}
This is actually a little bit weird because of the state parameters, but
\begin{equation}
  \vect{p}_{i/v}^v =
  \begin{bmatrix}
    \e_1^\transpose \left( R_g^v \vect{p}_{i/g}^g + \vect{p}_{g/v}^v \right) \\
    \e_2^\transpose \left( R_g^v \vect{p}_{i/g}^g + \vect{p}_{g/v}^v \right) \\
    \e_3^\transpose \left( \vect{p}_{i/g}^g - \vect{p}_{b/I}^I \right)
  \end{bmatrix}.
\end{equation}

\subsubsection{Measurement Jacobian}
For the purpose of deriving the jacobians, we will just look at the jacobians
with respect to $p_x$, the pixel location along the $x$ axis of the camera
frame. From above we have that
\begin{align}
  p_x &= f_x \frac{\e_1^\transpose \vect{p}_{i/c}^c}{\e_3^\transpose \vect{p}_{i/c}^c} + c_x \\ 
  p_x &= f_x \frac{\e_1^\transpose R_b^c \left(R_v^b \vect{p}_{i/v}^v -
      \vect{p}_{c/b}^b\right) }{\e_3^\transpose R_b^c \left(R_v^b \vect{p}_{i/v}^v -
  \vect{p}_{c/b}^b\right) } + c_x 
\end{align}
Note that all values are constants except for $\vect{p}_{i/v}^v$ and $R_v^b$.
The individual parts of the jacobian are given here:
\begin{equation}
  \frac{\partial p_x}{\partial \vect{p}_{g/v}^v} =
  \frac{f_x \e_1 R_b^c R_v^b}
    {\left(\e_3 R_b^c \left(R_v^b \vect{p}_{i/v}^v -
    \vect{p}_{c/b}^b\right)\right)}
    - \frac{\left(\e_3 R_b^c R_v^b \right) f_x \left(\e_1 R_b^c \left(R_v^b \vect{p}_{i/v}^v -
        \vect{p}_{c/b}^b\right)\right)} {\left(\e_3 R_b^c \left(R_v^b \vect{p}_{i/v}^v -
  \vect{p}_{c/b}^b\right)\right)^2}
\end{equation}
where only the first two (of three) entries of the vector are used. The third
entry is used for the jacobian w.r.t. $\vect{p}_{b/I}^I(2)$ as
\begin{equation}
  \frac{\partial p_x}{\partial \vect{p}_{b/I}^I(2)} = -\frac{\partial p_x}{\partial
  \vect{p}_{g/v}^v}\left(2\right).
\end{equation}
The derivatives for the attitude representation are less straight forward. From
"Micro Lie Theory" we use the fact that the jacobian of the rotation action can
be shown to be
\begin{equation}
  J_R^{R \cdot v} = -R \skewmat{v}.
\end{equation}
\begin{equation}
  \frac{\partial p_x}{\partial \q} =
  \frac{-f_x \e_1 R_b^c R_v^b \skewmat{\vect{p}_{i/v}^v}}
    {\left(\e_3 R_b^c \left(R_v^b \vect{p}_{i/v}^v -
    \vect{p}_{c/b}^b\right)\right)}
    - \frac{\left(-\e_3 R_b^c R_v^b \skewmat{\vect{p}_{i/v}^v} \right) f_x \left(\e_1 R_b^c \left(R_v^b \vect{p}_{i/v}^v -
        \vect{p}_{c/b}^b\right)\right)} {\left(\e_3 R_b^c \left(R_v^b \vect{p}_{i/v}^v -
  \vect{p}_{c/b}^b\right)\right)^2}
\end{equation}
where the jacobians for the other angles are similar, just substituting in.
The derivative for the goal angle, $\theta_g$ is given by using
\begin{equation}
  \frac{\partial \vect{p}_{i/v}^v}{\partial \theta_g} =
  \begin{bmatrix}
    \e_1^\transpose \frac{\partial R_g^v}{\partial \theta_g} \vect{p}_{i/g}^g \\
    \e_2^\transpose \frac{\partial R_g^v}{\partial \theta_g} \vect{p}_{i/g}^g \\
    0
  \end{bmatrix}
\end{equation}
in the jacobian given by
\begin{equation}
  \frac{\partial p_x}{\partial \theta_g} =
  \frac{f_x \e_1 R_b^c R_v^b \frac{\partial \vect{p}_{i/v}^v}{\partial \theta_g}}
    {\left(\e_3 R_b^c \left(R_v^b \vect{p}_{i/v}^v -
    \vect{p}_{c/b}^b\right)\right)}
    - \frac{\left(\e_3 R_b^c R_v^b \frac{\partial \vect{p}_{i/v}^v}{\partial \theta_g} \right) f_x \left(\e_1 R_b^c \left(R_v^b \vect{p}_{i/v}^v -
        \vect{p}_{c/b}^b\right)\right)} {\left(\e_3 R_b^c \left(R_v^b \vect{p}_{i/v}^v -
  \vect{p}_{c/b}^b\right)\right)^2}
\end{equation}
Similarly, the derivative for the landmark offset, $\vect{r}_i$ or
$\vect{p}_{i/g}^g$ is given by using
\begin{equation}
  \frac{\partial \vect{p}_{i/v}^v}{\partial \vect{p}_{i/g}^g} =
  \begin{bmatrix}
    \e_1^\transpose R_g^v \\
    \e_2^\transpose R_g^v \\
    \e_3^\transpose
  \end{bmatrix}
\end{equation}
in the jacobian given by
\begin{equation}
  \frac{\partial p_x}{\partial \vect{p}_{i/g}^g} =
  \frac{f_x \e_1 R_b^c R_v^b \frac{\partial \vect{p}_{i/v}^v}{\partial \vect{p}_{i/g}^g}}
    {\left(\e_3 R_b^c \left(R_v^b \vect{p}_{i/v}^v -
    \vect{p}_{c/b}^b\right)\right)}
    - \frac{\left(\e_3 R_b^c R_v^b \frac{\partial \vect{p}_{i/v}^v}{\partial \vect{p}_{i/g}^g} \right) f_x \left(\e_1 R_b^c \left(R_v^b \vect{p}_{i/v}^v -
        \vect{p}_{c/b}^b\right)\right)} {\left(\e_3 R_b^c \left(R_v^b \vect{p}_{i/v}^v -
  \vect{p}_{c/b}^b\right)\right)^2}
\end{equation}
The jacobians for the y pixel measurement, $p_y$ are similar to the ones derived
above, just changing $f_y$ in place of $f_x$ and $\e_2$ instead of $\e_1$.

\subsection{Landmark Initialization}
Each time a new landmark is acquired and added to the estimated vector, we
should initialize its location to something intelligent. Theoretically, we could
just initialize the landmark's location to zero with a large enough associated
covariance. However, by initializing the landmark to the location based on its
initial pixel measurement, we can initialize the landmark with a smaller
covariance.

When we first acquire a landmark, the only information we have about it is a
measurement of its pixel location in the image frame. This measurement, denoted
by
\begin{equation}
  h \left( \x \right) = 
  \begin{bmatrix}
    p_x & p_y
  \end{bmatrix}^\transpose
\end{equation}
can be used to deduce the landmark's relative position vector that is estimated.
The easiest way to see this is by first creating a virtual image plane. This
virtual image plane projects the landmark pixel location into the image plane as
if the camera were perfectly aligned with the inertial, vehicle frame. We can
derive this using the pin hole camera model where
\begin{align}
  \begin{bmatrix}
    p_x \\ p_y \\ 1
  \end{bmatrix} &= K
  \begin{bmatrix}
    X^c / Z^c \\
    Y^c / Z^c \\
    1
  \end{bmatrix} \\
  \begin{bmatrix}
    X^c / Z^c \\
    Y^c / Z^c \\
    1
  \end{bmatrix}
   &= K^{-1}
  \begin{bmatrix}
    p_x \\ p_y \\ 1
  \end{bmatrix} \\
  \begin{bmatrix}
    X^c / Z^c \\
    Y^c / Z^c \\
    1
  \end{bmatrix}^v
   &= R_b^v R_c^b K^{-1}
  \begin{bmatrix}
    p_x \\ p_y \\ 1
  \end{bmatrix} \\
\end{align}
The resulting vector, $
  \begin{bmatrix}
    X^c / Z^c \\
    Y^c / Z^c \\
    1
  \end{bmatrix}^v$
  is the vector $\vect{p}_{i/c}^v$ up to a scale factor. The vector can be scaled by assuming that the altitude of the landmark is equal to the
altitude of the goal. To get the expected altitude, we solve for
\begin{align}
  \e_3^\transpose \vect{p}_{g/c}^v &= \e_3^\transpose \vect{p}_{g/b}^v - \e_3^\transpose \vect{p}_{c/b}^v \\
  \e_3^\transpose \vect{p}_{g/c}^v &= \e_3^\transpose \vect{p}_{g/b}^v -
  \e_3^\transpose R_b^I \vect{p}_{c/b}^b \\
  \e_3^\transpose \vect{p}_{g/c}^v &= \frac{1}{\rho_g} -
  \e_3^\transpose R_b^I \vect{p}_{c/b}^b \\
\end{align}

This gives us the vector, $\vect{p}_{i/c}^v$.
With this vector, we can then reach the estimated state vector,
$\vect{p}_{i/g}^g$ with the following
\begin{align}
  \vect{p}_{i/v}^v &= \vect{p}_{i/c}^v + R_b^I \vect{p}_{c/b}^b \\
  \vect{p}_{i/g}^g &= R_v^g \left( \vect{p}_{i/v}^v - \vect{p}_{g/v}^v \right).
\end{align}


% !TEX root=./root.tex

\subsection{Propagation Model}
To model the motion of the UAV, we use common rigid-body kinematics given by
% We use common rigid body kinematics to model the dynamics of
% the UAV given by
\begin{align}
  \dot{\vect{p}}_{b/I}^I
  &=
  \left( R_I^b \right)^\transpose \vect{v}_{b/I}^b
  \label{eq:uav_dynamics}
  \\
  \dot{\vect{q}}_{I}^{b} 
	&= 	
  \q_I^b \otimes \begin{pmatrix} 0 \\ \frac{1}{2}
    \left( \bar{\vect{\omega}}_{b/I}^b - \vect{\beta}_\omega - \vect{\upsilon}_\omega \right)
\end{pmatrix} \nonumber \\
  \dot{\vect{v}}_{b/I}^b 
  &=
  R_I^b \vect{g}^I
  +
  \skewmat{\vect{v}_{b/I}^b}
  \left( \bar{\vect{\omega}}_{b/I}^b - \vect{\beta}_\omega -
  \vect{\upsilon}_\omega \right) \nonumber \\
  & \qquad +
  \left( \bar{\vect{a}}_{b/I}^b - \vect{\beta}_a - \vect{\upsilon}_a \right) \nonumber
  % \vect{\eta}_v 
  \\
  \dot{\vect{\beta}}_a &= \vect{\eta}_{\beta_a} \nonumber
  \\
  \dot{\vect{\beta}}_\omega &= \vect{\eta}_{\beta_\omega}, \nonumber
\end{align}
% $\vect{\eta}_v$, 
where $\vect{g}^I$ represents the gravity vector expressed in the inertial
frame, $\vect{\eta}_{\beta_a}$ and $\vect{\eta}_{\beta_\omega}$ are zero-mean
Gaussian noise processes corresponding to the state dynamics, and
$\vect{\upsilon}_\omega$ and $\vect{\upsilon}_a$ are zero-mean Gaussian noise
processes corresponding to the noise in the inputs to the system.

We model the motion of the landing target vehicle
with a constant-velocity and constant-angular-velocity motion model such that
% Though this is a very simplified model, we show
% in~\secref{sec:simulation} and~\secref{sec:hardware} that it is sufficient for
% cases that do not perfectly match this model. The dynamics of $\x_{\text{Goal}}$ are expressed as
% \begin{align}
  % \dot{\hat{\vect{p}}}_{g/b}^{v} &= \left( \hat{R}_{I}^{g} \right)^\transpose
  % \hat{\vect{v}}_{g/I}^{g} - \left( \hat{R}_{I}^{b} \right)^\transpose
  % \hat{\vect{v}}_{b/I}^{b} \label{eq:goal_dynamics} \\
  % \dot{\hat{\vect{v}}}_{g/I}^{g} &= \vect{0} \nonumber \\
  % \dot{\hat{\theta}}_{I}^{g} &= \hat{\omega}_{g/I}^g \nonumber \\
  % \dot{\hat{\omega}}_{g/I}^{g} &= 0. \nonumber
% \end{align}
\begin{align}
  \dot{\vect{p}}_{g/b}^{v} &= I_{3 \times 2} \left( R_{I}^{g} \right)^\transpose
   \vect{v}_{g/I}^{g} - \left( R_{I}^{b} \right)^\transpose
  \vect{v}_{b/I}^{b} \label{eq:goal_dynamics} \\
  \dot{\vect{v}}_{g/I}^{g} &= \vect{\eta}_{gv} \nonumber \\
  \dot{\psi}_{I}^{g} &= \omega_{g/I}^g \nonumber \\
  % \dot{R}_{I}^{g} &= -\skewmat{ \omega_{g/I}^g } R_I^g \nonumber \\
  \dot{\omega}_{g/I}^{g} &= \eta_{g\omega}, \nonumber
\end{align}
where $\vect{\eta}_{gv}$ and $\eta_{g\omega}$ are zero-mean Gaussian noise
processes. Though this is a simplified motion model for the target
vehicle, we show in~\secref{sec:est_paper_simulation}
and~\secref{sec:est_paper_hardware} that it is
satisfactory for our experiments. We intend the motion model of the target
vehicle to be easily modified for vehicles with more complex motion such
as a boat at sea.

As mentioned previously, we assume the tracked visual features are rigidly
attached to the landing vehicle such that
% there are no dynamics 
% of their associated states
\begin{equation}
  \dot{\vect{r}}_{i/g}^g = \vect{0}. \label{eq:feature_dynamics}
  % \dot{\hat{\vect{r}}}_{i/g}^g = \vect{0}. \label{eq:feature_dynamics}
\end{equation}

In the ESKF, the estimated state is propagated independently of the filter using
the expected value of the modeled dynamics.
% We use the expected values
% of~\eqref{eq:uav_dynamics}, \eqref{eq:goal_dynamics}, and
% \eqref{eq:feature_dynamics} given by
% % The estimated state is propagated
% \begin{align}
  % \dot{\hat{\vect{p}}}_{b/I}^I
  % &=
  % \left( \hat{R}_I^b \right)^\transpose \hat{\vect{v}}_{b/I}^b
  % \label{eq:estimated_dynamics}
  % \\
  % \dot{\hat{\vect{q}}}_{I}^{b} 
  % &= 	
  % \hat{\q}_I^b \otimes \begin{pmatrix} 0 \\ \frac{1}{2}
  % \left( \bar{\vect{\omega}}_{b/I}^b - \hat{\vect{\beta}}_\omega \right)
% \end{pmatrix} \nonumber \\
  % \dot{\hat{\vect{v}}}_{b/I}^b 
  % &=
  % \hat{R}_I^b \vect{g}^I
  % +
  % \skewmat{\hat{\vect{v}}_{b/I}^b}
  % \left( \bar{\vect{\omega}}_{b/I}^b - \hat{\vect{\beta}}_\omega \right)
  % \nonumber \\
  % & \qquad +
  % \left( \bar{\vect{a}}_{b/I}^b - \hat{\vect{\beta}}_a \right) \nonumber
  % \\
  % \dot{\hat{\vect{\beta}}}_a &= \vect{0} \nonumber
  % \\
  % \dot{\hat{\vect{\beta}}}_\omega &= \vect{0} \nonumber
  % \\
  % \dot{\hat{\vect{p}}}_{g/b}^{v} &= I_{3 \times 2} \left( \hat{R}_{I}^{g} \right)^\transpose
   % \hat{\vect{v}}_{g/I}^{g} - \left( \hat{R}_{I}^{b} \right)^\transpose
  % \hat{\vect{v}}_{b/I}^{b} \nonumber \\
  % \dot{\hat{\vect{v}}}_{g/I}^{g} &= \vect{0} \nonumber \\
  % \dot{\hat{\psi}}_{I}^{g} &= \hat{\omega}_{g/I}^g \nonumber \\
  % % \dot{\hat{R}}_{I}^{g} &= -\skewmat{ \hat{\omega}_{g/I}^g } \hat{R}_I^g \nonumber \\
  % \dot{\hat{\omega}}_{g/I}^{g} &= 0 \nonumber \\
  % \dot{\hat{\vect{r}}}_{i/g}^g &= \vect{0}. \nonumber
% \end{align}

The error-state dynamics used to propagate the filter are found by relating
the modeled true-state dynamics from~\eqref{eq:uav_dynamics},
\eqref{eq:goal_dynamics}, and \eqref{eq:feature_dynamics}
with their expected values. Using the error-state definitions
% with~\eqref{eq:estimated_dynamics} using the error-state definitions
from~\eqref{eq:est_paper_vector_error_state} and~\eqref{eq:quat_error_state}, the
first-order approximation of the error-state dynamics is given by 
\begin{align}
  \dot{\tilde{\vect{p}}}_{b/I}^I
  &\approx
  \left( \hat{R}_I^b \right)^\transpose \tilde{\vect{v}}_{b/I}^b
  - \left( \hat{R}_I^b \right)^\transpose \skewmat{\hat{\vect{v}}_{b/I}^b}
  \tilde{\vect{\theta}}_I^b
  \\
  \dot{\tilde{\vect{\theta}}}_{I}^{b} 
  &\approx 	
  -\skewmat{ \bar{\vect{\omega}}_{b/I}^b - \hat{\vect{\beta}}_\omega}
    \tilde{\vect{\theta}}_I^b
    - \tilde{\vect{\beta}}_\omega -
    \vect{\upsilon}_\omega
  \nonumber \\
  \dot{\tilde{\vect{v}}}_{b/I}^b 
  &\approx
  \skewmat{ \hat{R}_I^b \vect{g}^I } \tilde{\vect{\theta}}_I^b 
  -
  \skewmat{ \hat{\vect{v}}_{b/I}^b } \tilde{\vect{\beta}}_\omega
  -
  \skewmat{ \hat{\vect{v}}_{b/I}^b } \vect{\upsilon}_\omega
  \nonumber \\
  & \qquad -
  \skewmat{ \bar{\omega}_{b/I}^b - \hat{\vect{\beta}}_\omega }
  \tilde{\vect{v}}_{b/I}^b
  -
  \tilde{\vect{\beta}}_a
  -
  \vect{\upsilon}_a \nonumber
  \\
  \dot{\tilde{\vect{\beta}}}_a &= \vect{\eta}_{\beta_a} \nonumber
  \\
  \dot{\tilde{\vect{\beta}}}_\omega &= \vect{\eta}_{\beta_\omega} \nonumber
  \\
  \dot{\tilde{\vect{p}}}_{g/b}^{v}
                                  &\approx
  I_{3 \times 2} \left( \hat{R}_{I}^{g} \right)^\transpose
  \tilde{\vect{v}}_{g/I}^g
  +
  I_{3 \times 2} \left( \hat{R}_{I}^{g} \right)^\transpose
  \skewmat{ \tilde{\psi}_I^g } \hat{\vect{v}}_{g/I}^g
  \nonumber \\
  & \qquad +
  \left( \hat{R}_{I}^{b} \right)^\transpose \skewmat{ \hat{\vect{v}}_{b/I}^{b} } 
  \tilde{\vect{\theta}}_I^b
  -
  \left( \hat{R}_{I}^{b} \right)^\transpose \tilde{\vect{v}}_{b/I}^{b} \nonumber \\
  \dot{\tilde{\vect{v}}}_{g/I}^{g} &= \vect{\eta}_{gv} \nonumber \\
  \dot{\tilde{\psi}}_{I}^{g} &= \tilde{\omega}_{g/I}^g \nonumber \\
  \dot{\tilde{\omega}}_{g/I}^{g} &= \eta_{g\omega} \nonumber \\
  \dot{\tilde{\vect{r}}}_{i/g}^g &= \vect{0}, \nonumber
\end{align}
or succinctly,
\begin{equation}
  \dot{\tilde{\x}} = f\left(\x, \tilde{\x}, \vect{u}, \tilde{\vect{u}}\right).
\end{equation}
% The derivation of these error-state dynamics can be found in
% Appendix~\ref{apdx:estimation_err_state_derivation}.
In practice, the expected
value of the error state remains zero over the propagation window, and only the
error covariance, $P$, is propagated.
The continous-time derivative of the error covariance is given by
% forward using a first-order approximation
% of the continous-time derivative of covariance given by
\begin{equation}
  \dot{P} = FP + PF^\transpose + G Q_{\vect{u}} G^\transpose + Q_{\vect{x}}
  % P^{+} = FP + PF^\transpose + G Q_{\vect{u}} G^\transpose + Q_{\vect{x}}
\end{equation}
where $Q_{\vect{u}}$ is the input noise covariance, $Q_{\vect{x}}$ is the
process noise covariance, and
\begin{align}
  F &= \frac{ \partial \dot{\tilde{\x}} }{ \partial \tilde{\x} } \\
  G &= \frac{ \partial \dot{\tilde{\x}} }{ \partial \vect{\upsilon} } .
\end{align}
% \MF{TODO define this matrix by vertical slices}
% \begin{align}
  % F &= \frac{ \partial \dot{\tilde{\x}} }{ \partial \tilde{\x} } \\
    % &=
    % \scriptsize
    % \begin{bmatrix}
    % % \begin{smallbmatrix}
      % \vect{0} & - \left( \hat{R}_I^b \right)^\transpose
      % \skewmat{\hat{\vect{v}}_{b/I}^b} & \left( \hat{R}_I^b \right)^\transpose &
      % \vect{0} & \vect{0} & \vect{0} & \vect{0}
               % & 0 & 0 & \vect{0} \\
      % \vect{0} & -\skewmat{ \bar{\vect{\omega}}_{b/I}^b
      % - \hat{\vect{\beta}}_\omega} & \vect{0} & \vect{0} & -I & \vect{0} & \vect{0}
               % & 0 & 0 & \vect{0} \\
      % \vect{0} & \skewmat{ \hat{R}_I^b \vect{g} } &
      % -\skewmat{ \bar{\omega}_{b/I}^b - \hat{\vect{\beta}}_\omega } & -I &
      % -\skewmat{ \hat{\vect{v}}_{b/I}^b } & \vect{0} & \vect{0}
               % & 0 & 0 & \vect{0} \\
      % \vect{0} & \vect{0} & \vect{0} & \vect{0} & \vect{0} & \vect{0} & \vect{0}
               % & 0 & 0 & \vect{0} \\
      % \vect{0} & \vect{0} & \vect{0} & \vect{0} & \vect{0} & \vect{0} & \vect{0}
               % & 0 & 0 & \vect{0} \\
      % \vect{0} & \left( \hat{R}_{I}^{b} \right)^\transpose
      % \skewmat{ \hat{\vect{v}}_{b/I}^{b} } & 
      % -\left( \hat{R}_{I}^{b} \right)^\transpose & \vect{0} & \vect{0} & \vect{0} & 
      % I_{3 \times 2} \left( \hat{R}_{I}^{g} \right)^\transpose
               % & I_{3 \times 2} \left( \hat{R}_{I}^{g} \right)^\transpose
               % \skewmat{ 1 } \hat{\vect{v}}_{g/I}^g
               % & 0 & \vect{0} \\
      % \vect{0} & \vect{0} & \vect{0} & \vect{0} & \vect{0} & \vect{0} & \vect{0}
               % & 0 & 0 & \vect{0} \\
      % \vect{0} & \vect{0} & \vect{0} & \vect{0} & \vect{0} & \vect{0} & \vect{0}
               % & 0 & 1 & \vect{0} \\
      % \vect{0} & \vect{0} & \vect{0} & \vect{0} & \vect{0} & \vect{0} & \vect{0}
               % & 0 & 0 & \vect{0} \\
      % \vect{0} & \vect{0} & \vect{0} & \vect{0} & \vect{0} & \vect{0} & \vect{0}
               % & 0 & 0 & \vect{0}
    % \end{bmatrix},
    % % \end{smallbmatrix},
% \end{align}
% \begin{align}
  % F &= \frac{ \partial \dot{\tilde{\x}} }{ \partial \tilde{\x} } \\
    % &=
    % \begin{bmatrix}
      % \vect{0} &
    % \cfrac{ \partial \dot{\tilde{\vect{p}}}_{b/I}^{I}}{ \partial \tilde{\vect{\theta}}_I^{b}}
      % % - \left( \hat{R}_I^b \right)^\transpose
      % % \skewmat{\hat{\vect{v}}_{b/I}^b}
               % &
      % % \left( \hat{R}_I^b \right)^\transpose
               % \cfrac{ \partial \dot{\tilde{\vect{p}}}_{b/I}^{I} }{ \partial \tilde{\vect{v}}_{b/I}^b }
               % &
      % \vect{0} & \vect{0} & \vect{0} & \vect{0}
               % & \vect{0} & \vect{0} & \vect{0} & \dots & \vect{0} \\
      % \vect{0} &
      % % -\skewmat{ \bar{\vect{\omega}}_{b/I}^b
      % % - \hat{\vect{\beta}}_\omega}
      % \cfrac{ \partial \dot{\tilde{\vect{\theta}}}_I^{b} }{ \partial \tilde{\vect{\theta}}_I^{b} }
               % & \vect{0} & \vect{0} &
      % \cfrac{ \partial \dot{\tilde{\vect{\theta}}}_I^{b} }{ \partial \tilde{\vect{\beta}}_{\omega} }
      % % -I
               % & \vect{0} & \vect{0}
               % & \vect{0} & \vect{0} & \vect{0} & \dots & \vect{0} \\
      % \vect{0} &
      % % \skewmat{ \hat{R}_I^b \vect{g} }
      % \cfrac{ \partial \dot{\tilde{\vect{v}}}_{b/I}^b }{ \partial \tilde{\vect{\theta}}_I^{b} }
               % &
      % % -\skewmat{ \bar{\omega}_{b/I}^b - \hat{\vect{\beta}}_\omega }
      % \cfrac{ \partial \dot{\tilde{\vect{v}}}_{b/I}^b }{ \partial \tilde{\vect{v}}_{b/I}^b }
               % &
      % \cfrac{ \partial \dot{\tilde{\vect{v}}}_{b/I}^b }{ \partial \tilde{\vect{\beta}}_a }
      % % -I
               % &
      % \cfrac{ \partial \dot{\tilde{\vect{v}}}_{b/I}^b }{ \partial \tilde{\vect{\beta}}_{\omega} }
      % % -\skewmat{ \hat{\vect{v}}_{b/I}^b }
               % & \vect{0} & \vect{0}
               % & \vect{0} & \vect{0} & \vect{0} & \dots & \vect{0} \\
      % \vect{0} & \vect{0} & \vect{0} & \vect{0} & \vect{0} & \vect{0} & \vect{0}
               % & \vect{0} & \vect{0} & \vect{0} & \dots & \vect{0} \\
      % \vect{0} & \vect{0} & \vect{0} & \vect{0} & \vect{0} & \vect{0} & \vect{0}
               % & \vect{0} & \vect{0} & \vect{0} & \dots & \vect{0} \\
      % \vect{0} &
      % % \left( \hat{R}_{I}^{b} \right)^\transpose
      % % \skewmat{ \hat{\vect{v}}_{b/I}^{b} }
      % \cfrac{ \partial \dot{\tilde{\vect{p}}}_{g/b}^{v} }{ \partial \tilde{\vect{\theta}}_I^{b} }
               % & 
      % % -\left( \hat{R}_{I}^{b} \right)^\transpose
      % \cfrac{ \partial \dot{\tilde{\vect{p}}}_{g/b}^{v} }{ \partial \tilde{\vect{v}}_{b/I}^b }
               % & \vect{0} & \vect{0} & \vect{0} & 
      % % I_{3 \times 2} \left( \hat{R}_{I}^{g} \right)^\transpose
      % \cfrac{ \partial \dot{\tilde{\vect{p}}}_{g/b}^{v} }{ \partial \tilde{\vect{v}}_{g/I}^{g} }
               % &
               % % I_{3 \times 2} \left( \hat{R}_{I}^{g} \right)^\transpose
               % % \skewmat{ 1 } \hat{\vect{v}}_{g/I}^g
      % \cfrac{ \partial \dot{\tilde{\vect{p}}}_{g/b}^{v} }{ \partial \tilde{\psi}_{I}^{g} }
               % & \vect{0} & \vect{0} & \dots & \vect{0} \\
      % \vect{0} & \vect{0} & \vect{0} & \vect{0} & \vect{0} & \vect{0} & \vect{0}
               % & \vect{0} & \vect{0} & \vect{0} & \dots & \vect{0} \\
      % \vect{0} & \vect{0} & \vect{0} & \vect{0} & \vect{0} & \vect{0} & \vect{0}
               % & 0 & 
      % % 1 
               % \cfrac{ \partial \dot{\tilde{\psi}}_{I}^{g} }{ \partial \tilde{\omega}_{g/I}^{g} }
               % & \vect{0} & \dots & \vect{0} \\
      % \vect{0} & \vect{0} & \vect{0} & \vect{0} & \vect{0} & \vect{0} & \vect{0}
               % & 0 & 0 & \vect{0} & \dots & \vect{0} \\
      % \vect{0} & \vect{0} & \vect{0} & \vect{0} & \vect{0} & \vect{0} & \vect{0}
               % & \vect{0} & \vect{0} & \vect{0} & \dots & \vect{0} \\
      % \vdots & \vdots & \vdots & \vdots & \vdots & \vdots & \vdots
             % & \vdots & \vdots & \vdots & \ddots & \vdots \\
      % \vect{0} & \vect{0} & \vect{0} & \vect{0} & \vect{0} & \vect{0} & \vect{0}
               % & \vect{0} & \vect{0} & \vect{0} & \dots & \vect{0}
    % \end{bmatrix},
% \end{align}
% with
% \begin{align}
    % \cfrac{ \partial \dot{\tilde{\vect{p}}}_{b/I}^{I}}{ \partial \tilde{\vect{\theta}}_I^{b}}
    % &=
      % - \left( \hat{R}_I^b \right)^\transpose
      % \skewmat{\hat{\vect{v}}_{b/I}^b} \\
               % \cfrac{ \partial \dot{\tilde{\vect{p}}}_{b/I}^{I} }{ \partial \tilde{\vect{v}}_{b/I}^b }
    % &=
      % \left( \hat{R}_I^b \right)^\transpose \\
      % \cfrac{ \partial \dot{\tilde{\vect{\theta}}}_I^{b} }{ \partial \tilde{\vect{\theta}}_I^{b} }
    % &=
      % -\skewmat{ \bar{\vect{\omega}}_{b/I}^b
      % - \hat{\vect{\beta}}_\omega}
% \end{align}
% \begin{align}
      % \cfrac{ \partial \dot{\tilde{\vect{\theta}}}_I^{b} }{ \partial \tilde{\vect{\beta}}_{\omega} }
    % &=
      % -I \\
      % \cfrac{ \partial \dot{\tilde{\vect{v}}}_{b/I}^b }{ \partial \tilde{\vect{\theta}}_I^{b} }
    % &=
      % \skewmat{ \hat{R}_I^b \vect{g} } \\
      % \cfrac{ \partial \dot{\tilde{\vect{v}}}_{b/I}^b }{ \partial \tilde{\vect{v}}_{b/I}^b }
    % &=
      % -\skewmat{ \bar{\omega}_{b/I}^b - \hat{\vect{\beta}}_\omega } \\
      % \cfrac{ \partial \dot{\tilde{\vect{v}}}_{b/I}^b }{ \partial \tilde{\vect{\beta}}_a }
    % &=
      % -I \\
      % \cfrac{ \partial \dot{\tilde{\vect{v}}}_{b/I}^b }{ \partial \tilde{\vect{\beta}}_{\omega} }
    % &=
      % -\skewmat{ \hat{\vect{v}}_{b/I}^b } \\
      % \cfrac{ \partial \dot{\tilde{\vect{p}}}_{g/b}^{v} }{ \partial \tilde{\vect{\theta}}_I^{b} }
    % &=
      % \left( \hat{R}_{I}^{b} \right)^\transpose
      % \skewmat{ \hat{\vect{v}}_{b/I}^{b} } \\
      % \cfrac{ \partial \dot{\tilde{\vect{p}}}_{g/b}^{v} }{ \partial \tilde{\vect{v}}_{b/I}^b }
    % &=
      % -\left( \hat{R}_{I}^{b} \right)^\transpose \\
      % \cfrac{ \partial \dot{\tilde{\vect{p}}}_{g/b}^{v} }{ \partial \tilde{\vect{v}}_{g/I}^{g} }
    % &=
      % I_{3 \times 2} \left( \hat{R}_{I}^{g} \right)^\transpose \\
      % \cfrac{ \partial \dot{\tilde{\vect{p}}}_{g/b}^{v} }{ \partial \tilde{\psi}_{I}^{g} }
    % &=
               % I_{3 \times 2} \left( \hat{R}_{I}^{g} \right)^\transpose
               % \skewmat{ 1 } \hat{\vect{v}}_{g/I}^g \\
               % \cfrac{ \partial \dot{\tilde{\psi}}_{I}^{g} }{ \partial \tilde{\omega}_{g/I}^{g} }
    % &=
      % 1 ,
% \end{align}
% and
% \begin{align}
  % G &= \frac{ \partial \dot{\tilde{\x}} }{ \partial \vect{\upsilon} } \\
    % &=
    % \begin{bmatrix}
      % \vect{0} & \vect{0} \\
      % \vect{0} & -I \\
      % -I & -\skewmat{\hat{\vect{v}}_{b/I}^b} \\
      % \vect{0} & \vect{0} \\
      % \vect{0} & \vect{0} \\
      % \vect{0} & \vect{0} \\
      % \vect{0} & \vect{0} \\
      % \vect{0} & \vect{0} \\
      % \vect{0} & \vect{0} \\
      % \vect{0} & \vect{0} \\
      % \vdots & \vdots \\
      % \vect{0} & \vect{0} 
    % \end{bmatrix}.
% \end{align}
However, to ensure numerical stability, we propagate the covariance using a first-order discrete approximation
defined by
\begin{equation}
  P_{k+1} = F_{k} P_{k} F_{k}^\transpose + G_k Q_u G_k^\transpose + Q_x \Delta
  t^2
\end{equation}
where
\begin{align}
  F_k &\approx I + F \Delta t \\
  G_k &\approx G \Delta t.
\end{align}



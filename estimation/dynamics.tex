% !TEX root=./root.tex

\subsection{Propagation Model}
We use common rigid body kinematics to model the dynamics of
the UAV given by
\begin{align}
  \dot{\vect{p}}_{b/I}^I
  &=
  \left( R_I^b \right)^\transpose \vect{v}_{b/I}^b
  \label{eq:uav_dynamics}
  \\
  \dot{\vect{q}}_{I}^{b} 
	&= 	
  \q_I^b \otimes \begin{pmatrix} 0 \\ \frac{1}{2}
    \left( \bar{\vect{\omega}}_{b/I}^b - \vect{\beta}_\omega - \vect{\upsilon}_\omega \right)
\end{pmatrix} \nonumber \\
  \dot{\vect{v}}_{b/I}^b 
  &=
  R_I^b \vect{g}
  +
  \skewmat{\vect{v}_{b/I}^b}
  \left( \bar{\vect{\omega}}_{b/I}^b - \vect{\beta}_\omega -
  \vect{\upsilon}_\omega \right)
  +
  \left( \bar{\vect{a}} - \vect{\beta}_a - \vect{\upsilon}_a \right) \nonumber
  % \vect{\eta}_v 
  \\
  \dot{\vect{\beta}}_a &= \vect{\eta}_{\beta_a} \nonumber
  \\
  \dot{\vect{\beta}}_\omega &= \vect{\eta}_{\beta_\omega} \nonumber
\end{align}
% $\vect{\eta}_v$, 
where $\vect{\eta}_{\beta_a}$, $\vect{\eta}_{\beta_\omega}$,
$\vect{\upsilon}_\omega$, and $\vect{\upsilon}_a$
are zero-mean Gaussian noise processes.

We model the motion of the landing vehicle
with a constant velocity and constant
angular velocity motion model such that
% Though this is a very simplified model, we show
% in~\secref{sec:simulation} and~\secref{sec:hardware} that it is sufficient for
% cases that do not perfectly match this model. The dynamics of $\x_{\text{Goal}}$ are expressed as
% \begin{align}
  % \dot{\hat{\vect{p}}}_{g/b}^{v} &= \left( \hat{R}_{I}^{g} \right)^\transpose
  % \hat{\vect{v}}_{g/I}^{g} - \left( \hat{R}_{I}^{b} \right)^\transpose
  % \hat{\vect{v}}_{b/I}^{b} \label{eq:goal_dynamics} \\
  % \dot{\hat{\vect{v}}}_{g/I}^{g} &= \vect{0} \nonumber \\
  % \dot{\hat{\theta}}_{I}^{g} &= \hat{\omega}_{g/I}^g \nonumber \\
  % \dot{\hat{\omega}}_{g/I}^{g} &= 0. \nonumber
% \end{align}
\begin{align}
  \dot{\vect{p}}_{g/b}^{v} &= I_{3 \times 2} \left( R_{I}^{g} \right)^\transpose
   \vect{v}_{g/I}^{g} - \left( R_{I}^{b} \right)^\transpose
  \vect{v}_{b/I}^{b} \label{eq:goal_dynamics} \\
  \dot{\vect{v}}_{g/I}^{g} &= \vect{\eta}_{gv} \nonumber \\
  \dot{\psi}_{I}^{g} &= \omega_{g/I}^g \nonumber \\
  \dot{\omega}_{g/I}^{g} &= \eta_{g\omega} \nonumber,
\end{align}
where $\vect{\eta}_{gv}$ and $\eta_{g\omega}$ are zero-mean Gaussian noise
processes. Though this is a very simplified motion model for the landing
vehicle, we show in~\secref{sec:simulation} and~\secref{sec:hardware} that is is
satisfactory for our experiments. We intend the motion model of the landing
vehicle to be easily modified for landing vehicles with more complex motion such
as a boat at sea.

As mentioned previously, we assume the tracked visual features are rigidly
attached to the landing vehicle such that
% there are no dynamics 
% of their associated states
\begin{equation}
  \dot{\vect{r}}_{i/g}^g = \vect{0}. \label{eq:feature_dynamics}
  % \dot{\hat{\vect{r}}}_{i/g}^g = \vect{0}. \label{eq:feature_dynamics}
\end{equation}

In the ESKF, the estimated state is propagated independently of the filter using
the expected value of the modeled dynamics. We use the expected values
of~\eqref{eq:uav_dynamics}, \eqref{eq:goal_dynamics}, and
\eqref{eq:feature_dynamics} given by
% The estimated state is propagated
\begin{align}
  \dot{\hat{\vect{p}}}_{b/I}^I
  &=
  \left( \hat{R}_I^b \right)^\transpose \hat{\vect{v}}_{b/I}^b
  \label{eq:estimated_dynamics}
  \\
  \dot{\hat{\vect{q}}}_{I}^{b} 
  &= 	
  \hat{\q}_I^b \otimes \begin{pmatrix} 0 \\ \frac{1}{2}
  \left( \bar{\vect{\omega}}_{b/I}^b - \hat{\vect{\beta}}_\omega \right)
\end{pmatrix} \nonumber \\
  \dot{\hat{\vect{v}}}_{b/I}^b 
  &=
  \hat{R}_I^b \vect{g}
  +
  \skewmat{\hat{\vect{v}}_{b/I}^b}
  \left( \bar{\vect{\omega}}_{b/I}^b - \hat{\vect{\beta}}_\omega \right)
  +
  \left( \bar{\vect{a}} - \hat{\vect{\beta}}_a \right) \nonumber
  \\
  \dot{\hat{\vect{\beta}}}_a &= \vect{0} \nonumber
  \\
  \dot{\hat{\vect{\beta}}}_\omega &= \vect{0} \nonumber
  \\
  \dot{\hat{\vect{p}}}_{g/b}^{v} &= I_{3 \times 2} \left( \hat{R}_{I}^{g} \right)^\transpose
   \hat{\vect{v}}_{g/I}^{g} - \left( \hat{R}_{I}^{b} \right)^\transpose
  \hat{\vect{v}}_{b/I}^{b} \nonumber \\
  \dot{\hat{\vect{v}}}_{g/I}^{g} &= \vect{0} \nonumber \\
  \dot{\hat{\psi}}_{I}^{g} &= \hat{\omega}_{g/I}^g \nonumber \\
  \dot{\hat{\omega}}_{g/I}^{g} &= 0 \nonumber \\
  \dot{\hat{\vect{r}}}_{i/g}^g &= \vect{0}. \nonumber
\end{align}

The error-state dynamics used to propagate the filter are found by relating
the modeled true-state dynamics from~\eqref{eq:uav_dynamics},
\eqref{eq:goal_dynamics}, and \eqref{eq:feature_dynamics}
with~\eqref{eq:estimated_dynamics} using the error-state definitions
from~\eqref{eq:vector_error_state} and~\eqref{eq:quat_error_state}. The
first-order approximation of the error-state dynamics are given by \MF{TODO put
the right err state dynamics here}
\begin{align}
  \dot{\hat{\vect{p}}}_{b/I}^I
  &=
  \left( \hat{R}_I^b \right)^\transpose \hat{\vect{v}}_{b/I}^b
  \\
  \dot{\hat{\vect{q}}}_{I}^{b} 
  &= 	
  \hat{\q}_I^b \otimes \begin{pmatrix} 0 \\ \frac{1}{2}
  \left( \bar{\vect{\omega}}_{b/I}^b - \hat{\vect{\beta}}_\omega \right)
\end{pmatrix} \nonumber \\
  \dot{\hat{\vect{v}}}_{b/I}^b 
  &=
  \hat{R}_I^b \vect{g}
  +
  \skewmat{\hat{\vect{v}}_{b/I}^b}
  \left( \bar{\vect{\omega}}_{b/I}^b - \hat{\vect{\beta}}_\omega \right)
  +
  \left( \bar{\vect{a}} - \hat{\vect{\beta}}_a \right) \nonumber
  \\
  \dot{\hat{\vect{\beta}}}_a &= \vect{0} \nonumber
  \\
  \dot{\hat{\vect{\beta}}}_\omega &= \vect{0} \nonumber
  \\
  \dot{\hat{\vect{p}}}_{g/b}^{v} &= \left( \hat{R}_{I}^{g} \right)^\transpose
  \hat{\vect{v}}_{g/I}^{g} - \left( \hat{R}_{I}^{b} \right)^\transpose
  \hat{\vect{v}}_{b/I}^{b} \nonumber \\
  \dot{\hat{\vect{v}}}_{g/I}^{g} &= \vect{0} \nonumber \\
  \dot{\hat{\psi}}_{I}^{g} &= \hat{\omega}_{g/I}^g \nonumber \\
  \dot{\hat{\omega}}_{g/I}^{g} &= 0 \nonumber \\
  \dot{\hat{\vect{r}}}_{i/g}^g &= \vect{0}. \nonumber
\end{align}
The derivation of these error-state dynamics can be found in the Appendix
\MF{TODO reference correct section}.

% \subsubsection{UAV State Jacobians}
% \begin{equation}
  % A =
  % \begin{bmatrix}
    % \vect{0} & \frac{\partial \dot{\vect{p}}}{\partial \vect{\theta}} &
    % \frac{\partial \dot{\vect{p}}}{\partial \vect{v}} & \vect{0} & \vect{0} &
    % \vect{0} \\
    % \vect{0} & \frac{\partial \dot{\vect{\theta}}}{\partial \vect{\theta}} &
    % \vect{0} & \vect{0} & \vect{0} &
    % \frac{\partial \dot{\vect{\theta}}}{\partial \vect{\beta}_\omega} \\
    % \vect{0} & \frac{\partial \dot{\vect{v}}}{\partial \vect{\theta}} &
    % \frac{\partial \dot{\vect{v}}}{\partial \vect{v}} & \frac{\partial
    % \dot{\vect{v}}}{\partial \mu} & 
    % \frac{\partial \dot{\vect{v}}}{\partial \vect{\beta}_a} &
    % \frac{\partial \dot{\vect{v}}}{\partial \vect{\beta}_\omega} \\
    % \vect{0} & \vect{0} & \vect{0} & 0 & \vect{0} & \vect{0} \\
    % \vect{0} & \vect{0} & \vect{0} & 0 & \vect{0} & \vect{0} \\
    % \vect{0} & \vect{0} & \vect{0} & 0 & \vect{0} & \vect{0}
  % \end{bmatrix}
% \end{equation}
% The partial jacobians are then given by:
% \begin{align*}
  % \dot{\vect{p}}_{b/I}^I &= R_b^I \vect{v}_{b/I}^b \\
  % \frac{\partial \dot{\vect{p}}}{\partial \vect{\theta}} &= 
  % \frac{\partial R_b^I}{\partial \vect{\theta}} \vect{v}_{b/I}^b \\
  % \frac{\partial \dot{\vect{p}}}{\partial \vect{v}} &= R_b^I
% \end{align*}

% \begin{align*}
  % \begin{bmatrix}
    % \dot{\phi} \\
    % \dot{\theta} \\
    % \dot{\psi}
  % \end{bmatrix}
  % &=
  % \begin{bmatrix}
    % 1 & \sin\phi\tan\theta & \cos\phi\tan\theta \\
    % 0 & \cos\phi & -\sin\phi \\
    % 0 & \frac{\sin\phi}{\cos\theta} & \frac{\cos\phi}{\cos\theta}
  % \end{bmatrix}
  % \left( \bar{\vect{\omega}}_{b/I}^b - \vect{\beta}_\omega \right) \\
  % \frac{\partial \dot{\vect{\theta}}}{\partial \vect{\theta}} &= 
  % \frac{\partial W_{\text{mat}}}{\partial \vect{\theta}} \left(
  % \bar{\vect{\omega}}_{b/I}^b - \vect{\beta}_\omega \right) \\
  % \frac{\partial \dot{\vect{\theta}}}{\partial \vect{\beta}_\omega} &= -
  % \begin{bmatrix}
    % 1 & \sin\phi\tan\theta & \cos\phi\tan\theta \\
    % 0 & \cos\phi & -\sin\phi \\
    % 0 & \frac{\sin\phi}{\cos\theta} & \frac{\cos\phi}{\cos\theta}
  % \end{bmatrix}
% \end{align*}

% \begin{align*}
  % \dot{\vect{v}}_{b/I}^b 
  % &=
  % R_I^b \vect{g}
  % +
  % \skewmat{\vect{v}_{b/I}^b}
  % \left( \bar{\vect{\omega}}_{b/I}^b - \vect{\beta}_\omega \right)
  % -
  % \left( \bar{a}_z - \beta_{a_z} \right) \e_3
  % -
  % \begin{bmatrix}
    % \mu u \\
    % \mu v \\
    % 0
  % \end{bmatrix} \\
  % \frac{\partial \dot{\vect{v}}}{\partial \vect{\theta}} &= 
  % \frac{\partial R_I^b}{\partial \vect{\theta}} \vect{g} \\
  % \frac{\partial \dot{\vect{v}}}{\partial \vect{v}} &= 
  % -\skewmat{\bar{\vect{\omega}}_{b/I}^b - \vect{\beta}_\omega} -
  % \begin{bmatrix}
    % \mu & 0 & 0 \\
    % 0 & \mu & 0 \\
    % 0 & 0 & 0
  % \end{bmatrix}\\
  % \frac{\partial \dot{\vect{v}}}{\partial \mu} &= -
  % \begin{bmatrix}
    % u \\ v \\ 0
  % \end{bmatrix}\\
  % \\
  % \frac{\partial \dot{\vect{v}}}{\partial \vect{\beta}_a} &= 
  % \begin{bmatrix}
    % 0 & 0 & 0 \\
    % 0 & 0 & 0 \\
    % 0 & 0 & 1
  % \end{bmatrix}\\
  % \frac{\partial \dot{\vect{v}}}{\partial \vect{\beta}_\omega} &= 
  % - \skewmat{\vect{v}_{b/I}^b}
% \end{align*}

% \subsubsection{Input Jacobians}
% \begin{equation}
  % B =
  % \begin{bmatrix}
    % 0 & \vect{0} \\
    % 0 & \frac{\partial \dot{\vect{\theta}}}{\partial \vect{\omega}} \\
    % \frac{\partial \dot{\vect{v}}}{\partial a_z} & \frac{\partial
      % \dot{\vect{v}}}{\partial \vect{\omega}} \\
    % 0 & \vect{0} \\
    % 0 & \vect{0} \\
    % 0 & \vect{0}
  % \end{bmatrix}
% \end{equation}

% \begin{align*}
  % \frac{\partial \dot{\vect{\theta}}}{\partial \vect{\omega}} &=
  % \begin{bmatrix}
    % 1 & \sin\phi\tan\theta & \cos\phi\tan\theta \\
    % 0 & \cos\phi & -\sin\phi \\
    % 0 & \frac{\sin\phi}{\cos\theta} & \frac{\cos\phi}{\cos\theta}
  % \end{bmatrix} \\
  % \frac{\partial \dot{\vect{v}}}{\partial a_z} &= 
  % \begin{bmatrix}
    % 0 & 0 & -1
  % \end{bmatrix}^\transpose \\
  % \frac{\partial \dot{\vect{v}}}{\partial \vect{\omega}} &=
  % \skewmat{\vect{v}_{b/I}^b}
% \end{align*}


% \subsection{Motion Model Jacobians}
% The full state jacobian is given by
% \begin{equation}
  % A =
  % \begin{bmatrix}
    % \frac{\partial \dot{\x}_{\text{UAV}}}{\partial \x_{\text{UAV}}} &
    % \frac{\partial \dot{\x}_{\text{UAV}}}{\partial \x_{\text{Goal}}} \\
    % \frac{\partial \dot{\x}_{\text{Goal}}}{\partial \x_{\text{UAV}}} &
    % \frac{\partial \dot{\x}_{\text{Goal}}}{\partial \x_{\text{Goal}}} 
  % \end{bmatrix}.
% \end{equation}
% We have defined the first term, $\frac{\partial \dot{\x}_{\text{UAV}}}{\partial
% \x_{\text{UAV}}}$ above in the previous section. We also note that
% \begin{equation}
  % \frac{\partial \dot{\x}_{\text{UAV}}}{\partial \x_{\text{Goal}}} = \vect{0}
% \end{equation}
% as the UAV dynamics do not depend on the landing vehicle dynamics. We define the
% remaining terms here below.

% \subsubsection{Jacobian w.r.t. UAV State}
% \begin{equation}
  % \frac{\partial \dot{\x}_{\text{Goal}}}{\partial \x_{\text{UAV}}}
  % =
  % \begin{bmatrix}
    % \vect{0} & \frac{\partial \dot{\vect{p}}_{\text{Goal}}}{\partial
      % \vect{\theta}_{\text{UAV}}} & \frac{\partial
      % \dot{\vect{p}}_{\text{Goal}}}{\partial \vect{v}_{\text{UAV}}} & 0 \\
    % \vect{0} & \frac{\partial \dot{\rho}_{\text{Goal}}}{\partial
      % \vect{\theta}_{\text{UAV}}} & \frac{\partial
      % \dot{\rho}_{\text{Goal}}}{\partial \vect{v}_{\text{UAV}}} & 0 \\
      % \vect{0} & 0 & \vect{0} & 0 \\
    % \vect{0} & \vect{0} & \vect{0} & 0 \\
    % \vect{0} & \vect{0} & \vect{0} & 0 \\
    % \vect{0} & \vect{0} & \vect{0} & 0 \\
  % \end{bmatrix}
% \end{equation}
% \begin{align}
    % \frac{\partial \dot{\vect{p}}_{\text{Goal}}}{\partial
      % \vect{\theta}_{\text{UAV}}}
      % &=
      % - I_{2 \times 3} \frac{\partial}{\partial \vect{\theta}_{\text{UAV}}}
      % \left(R_{v}^{b}\right)^\transpose \vect{v}_{b/I}^{b}
      % \\
    % \frac{\partial \dot{\vect{p}}_{\text{Goal}}}{\partial \vect{v}_{\text{UAV}}}
      % &=
      % - I_{2 \times 3}\left(R_{v}^{b}\right)^\transpose
      % \\
    % \frac{\partial \dot{\rho}_{\text{Goal}}}{\partial
      % \vect{\theta}_{\text{UAV}}}
      % &=
      % \rho^{2} \vect{e}_{3}^\transpose
        % \frac{\partial}{\partial \vect{\theta}_{\text{UAV}}}
        % \left(R_{v}^{b}\right)^\transpose \vect{v}_{b/I}^{b}
      % \\
    % \frac{\partial \dot{\rho}_{\text{Goal}}}{\partial \vect{v}_{\text{UAV}}}
      % &=
      % \rho^{2} \vect{e}_{3}^\transpose \left(R_{v}^{b}\right)^\transpose
      % \\
% \end{align}

% \subsubsection{Jacobian w.r.t. Goal State}
% \begin{equation}
  % \frac{\partial \dot{\x}_{\text{Goal}}}{\partial \x_{\text{Goal}}}
  % =
  % \begin{bmatrix}
    % \vect{0} & 0 & \frac{\partial \dot{\vect{p}}_{\text{Goal}}}{\partial
      % \vect{v}_{\text{Goal}}} & \frac{\partial
      % \dot{\vect{p}}_{\text{Goal}}}{\partial \theta_{\text{Goal}}} & 0 & \vect{0} & 0 \\
    % \vect{0} & \frac{\partial \dot{\rho}}{\partial \rho} & \vect{0} & 0 & 0
             % & \vect{0} & 0 \\
    % \vect{0} & 0 & \vect{0} & 0 & 0 & \vect{0} & 0 \\
    % \vect{0} & 0 & \vect{0} & 0 & \frac{\partial
      % \dot{\theta}_{\text{Goal}}}{\partial \omega_{Goal}} & \vect{0} & 0 \\
    % \vect{0} & 0 & \vect{0} & 0 & 0 & \vect{0} & 0 \\
    % \vect{0} & 0 & \vect{0} & 0 & 0 & \vect{0} & 0
  % \end{bmatrix}
% \end{equation}
% \begin{align}
    % \frac{\partial \dot{\vect{p}}_{\text{Goal}}}{\partial
      % \vect{v}_{\text{Goal}}}
      % &=
      % \left(R_{v}^{g}\right)^\transpose
      % \\
    % \frac{\partial \dot{\vect{p}}_{\text{Goal}}}{\partial \theta_{\text{Goal}}}
      % &=
      % \frac{\partial}{\partial \theta_{\text{Goal}}} \left(R_{v}^{g}\right)^\transpose \vect{v}_{g/I}^{g}
      % \\
    % \frac{\partial \dot{\rho}}{\partial \rho}
      % &=
      % 2 \rho \vect{e}_{3}^\transpose \left(R_{v}^{b}\right)^\transpose \vect{v}_{b/I}^{b}
      % \\
    % \frac{\partial \dot{\theta}_{\text{Goal}}}{\partial \omega_{Goal}}
      % &=
      % 1
% \end{align}

% \subsection{Accelerometer Measurement Update}
% As in~\cite{leishman2014accel} we use the accelerometer measurements from the
% $x$ and $y$ axes as a measurement update. The expected measurement due to drag
% is expressed as
% \begin{align}
  % \bar{a}_x &= -\mu u + \beta_{a_x} \\
  % \bar{a}_y &= -\mu v + \beta_{a_y}.
% \end{align}
% The measurement jacobian therefore is given by
% \begin{align}
  % \frac{\partial h_x}{\partial u} &= -\mu \\
  % \frac{\partial h_x}{\partial \mu} &= -u \\
  % \frac{\partial h_x}{\partial \beta_x} &= 1 \\
  % \frac{\partial h_y}{\partial v} &= -\mu \\
  % \frac{\partial h_y}{\partial \mu} &= -v \\
  % \frac{\partial h_y}{\partial \beta_y} &= 1.
% \end{align}

% \subsection{Goal ArUco Measurement}
% \subsubsection{Measurement model}
% In the case that and ArUco marker is placed as the landing pad, and therefore
% defines the goal frame, we get a measurement of relative transfrom from the
% camera to the ArUco frame. If we assume that there is a known rotation offset
% between the ArUco frame and the goal frame, the measurements can be expressed as
% \begin{equation}
  % h \left( \x \right) = 
  % \begin{bmatrix}
    % \vect{p}_{a/c}^c \\
    % \vect{q}_c^a,
  % \end{bmatrix}
% \end{equation}
% where $F^a$ is the ArUco frame.
% We can expand this measurement model using our estimated state
% \begin{align}
  % \hat{h} \left( \x \right) &=
  % \begin{bmatrix}
    % \hat{\vect{p}}_{a/c}^c \\
    % \hat{\vect{q}}_c^a,
  % \end{bmatrix} \\
  % \hat{\vect{p}}_{a/c}^c  &= R_b^c \left( \hat{R}_I^b \hat{\vect{p}}_{g/v}^v -
    % \vect{p}_{c/v}^b \right) \\
  % \hat{\vect{q}}_{c}^a  &= R_g^a \hat{R}_I^g \hat{R}_b^I R_c^b
% \end{align}

% \subsubsection{Measurement Jacobians}
% \begin{align}
  % \frac{\partial \hat{\vect{p}}_{a/c}^c}{\partial \phi} =& R_b^c \frac{\partial
  % R_I^b}{\partial \phi} \hat{\vect{p}}_{g/v}^v \\
  % \frac{\partial \hat{\vect{p}}_{a/c}^c}{\partial \theta} =& R_b^c \frac{\partial
  % R_I^b}{\partial \theta} \hat{\vect{p}}_{g/v}^v \\
  % \frac{\partial \hat{\vect{p}}_{a/c}^c}{\partial \psi} =& R_b^c \frac{\partial
  % R_I^b}{\partial \psi} \hat{\vect{p}}_{g/v}^v \\
    % \frac{\partial \hat{\vect{p}}_{a/c}^c}{\partial \vect{p}_{g/v}^v} =& R_b^c \hat{R}_I^b
% \end{align}

% \subsection{Goal Pixel Measurement}
% \subsubsection{Measurement model}
% We assume that the UAV is equiped with a downward facing camera that is rigidly
% attached to the UAV. From this camera, we can get a measurement of the pixel
% location of the goal landing location in the image frame
% \begin{equation}
  % \vect{z} =
  % \begin{bmatrix}
    % p_x & p_y
  % \end{bmatrix}^\transpose.
% \end{equation}
% We can expand this measurements using our estimated states
% \begin{align}
  % \vect{z} &=
  % \begin{bmatrix}
    % p_x & p_y
  % \end{bmatrix}^\transpose \\
  % \vect{z} &=
  % \begin{bmatrix}
    % f_x \frac{\e_1 \vect{p}_{g/c}^c}{\e_3 \vect{p}_{g/c}^c} + c_x \\
    % f_y \frac{\e_2 \vect{p}_{g/c}^c}{\e_3 \vect{p}_{g/c}^c} + c_y
  % \end{bmatrix},
% \end{align}
% where $f_x$, $f_y$, $c_x$, and $c_y$ are constant parameters of the camera. We
% can then express the position of the goal with respect to the camera in the
% camera frame 
% \begin{align}
  % \vect{p}_{g/c}^c &= R_b^c R_v^b \left(\vect{p}_{g/v}^v -
    % \vect{p}_{c/v}^v\right) \\
    % \vect{p}_{g/c}^c &= R_b^c \left(R_v^b \vect{p}_{g/v}^v -
    % \vect{p}_{c/b}^b\right)
% \end{align}
% where $R_b^c$ and $p_{c/b}^b$ are constant, known values and
% \begin{equation}
  % \vect{p}_{g/v}^v =
    % \begin{bmatrix}
      % \vect{p}_{g/v}^v(0) \\
      % \vect{p}_{g/v}^v(1) \\
      % 1 / \rho
    % \end{bmatrix}
% \end{equation}
% Therefore, the measurement model expanded looks like this
% \begin{equation}
  % \vect{z} =
  % \begin{bmatrix}
    % f_x \frac{\e_1 R_b^c \left(R_v^b \vect{p}_{g/v}^v - \vect{p}_{c/b}^b\right)}{\e_3 R_b^c \left(R_v^b \vect{p}_{g/v}^v - \vect{p}_{c/b}^b\right)} + c_x \\
    % f_y \frac{\e_2 R_b^c \left(R_v^b \vect{p}_{g/v}^v - \vect{p}_{c/b}^b\right)}{\e_3 R_b^c \left(R_v^b \vect{p}_{g/v}^v - \vect{p}_{c/b}^b\right)} + c_y
  % \end{bmatrix}.
% \end{equation}

% \subsubsection{Jacobian}
% To derive  the jacobian of this measurement model, we first look at just the
% first measurement, $p_x$ which can be written as
% \begin{equation}
    % p_x = f_x \left(\e_1 R_b^c \left(R_v^b \vect{p}_{g/v}^v -
      % \vect{p}_{c/b}^b\right)\right) \left(\e_3 R_b^c \left(R_v^b \vect{p}_{g/v}^v -
      % \vect{p}_{c/b}^b\right)\right)^{-1} + c_x.
% \end{equation}
% Note that all values are constants except for $\vect{p}_{g/v}^v$ and $R_v^b$.
% The individual parts of the jacobian are given here:
% \begin{equation}
  % \frac{\partial p_x}{\partial \vect{p}_{g/v}^v} =
  % \frac{f_x \e_1 R_b^c R_v^b}
    % {\left(\e_3 R_b^c \left(R_v^b \vect{p}_{g/v}^v -
    % \vect{p}_{c/b}^b\right)\right)}
    % - \frac{\left(\e_3 R_b^c R_v^b \right) f_x \left(\e_1 R_b^c \left(R_v^b \vect{p}_{g/v}^v -
        % \vect{p}_{c/b}^b\right)\right)} {\left(\e_3 R_b^c \left(R_v^b \vect{p}_{g/v}^v -
  % \vect{p}_{c/b}^b\right)\right)^2}
% \end{equation}
% where only the first two (of three) entries of the vector are used. The third
% entry is used for the jacobian w.r.t. $\rho$ as
% \begin{equation}
  % \frac{\partial p_x}{\partial \rho} = -\frac{1}{\rho^2} \frac{\partial p_x}{\partial
  % \vect{p}_{g/v}^v}\left(2\right).
% \end{equation}
% The derivatives for the euler angles are given by
% \begin{equation}
  % \frac{\partial p_x}{\partial \phi} =
  % \frac{f_x \e_1 R_b^c \frac{\partial R_v^b}{\partial \phi} \vect{p}_{g/v}^v}
    % {\left(\e_3 R_b^c \left(R_v^b \vect{p}_{g/v}^v -
    % \vect{p}_{c/b}^b\right)\right)}
    % - \frac{\left(\e_3 R_b^c \frac{\partial R_v^b}{\partial \phi} \vect{p}_{g/v}^v \right) f_x \left(\e_1 R_b^c \left(R_v^b \vect{p}_{g/v}^v -
        % \vect{p}_{c/b}^b\right)\right)} {\left(\e_3 R_b^c \left(R_v^b \vect{p}_{g/v}^v -
  % \vect{p}_{c/b}^b\right)\right)^2}
% \end{equation}
% where the jacobians for the other angles are similar, just substituting in. The
% jacobians for the y pixel measurement, $p_y$ are similar to the ones derived
% above, just changing $f_y$ in place of $f_x$ and $\e_2$ instead of $\e_1$.


% % Python code
% %d1 = -np.matmul(E3, np.matmul(RBC, R_I_b)) * FX * (p_g_c_c[0] / p_g_c_c[2] /
        % %p_g_c_c[2]) + FX * np.matmul(E1, np.matmul(RBC, R_I_b)) \
                % %/ p_g_c_c[2]
% %d2 = -np.matmul(E3, np.matmul(RBC, R_I_b)) * FY * (p_g_c_c[1] / p_g_c_c[2] /
        % %p_g_c_c[2]) + FY * np.matmul(E2, np.matmul(RBC, R_I_b)) \
                % %/ p_g_c_c[2]
% %jac[0, 10:12] = d1[0:2]
% %jac[1, 10:12] = d2[0:2]

% %# d / d rho
% %jac[0, 12] = -d1[2] / rho_g / rho_g
% %jac[1, 12] = -d2[2] / rho_g / rho_g

% %d1dphi = -np.matmul(E3, np.matmul(RBC, np.matmul(dRdPhi, p_g_v_v))) * FX * (p_g_c_c[0] / p_g_c_c[2] /
        % %p_g_c_c[2]) + FX * np.matmul(E1, np.matmul(RBC, np.matmul(dRdPhi,
            % %p_g_v_v))) / p_g_c_c[2]
% %d1dtheta = -np.matmul(E3, np.matmul(RBC, np.matmul(dRdTheta, p_g_v_v))) * FX * (p_g_c_c[0] / p_g_c_c[2] /
        % %p_g_c_c[2]) + FX * np.matmul(E1, np.matmul(RBC, np.matmul(dRdTheta,
            % %p_g_v_v))) / p_g_c_c[2]
% %d1dpsi = -np.matmul(E3, np.matmul(RBC, np.matmul(dRdPsi, p_g_v_v))) * FX * (p_g_c_c[0] / p_g_c_c[2] /
        % %p_g_c_c[2]) + FX * np.matmul(E1, np.matmul(RBC, np.matmul(dRdPsi,
            % %p_g_v_v))) / p_g_c_c[2]

% %jac[0, 3] = d1dphi
% %jac[0, 4] = d1dtheta
% %jac[0, 5] = d1dpsi

% \subsection{Goal Depth Measurement}
% We assume that the camera also provides a measurement of the distance to the
% goal. This is given in the form of the z coordinate of the goal location in the
% camera frame.

% \subsubsection{Measurement Model}
% We can express the goal depth measurement as
% \begin{equation}
  % \vect{z} = \e_3^\transpose \vect{p}_{g/c}^c.
% \end{equation}
% We can expand this to be in terms of our estimated state as
% \begin{align}
  % \vect{z} &= \e_3^\transpose \vect{p}_{g/c}^c \\
  % \vect{z} &= \e_3^\transpose R_b^c R_v^b \vect{p}_{g/c}^v \\
  % %\vect{z} &= \e_3^\transpose R_b^c R_v^b \left( \vect{p}_{g/v}^v \right) \\
  % \vect{z} &= \e_3 R_b^c \left(R_v^b \vect{p}_{g/v}^v -
    % \vect{p}_{c/b}^b\right).
% \end{align}

% \subsubsection{Measurement Jacobian}
% As seen in the measurement model above, the measurement is dependent on the UAV
% attitude as well as the goal position. The jacobian terms for these parts are
% given by
% \begin{align}
  % \frac{\partial \vect{z}}{\partial \phi} &= \e_3^\transpose R_b^c \frac{
    % \partial R_v^b}{\partial \phi} \left( \vect{p}_{g/v}^v \right) \\
  % \frac{\partial \vect{z}}{\partial \theta} &= \e_3^\transpose R_b^c \frac{
    % \partial R_v^b}{\partial \theta} \left( \vect{p}_{g/v}^v \right) \\
  % \frac{\partial \vect{z}}{\partial \psi} &= \e_3^\transpose R_b^c \frac{
    % \partial R_v^b}{\partial \psi} \left( \vect{p}_{g/v}^v \right) \\
    % \frac{\partial \vect{z}}{\partial \vect{p}_{g/v}^v} &= \e_3^\transpose R_b^c
    % R_v^b \\
  % \frac{\partial \vect{z}}{\partial \rho} &= -\frac{1}{\rho^2}
    % \frac{\partial \vect{z}}{\partial \vect{p}_{g/v}^v}\left(2\right).
% \end{align}

% \subsection{Visual Feature Pixel Measurement}
% We assume that feature locations that are rigidly attached to the landing
% vehicle are tracked in the camera image. For now we will just have a fixed
% number of feature that are tracked for the entire duration of the flight, even
% when they leave the FOV of the camera and reenter. In real life, however, the
% feature must be dynamically added and removed from the estimated states vector
% as they leave the FOV of the camera and more are acquired.

% \subsubsection{Measurement Model}
% The measurement is a pixel location of each feature in the camera image
% \begin{equation}
  % \vect{z} =
  % \begin{bmatrix}
    % p_x & p_y
  % \end{bmatrix}^\transpose.
% \end{equation}
% We can expand this measurements using our estimated states
% \begin{align}
  % \vect{z} &=
  % \begin{bmatrix}
    % p_x & p_y
  % \end{bmatrix}^\transpose \\
  % \vect{z} &=
  % \begin{bmatrix}
    % f_x \frac{\e_1 \vect{p}_{i/c}^c}{\e_3 \vect{p}_{i/c}^c} + c_x \\
    % f_y \frac{\e_2 \vect{p}_{i/c}^c}{\e_3 \vect{p}_{i/c}^c} + c_y
  % \end{bmatrix},
% \end{align}
% where $f_x$, $f_y$, $c_x$, and $c_y$ are constant parameters of the camera. We
% can then express the position of the feature $i$ with respect to the camera in the
% camera frame as
% \begin{align}
  % \vect{p}_{i/c}^c &= R_b^c R_v^b \left(\vect{p}_{i/v}^v -
    % \vect{p}_{c/v}^v\right) \\
    % \vect{p}_{i/c}^c &= R_b^c \left(R_v^b \vect{p}_{i/v}^v -
    % \vect{p}_{c/b}^b\right)
% \end{align}
% This is actually a little bit weird because of the state parameters, but
% \begin{equation}
  % \vect{p}_{i/v}^v =
  % \begin{bmatrix}
    % \e_1^\transpose \left( R_g^v \vect{p}_{i/g}^g + \vect{p}_{g/v}^v \right) \\
    % \e_2^\transpose \left( R_g^v \vect{p}_{i/g}^g + \vect{p}_{g/v}^v \right) \\
    % \left( \e_3^\transpose \vect{p}_{i/g}^g + 1 / \rho \right)
  % \end{bmatrix}.
% \end{equation}
=======
>>>>>>> 28198c48be2c86623ca57f088db60d0dfabd3f40

The error-state dynamics used to propagate the filter are found by relating
the modeled true-state dynamics from~\eqref{eq:uav_dynamics},
\eqref{eq:goal_dynamics}, and \eqref{eq:feature_dynamics}
with~\eqref{eq:estimated_dynamics} using the error-state definitions
from~\eqref{eq:est_paper_vector_error_state} and~\eqref{eq:quat_error_state}. The
first-order approximation of the error-state dynamics are given by \MF{TODO put
the right err state dynamics here}
\begin{align}
  \dot{\hat{\vect{p}}}_{b/I}^I
  &=
  \left( \hat{R}_I^b \right)^\transpose \hat{\vect{v}}_{b/I}^b
  \\
  \dot{\hat{\vect{q}}}_{I}^{b} 
  &= 	
  \hat{\q}_I^b \otimes \begin{pmatrix} 0 \\ \frac{1}{2}
  \left( \bar{\vect{\omega}}_{b/I}^b - \hat{\vect{\beta}}_\omega \right)
\end{pmatrix} \nonumber \\
  \dot{\hat{\vect{v}}}_{b/I}^b 
  &=
  \hat{R}_I^b \vect{g}
  +
  \skewmat{\hat{\vect{v}}_{b/I}^b}
  \left( \bar{\vect{\omega}}_{b/I}^b - \hat{\vect{\beta}}_\omega \right)
  +
  \left( \bar{\vect{a}} - \hat{\vect{\beta}}_a \right) \nonumber
  \\
  \dot{\hat{\vect{\beta}}}_a &= \vect{0} \nonumber
  \\
  \dot{\hat{\vect{\beta}}}_\omega &= \vect{0} \nonumber
  \\
  \dot{\hat{\vect{p}}}_{g/b}^{v} &= \left( \hat{R}_{I}^{g} \right)^\transpose
  \hat{\vect{v}}_{g/I}^{g} - \left( \hat{R}_{I}^{b} \right)^\transpose
  \hat{\vect{v}}_{b/I}^{b} \nonumber \\
  \dot{\hat{\vect{v}}}_{g/I}^{g} &= \vect{0} \nonumber \\
  \dot{\hat{\psi}}_{I}^{g} &= \hat{\omega}_{g/I}^g \nonumber \\
  \dot{\hat{\omega}}_{g/I}^{g} &= 0 \nonumber \\
  \dot{\hat{\vect{r}}}_{i/g}^g &= \vect{0}. \nonumber
\end{align}
The derivation of these error-state dynamics can be found in the continuing
subsections.

\subsubsection{UAV Position Error-State Dynamics}
To derive the error-state dynamics for the UAV position state, we start by
differentiating the error-state definition given
in~\eqref{eq:est_paper_uav_pos_err_state} with respect to time to yield
\begin{equation}
  \dot{\tilde{\vect{p}}}_{b/I}^I = \dot{\vect{p}}_{b/I}^I - \dot{\hat{\vect{p}}}_{b/I}^I.
\end{equation}
We then substitue in the corresponding dynamics from~\eqref{eq:uav_dynamics}
and~\eqref{eq:estimated_dynamics}
\begin{align}
  \dot{\tilde{\vect{p}}}_{b/I}^I
  &=
  \left( R_I^b \right)^\transpose \vect{v}_{b/I}^b
  - \left( \hat{R}_I^b \right)^\transpose \hat{\vect{v}}_{b/I}^b.
\end{align}
We expand this equation using the approximation for $\left( R_I^b \right)^\transpose$ given
in~\eqref{eq:est_paper_RTapprox} and the error-state definition given
in~\eqref{eq:est_paper_vector_error_state} resulting in
\begin{align}
  \dot{\tilde{\vect{p}}}_{b/I}^I
  &=
  \left( \hat{R}_I^b \right)^\transpose
  \left( I + \skewmat{\tilde{\vect{\theta}}_I^b} \right)
  \left( \hat{\vect{v}}_{b/I}^b + \tilde{\vect{v}}_{b/I}^b \right)
  - \left( \hat{R}_I^b \right)^\transpose \hat{\vect{v}}_{b/I}^b \\
  &=
  \left( \hat{R}_I^b \right)^\transpose
  \left( \hat{\vect{v}}_{b/I}^b + \tilde{\vect{v}}_{b/I}^b
  + \skewmat{\tilde{\vect{\theta}}_I^b} \hat{\vect{v}}_{b/I}^b
  + \skewmat{\tilde{\vect{\theta}}_I^b} \tilde{\vect{v}}_{b/I}^b \right)
  - \left( \hat{R}_I^b \right)^\transpose \hat{\vect{v}}_{b/I}^b \\
  &=
  \left( \hat{R}_I^b \right)^\transpose \tilde{\vect{v}}_{b/I}^b
  + \left( \hat{R}_I^b \right)^\transpose \skewmat{\tilde{\vect{\theta}}_I^b} \hat{\vect{v}}_{b/I}^b
  + \left( \hat{R}_I^b \right)^\transpose \skewmat{\tilde{\vect{\theta}}_I^b} \tilde{\vect{v}}_{b/I}^b.
\end{align}
We assume the error-state components to be small, neglecting the higher-order
terms to get the final expression
\begin{align}
  \dot{\tilde{\vect{p}}}_{b/I}^I
  &=
  \left( \hat{R}_I^b \right)^\transpose \tilde{\vect{v}}_{b/I}^b
  - \left( \hat{R}_I^b \right)^\transpose \skewmat{\hat{\vect{v}}_{b/I}^b}
  \tilde{\vect{\theta}}_I^b.
\end{align}

\subsubsection{UAV Attitude Error-State Dynamics}
To derive the error-state dynamics for the UAV attitude state, we
follow~\cite{koch2017relative}, starting
with~\eqref{eq:quat_true_state} and letting $\tilde{\q}_I^b = \exp_{\q} \left(
  \tilde{\vect{\theta}}_I^b \right)$ such that
\begin{equation}
  \q_I^b  = \hat{\q}_I^b \otimes \tilde{\q}_I^b
  \right).
\end{equation}
Differentiating with respect to time results in
\begin{equation}
  \dot{\q}_I^b  = \dot{\hat{\q}}_I^b \otimes \tilde{\q}_I^b + 
  \hat{\q}_I^b \otimes \dot{\tilde{\q}}_I^b
  \right)
\end{equation}
which we multiply all terms on the left by $\left( \hat{\q}_I^b \right)^{-1}$
and simplify to yield
\begin{align}
  \left( \hat{\q}_I^b \right)^{-1} \otimes \dot{\q}_I^b  &= \left( \hat{\q}_I^b
  \right)^{-1} \otimes \dot{\hat{\q}}_I^b \otimes \tilde{\q}_I^b + 
  \left( \hat{\q}_I^b \right)^{-1} \otimes \hat{\q}_I^b \otimes \dot{\tilde{\q}}_I^b
  \right) \\
  \left( \hat{\q}_I^b \right)^{-1} \otimes \dot{\q}_I^b  &= \left( \hat{\q}_I^b
  \right)^{-1} \otimes \dot{\hat{\q}}_I^b \otimes \tilde{\q}_I^b + 
  \dot{\tilde{\q}}_I^b
  \right).
\end{align}
We then rearrange the above equation and simplify using the true-state and
estimated-state dynamics of the UAV attitude state given
in~\eqref{eq:uav_dynamics} and~\eqref{eq:estimated_dynamics} along with the
error-state definition from~\eqref{eq:quat_error_state}:
\begin{align}
  \dot{\tilde{\q}}_I^b &=
  \left( \hat{\q}_I^b \right)^{-1} \otimes \dot{\q}_I^b  - \left( \hat{\q}_I^b
  \right)^{-1} \otimes \dot{\hat{\q}}_I^b \otimes \tilde{\q}_I^b
  \right) \\
  &=
  \left( \hat{\q}_I^b \right)^{-1} \otimes \q_I^b \otimes
  \begin{pmatrix}
    0 \\
    \frac{1}{2} \left( \bar{\vect{\omega}}_{b/I}^b - \vect{\beta}_\omega
    -\vect{\upsilon}_\omega \right)
  \end{pmatrix}
  - \left( \hat{\q}_I^b
  \right)^{-1} \otimes \hat{\q}_I^b \otimes
  \begin{pmatrix}
    0 \\
    \frac{1}{2} \left( \bar{\vect{\omega}}_{b/I}^b - \hat{\vect{\beta}}_\omega
  \end{pmatrix}
  \otimes \tilde{\q}_I^b
  \right) \\
  &=
  \label{eq:intermed_eq1}
  \frac{1}{2} \tilde{\q}_I^b \otimes
  \begin{pmatrix}
    0 \\
    \left( \bar{\vect{\omega}}_{b/I}^b - \vect{\beta}_\omega -
    \vect{\upsilon}_\omega \right)
  \end{pmatrix}
  - \frac{1}{2}
  \begin{pmatrix}
    0 \\
    \left( \bar{\vect{\omega}}_{b/I}^b - \hat{\vect{\beta}}_\omega
  \end{pmatrix}
  \otimes \tilde{\q}_I^b
  \right). 
\end{align}
Applying~\eqref{eq:quat_otimes_product} and noting that the quaternion group
operator $\otimes$ can also be written as
\begin{equation}
	\q^a \otimes \q^b = \begin{pmatrix} q_0^b & \left(-\bar{\q}^{b}\right)^\transpose \\ \bar{\q}^b & q^b_0 I - \skewmat{\bar{\q}^b} \end{pmatrix}
	\begin{pmatrix} q^a \\ \bar{\q}^a \end{pmatrix},
\end{equation}
we can expand~\eqref{eq:intermed_eq1} as matrix-like products
\begin{align}
  \dot{\tilde{\q}}_I^b &=
  \frac{1}{2}
  \begin{pmatrix}
    0 & -\left( \bar{\vect{\omega}}_{b/I}^b - \vect{\beta}_\omega -
      \vect{\upsilon}_\omega
    \right)^\transpose \\
      \left( \bar{\vect{\omega}}_{b/I}^b - \vect{\beta}_\omega -
    \vect{\upsilon}_\omega \right) &
    -\skewmat{\bar{\vect{\omega}}_{b/I}^b - \vect{\beta}_\omega -
    \vect{\upsilon}_\omega }
  \end{pmatrix}
  \tilde{\q}_I^b \\
                       & \qquad - \frac{1}{2}
  \begin{pmatrix}
    0 & -\left( \bar{\vect{\omega}}_{b/I}^b - \hat{\vect{\beta}}_\omega
    \right)^\transpose \\
      \left( \bar{\vect{\omega}}_{b/I}^b - \hat{\vect{\beta}}_\omega \right) &
      \skewmat{\bar{\vect{\omega}}_{b/I}^b - \hat{\vect{\beta}}_\omega }
  \end{pmatrix}
  \tilde{\q}_I^b
  \right). \nonumber
\end{align}
We then simplify the previous equation using the error-state definition
$\tilde{\vect{\beta}}_\omega \triangleq \vect{\beta}_\omega -
\hat{\vect{\beta}}_\omega$ giving
\begin{align}
  \dot{\tilde{\q}}_I^b &=
  \frac{1}{2}
  \begin{pmatrix}
    0 & -\left( - \tilde{\vect{\beta}}_\omega -
      \vect{\upsilon}_\omega
    \right)^\transpose \\
    \left( - \tilde{\vect{\beta}}_\omega -
    \vect{\upsilon}_\omega \right) &
    \skewmat{ -2\bar{\vect{\omega}}_{b/I}^b + 2\hat{\vect{\beta}}_\omega
      + \tilde{\vect{\beta}}_\omega + \vect{\upsilon}_\omega }
  \end{pmatrix}
  \tilde{\q}_I^b
\end{align}
which implies that
\begin{align}
  \begin{pmatrix}
    1 \\
    \frac{1}{2} \dot{\tilde{\vect{\theta}}}_I^b
  \end{pmatrix}
  &=
  \frac{1}{2}
  \begin{pmatrix}
    0 & -\left( - \tilde{\vect{\beta}}_\omega -
      \vect{\upsilon}_\omega
    \right)^\transpose \\
    \left( - \tilde{\vect{\beta}}_\omega -
    \vect{\upsilon}_\omega \right) &
    \skewmat{ -2\bar{\vect{\omega}}_{b/I}^b + 2\hat{\vect{\beta}}_\omega
      + \tilde{\vect{\beta}}_\omega + \vect{\upsilon}_\omega }
  \end{pmatrix}
  \begin{pmatrix}
    1 \\
    \frac{1}{2} \tilde{\vect{\theta}}_I^b
  \end{pmatrix}.
\end{align}
To get the final expression, we drop the scalar equation and neglect
second-order terms to yield
\begin{align}
  \dot{\tilde{\vect{\theta}}}_I^b
  &=
    \left( - \tilde{\vect{\beta}}_\omega -
    \vect{\upsilon}_\omega \right) 
    + \frac{1}{2}
    \skewmat{ -2\bar{\vect{\omega}}_{b/I}^b + 2\hat{\vect{\beta}}_\omega
      + \tilde{\vect{\beta}}_\omega + \vect{\upsilon}_\omega }
    \tilde{\vect{\theta}}_I^b \\
  &\approx
  -\skewmat{ \bar{\vect{\omega}}_{b/I}^b - \hat{\vect{\beta}}_\omega}
    \tilde{\vect{\theta}}_I^b
    - \tilde{\vect{\beta}}_\omega -
    \vect{\upsilon}_\omega .
\end{align}

\subsubsection{UAV Velocity Error-State Dynamics}
Similar to the UAV position error-state derivation, to derive the error-state
dynamics for the UAV velocity state, we begin with the time derivative of the
error-state definition
\begin{equation}
  \dot{\tilde{\vect{v}}}_{b/I}^I = \dot{\vect{v}}_{b/I}^I -
  \dot{\hat{\vect{v}}}_{b/I}^I.
\end{equation}
We then substitue in the dynamics from~\eqref{eq:uav_dynamics}
and~\eqref{eq:estimated_dynamics} and expand using the error-state definitions
and approximations
\begin{align}
  \dot{\tilde{\vect{v}}}_{b/I}^I
  =&
  R_I^b \vect{g}
  +
  \skewmat{\vect{v}_{b/I}^b}
  \left( \bar{\vect{\omega}}_{b/I}^b - \vect{\beta}_\omega -
  \vect{\upsilon}_\omega \right)
  +
  \left( \bar{\vect{a}} - \vect{\beta}_a - \vect{\upsilon}_a \right) \\
                                  & \qquad -
                                  \left( \hat{R}_I^b \vect{g}
  +
  \skewmat{\hat{\vect{v}}_{b/I}^b}
  \left( \bar{\vect{\omega}}_{b/I}^b - \hat{\vect{\beta}}_\omega \right)
  +
\left( \bar{\vect{a}} - \hat{\vect{\beta}}_a \right) \right) \nonumber \\
  \approx&
  \left( I - \skewmat{\tilde{\vect{\theta}}_I^b }\right) \hat{R}_I^b \vect{g}
  +
  \skewmat{ \hat{\vect{v}}_{b/I}^b + \tilde{\vect{v}}_{b/I}^b } 
  \left( \bar{\vect{\omega}}_{b/I}^b - \hat{\vect{\beta}}_\omega -
    \tilde{\vect{\beta}}_\omega -
  \vect{\upsilon}_\omega \right) \\
                                  & \qquad 
  + \left( \bar{\vect{a}} - \hat{\vect{\beta}}_a - \tilde{\vect{\beta}}_a - \vect{\upsilon}_a \right)
  - \left( \hat{R}_I^b \vect{g}
  +
  \skewmat{\hat{\vect{v}}_{b/I}^b}
  \left( \bar{\vect{\omega}}_{b/I}^b - \hat{\vect{\beta}}_\omega \right)
  +
\left( \bar{\vect{a}} - \hat{\vect{\beta}}_a \right) \right) \nonumber
\end{align}
which we can simplify and neglect high-order terms to yield the final
expression:
\begin{align}
  \dot{\tilde{\vect{v}}}_{b/I}^I
  \approx&
  \skewmat{ \hat{R}_I^b \vect{g} } \tilde{\vect{\theta}}_I^b 
  -
  \skewmat{ \hat{\vect{v}}_{b/I}^b } \tilde{\vect{\beta}}_\omega
  -
  \skewmat{ \hat{\vect{v}}_{b/I}^b } \vect{\upsilon}_\omega
  -
  \skewmat{ \bar{\omega}_{b/I}^b - \hat{\vect{\beta}}_\omega }
  \tilde{\vect{v}}_{b/I}^b
  -
  \tilde{\vect{\beta}}_a
  -
  \vect{\upsilon}_a.
\end{align}

\subsubsection{UAV Bias Error-State Dynamics}
To derive the error-state dynamics for the UAV bias states, we start with the
time derivative of their error-state definitions
\begin{align}
  \dot{\tilde{\vect{\beta}}}_a &= \dot{\vect{\beta}}_a -
  \dot{\hat{\vect{\beta}}}_a \\
  \dot{\tilde{\vect{\beta}}}_\omega &= \dot{\vect{\beta}}_\omega -
  \dot{\hat{\vect{\beta}}}_\omega
\end{align}
We then substitue in the dynamics from~\eqref{eq:uav_dynamics}
and~\eqref{eq:estimated_dynamics} to yield the final expression
\begin{align}
  \dot{\tilde{\vect{\beta}}}_a &= \vect{\eta}_{\beta_a} \\
  \dot{\tilde{\vect{\beta}}}_\omega &= \vect{\eta}_{\beta_\omega}.
\end{align}

\subsubsection{Landing Vehicle Position Error-State Dynamics}
Similar to the method used for the UAV states, to derive the error-state
dynamics for the landing vehicle position state, we start with the time
derivative of the error-state definition
\begin{equation}
  \dot{\tilde{\vect{p}}}_{g/b}^{v} = \dot{\vect{p}}_{g/b}^{v} -
  \dot{\hat{\vect{p}}}_{g/b}^{v}.
\end{equation}
We then substitue in the dynamics from~\eqref{eq:goal_dynamics}
and~\eqref{eq:estimated_dynamics} and expand using the error-state definitions
and approximations
\begin{align}
  \dot{\tilde{\vect{p}}}_{g/b}^{v}
  &=
  I_{3 \times 2} \left( R_{I}^{g} \right)^\transpose
  \vect{v}_{g/I}^{g} - \left( R_{I}^{b} \right)^\transpose
  \vect{v}_{b/I}^{b}
  -
  \left(
  I_{3 \times 2} \left( \hat{R}_{I}^{g} \right)^\transpose
  \hat{\vect{v}}_{g/I}^{g} - \left( \hat{R}_{I}^{b} \right)^\transpose
  \hat{\vect{v}}_{b/I}^{b}
  \right) \\
  &\approx
  \label{eq:goal_pos_err_state1}
  I_{3 \times 2} \left( \hat{R}_{I}^{g} \right)^\transpose
  \left( R \left( \tilde{\psi}_{I}^{g} \right) \right)^\transpose
  \left( \hat{\vect{v}}_{g/I}^{g} + \tilde{\vect{v}}_{g/I}^{g} \right) - \left(
  \hat{R}_{I}^{b} \right)^\transpose
  \left( I + \skewmat{\tilde{\vect{\theta}}_I^b} \right)
  \left( \hat{\vect{v}}_{b/I}^{b} + \tilde{\vect{v}}_{b/I}^{b} \right) \\
  & \qquad -
  \left(
  I_{3 \times 2} \left( \hat{R}_{I}^{g} \right)^\transpose
  \hat{\vect{v}}_{g/I}^{g} - \left( \hat{R}_{I}^{b} \right)^\transpose
  \hat{\vect{v}}_{b/I}^{b}
  \right). \nonumber
\end{align}
From~\cite{sola2018micro}, by assuming $\tilde{\psi}_I^g$ is small, we can approximate 
\begin{equation}
  \left( R \left( \tilde{\psi}_I^g \right) \right)^\transpose \approx I -
  \skewmat{\tilde{\psi}_I^g}
  \label{eq:2d_err_rot_approx}
\end{equation}
where the two dimensional skew-symmetric matrix is given by
\begin{equation}
  \skewmat{\psi} \triangleq
  \begin{bmatrix}
    0 & -\psi \\
    \psi & 0
  \end{bmatrix}.
\end{equation}
We can then substitue~\eqref{eq:2d_err_rot_approx}
into~\eqref{eq:goal_pos_err_state1} and simplify and neglect higher-order terms
to get the final expression
\begin{align}
  \dot{\tilde{\vect{p}}}_{g/b}^{v}
  &\approx
  \label{eq:goal_pos_err_state1}
  I_{3 \times 2} \left( \hat{R}_{I}^{g} \right)^\transpose
  \left( I - \skewmat{\tilde{\psi}_I^g} \right)
  \left( \hat{\vect{v}}_{g/I}^{g} + \tilde{\vect{v}}_{g/I}^{g} \right)
  - \left( \hat{R}_{I}^{b} \right)^\transpose
  \left( I + \skewmat{\tilde{\vect{\theta}}_I^b} \right)
  \left( \hat{\vect{v}}_{b/I}^{b} + \tilde{\vect{v}}_{b/I}^{b} \right) \\
  & \qquad -
  \left(
  I_{3 \times 2} \left( \hat{R}_{I}^{g} \right)^\transpose
  \hat{\vect{v}}_{g/I}^{g} - \left( \hat{R}_{I}^{b} \right)^\transpose
  \hat{\vect{v}}_{b/I}^{b}
  \right). \nonumber \\
  &\approx
  \label{eq:goal_pos_err_state1}
  I_{3 \times 2} \left( \hat{R}_{I}^{g} \right)^\transpose
  \tilde{\vect{v}}_{g/I}^g
  +
  I_{3 \times 2} \left( \hat{R}_{I}^{g} \right)^\transpose
  \skewmat{ \tilde{\psi}_I^g } \hat{\vect{v}}_{g/I}^g
  +
  \left( \hat{R}_{I}^{b} \right)^\transpose \skewmat{ \hat{\vect{v}}_{b/I}^{b} } 
  \tilde{\vect{\theta}}_I^b
  -
  \left( \hat{R}_{I}^{b} \right)^\transpose \tilde{\vect{v}}_{b/I}^{b} .
\end{align}

\subsubsection{Landing Vehicle Velocity Error-State Dynamics}
\subsubsection{Landing Vehicle Attitude Error-State Dynamics}
\subsubsection{Landing Vehicle Angular Rate Error-State Dynamics}
\subsubsection{Visual Feature Vector Error-State Dynamics}


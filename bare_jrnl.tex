% *** with the testflow diagnostic prior to trusting their LaTeX platform ***
% *** with production work. The IEEE's font choices and paper sizes can   ***
% *** trigger bugs that do not appear when using other class files.       ***                          ***
% The testflow support page is at:
% http://www.michaelshell.org/tex/testflow/

\documentclass[journal]{IEEEtran}
% \documentclass[draft]{IEEEtran}
%
% If IEEEtran.cls has not been installed into the LaTeX system files,
% manually specify the path to it like:
% \documentclass[journal]{../sty/IEEEtran}

%
\usepackage{cite}
\usepackage[hidelinks]{hyperref}
\usepackage{graphicx}
\usepackage{xcolor}
\usepackage{tabularx}
\usepackage{amsmath,amssymb,amsbsy,amsfonts}
\usepackage{accents}
\usepackage{nicefrac}
\usepackage{upgreek}
\usepackage[]{footmisc}
\usepackage{url}

% *** Do not adjust lengths that control margins, column widths, etc. ***
% *** Do not use packages that alter fonts (such as pslatex).         ***
% There should be no need to do such things with IEEEtran.cls V1.6 and later.
% (Unless specifically asked to do so by the journal or conference you plan
% to submit to, of course. )
\input{macros}

\setcounter{MaxMatrixCols}{20}

\hyphenation{michaeldavidfarrell}

\begin{document}
%
\title{
Improving State Estimation of a Landing Target Vehicle
From a Multirotor UAV Using \\
Visual Feature Tracking
}
%
%
% author names and IEEE memberships
% note positions of commas and nonbreaking spaces ( ~ ) LaTeX will not break
% a structure at a ~ so this keeps an author's name from being broken across
% two lines.
% use \thanks{} to gain access to the first footnote area
% a separate \thanks must be used for each paragraph as LaTeX2e's \thanks
% was not built to handle multiple paragraphs
%
\author{Michael~D.~Farrell
  and~Timothy~W.~McLain,~\IEEEmembership{Senior~Member,~IEEE}%
\thanks{Manuscript received January 19, 2020; revised January 26, 2020.
This work was supported by the National Science Foundation STTR Phase
II: Autonomous Landing of Small Unmanned Aircraft Systems onto Moving Platforms,
Award No. IIP-1758678 in partnership with Planck Aerosystems. (\textit{Corresponding
author: Michael D. Farrell}.)}%
\thanks{The authors are with the Department of Mechanical Engineering, Brigham
Young University, Provo, UT, 84602 USA (e-mail: michaeldavidfarrell@gmail.com;
mclain@byu.edu).}%
\thanks{Color versions of one or more of the figures in this paper are available
online at \url{http://ieeexplore.ieee.org}.}%
\thanks{Digital Object Identifier XX.XXXX/TMECH.20XX.XXXXXXX}
% \thanks{Michael Farrell is an M.S. student in the Department of Mechanical
% Engineering, Brigham Young University, Provo, UT, 84602, USA.}% <-this % stops a space
% \thanks{Tim McLain is a Professor in the Department of Mechanical
% Engineering, Brigham Young University, Provo, UT, 84602, USA.}% <-this % stops a space
}

% \author{Michael~Shell,~\IEEEmembership{Member,~IEEE,}
        % John~Doe,~\IEEEmembership{Fellow,~OSA,}
        % and~Jane~Doe,~\IEEEmembership{Life~Fellow,~IEEE}% <-this % stops a space
% \thanks{M. Shell was with the Department
% of Electrical and Computer Engineering, Georgia Institute of Technology, Atlanta,
% GA, 30332 USA e-mail: (see http://www.michaelshell.org/contact.html).}% <-this % stops a space
% \thanks{J. Doe and J. Doe are with Anonymous University.}% <-this % stops a space
% \thanks{Manuscript received April 19, 2005; revised August 26, 2015.}}

% note the % following the last \IEEEmembership and also \thanks - 
% these prevent an unwanted space from occurring between the last author name
% and the end of the author line. i.e., if you had this:
% 
% \author{....lastname \thanks{...} \thanks{...} }
%                     ^------------^------------^----Do not want these spaces!
%
% a space would be appended to the last name and could cause every name on that
% line to be shifted left slightly. This is one of those "LaTeX things". For
% instance, "\textbf{A} \textbf{B}" will typeset as "A B" not "AB". To get
% "AB" then you have to do: "\textbf{A}\textbf{B}"
% \thanks is no different in this regard, so shield the last } of each \thanks
% that ends a line with a % and do not let a space in before the next \thanks.
% Spaces after \IEEEmembership other than the last one are OK (and needed) as
% you are supposed to have spaces between the names. For what it is worth,
% this is a minor point as most people would not even notice if the said evil
% space somehow managed to creep in.



% The paper headers
% \markboth{Journal of \LaTeX\ Class Files,~Vol.~14, No.~8, August~2015}%
% {Shell \MakeLowercase{\textit{et al.}}: Bare Demo of IEEEtran.cls for IEEE Journals}
\markboth{IEEE/ASME TRANSACTIONS ON MECHATRONICS,~VOL.~XX, NO.~X, JANUARY~20XX}%
{FARRELL AND MCLAIN: 
IMPROVING STATE ESTIMATION OF A LANDING TARGET VEHICLE
FROM A MULTIROTOR UAV USING
VISUAL FEATURE TRACKING}
% MODEL-BASED CONTROLLER DESIGN FOR A LIFT-AND-DROP RAILWAY TRACK SWITCH ACTUATOR}
% {Shell \MakeLowercase{\textit{et al.}}: Bare Demo of IEEEtran.cls for IEEE Journals}
% The only time the second header will appear is for the odd numbered pages
% after the title page when using the twoside option.
% 
% *** Note that you probably will NOT want to include the author's ***
% *** name in the headers of peer review papers.                   ***
% You can use \ifCLASSOPTIONpeerreview for conditional compilation here if
% you desire.

% If you want to put a publisher's ID mark on the page you can do it like
% this:
%\IEEEpubid{0000--0000/00\$00.00~\copyright~2015 IEEE}
% Remember, if you use this you must call \IEEEpubidadjcol in the second
% column for its text to clear the IEEEpubid mark.

% use for special paper notices
%\IEEEspecialpapernotice{(Invited Paper)}

% make the title area
\maketitle

% As a general rule, do not put math, special symbols or citations
% in the abstract or keywords.
\begin{abstract}
  We propose an error-state Kalman filter to improve the state estimation of a
  landing target vehicle as observed by an approaching multirotor unmanned aerial
  vehicle (UAV). As is common, we
  assume that the target vehicle is outfitted with a fiducial landing marker.
  In addition to measurements from the fiducial landing marker, the proposed
  estimator uses measurements of visual features which are tracked in the UAV's
  camera frame. We assume that these tracked visual features are rigidly
  attached to the target vehicle such that they provide information to the
  filter about the relative pose between the UAV to the target vehicle. We show
  in both simulation and hardware experiments that the addition of these visual
  features to the filter allows for accurate and consistent estimates of the
  state of the landing vehicle even when the fiducial landing marker is not
  detected for significant periods of time.
\end{abstract}

% Note that keywords are not normally used for peerreview papers.
\begin{IEEEkeywords}
  unmanned aerial vehicles, state estimation, autonomous vehicles,
  visual-inertial data fusion
\end{IEEEkeywords}

% For peer review papers, you can put extra information on the cover
% page as needed:
% \ifCLASSOPTIONpeerreview
% \begin{center} \bfseries EDICS Category: 3-BBND \end{center}
% \fi
%
% For peerreview papers, this IEEEtran command inserts a page break and
% creates the second title. It will be ignored for other modes.
\IEEEpeerreviewmaketitle

% \section{Introduction}
% % The very first letter is a 2 line initial drop letter followed
% % by the rest of the first word in caps.
% % 
% % form to use if the first word consists of a single letter:
% % \IEEEPARstart{A}{demo} file is ....
% % 
% % form to use if you need the single drop letter followed by
% % normal text (unknown if ever used by the IEEE):
% % \IEEEPARstart{A}{}demo file is ....
% % 
% % Some journals put the first two words in caps:
% % \IEEEPARstart{T}{his demo} file is ....
% % 
% % Here we have the typical use of a "T" for an initial drop letter
% % and "HIS" in caps to complete the first word.
% \IEEEPARstart{T}{his} demo file is intended to serve as a ``starter file''
% for IEEE journal papers produced under \LaTeX\ using
% IEEEtran.cls version 1.8b and later.
% % You must have at least 2 lines in the paragraph with the drop letter
% % (should never be an issue)
% I wish you the best of success.

% \hfill mds
 
% \hfill August 26, 2015

\section{Introduction} \label{sec:intro}
% !TEX root=../root.tex

Small multirotor unmanned air vehicles (UAVs) have rapidly become popular platforms for
surveillance, delivery, search and rescue, and a variety of other applications.
The abilities of small, multirotor UAVs to agily operate in confined spaces and to
take off and land vertically give them a unique advantage over other robotic
platforms. For the majority of commercial use cases, these UAVs are required to operate
completely autonomously, as skilled pilots are often not feasible due to
limitations in sight of or communication with the UAV. Though many aspects of
multirotor UAV autonomy continue to be researched and developed, operation in
static environments has largely been solved. A variety of emerging use cases,
however, require the multirotor UAV to operate from a moving vehicle. These use
cases include martitime surveillance, where the UAV must take off and land from
a martitime vessel at sea, governmental surveillance, where the UAV must operate
from a truck, and package delivery, where the UAV operates from a large truck
carrying the packages to be delivered. The ability to operate reliably from a
moving vehicle is still a very active field of research.

A variety of approaches to the operation of multirotor UAVs with respect to
moving vehicles have been proposed. Nearly all of these approaches rely on the
detection of a visual fiducial marker on the moving vehicle

\cite{wenzel2011automatic} relys on infrared (IR) LED markers.
\cite{ling2014precision} AprilTag, kalman filter to predict
\cite{araar2017vision} relies on a known map of many AprilTags fiducials.
\cite{borowczyk2017autonomous} uses a single AprilTag
\cite{baca2019autonomous} main MBZIRC 2017 paper (tag = square with x)
\cite{falanga2017vision} MBZIRC, SVO + IMU
\cite{beul2017fast} MBZIRC
\cite{cantelli2017autonomous} MBZIRC
\cite{marantos2018vision} helicopter, tag (Aruco/April)

\cite{lee2012autonomous} IBVS
\cite{wynn2019visual} IBVS - nested tag

All of the previous approaches break if the fiducial marker is not detected for
significant periods of time. Ling notes that it is important to model the
dynamics of the landing vehicle so that they can be propagated forward for short
periods of time when the fiducial marker is not
detected~\cite{ling2014precision}, however, this method is only as good as the
motion model of the vehicle is. We propose an estimation algorithm that detects,
tracks, and estimates unknown visual features on the landing vehicle to improve
estimation accuracy in these cases.

The proposed method of detecting, tracking, and estimating visual features is
similar to that commonly used in the field of visual odometry. Many methods such
as VINS-MONO, OKVIS, MSCK, and ORB SLAM use an indirect approach
\cite{qin2018vins} VINS-MONO, feature extraction, optical flow frame to frame
\cite{leutenegger2013keyframe} OKVIS, BRISK features
\cite{mourikis2007multi} MSCKF, uses SIFT features
\cite{mur2015orb} ORB SLAM

In these methods, the tracked visual features are assumed to belong to the
static world. A method using visual odometry has been implemented and used for landing on
moving platforms~\cite{falanga2017vision}, however, only static features are
still tracked. During the landing phase, it is common for the landing vehicle
to occupy almost the entire field of view of the UAV's camera. This makes
tracking static features impossible. The estimator we propose, instead, tracks
visual features that are rigidly attached to the landing vehicle. By fitting a
motion model to these features, their measurements provide information about
the movement of the landing vehicle. 


% !TEX root=../root.tex

The outline of the paper is as follows.
\secref{sec:model} explains the mathematical notation used throughout the paper.
\secref{sec:estimation} presents the proposed estimation algorithm including the
state dynamics, state initialization and measurement models.
\secref{sec:est_paper_simulation} describes the simulation experiments conducted and
\secref{sec:est_paper_hardware} describes the hardware experiments conducted.
\secref{sec:conclusion} provides concluding remarks.


% needed in second column of first page if using \IEEEpubid
%\IEEEpubidadjcol

\section{Mathematical Preliminaries} \label{sec:math_prelim}
\input{model/notation}
\input{model/quaternions}
% TEX root=../root.tex

\subsection{S1}
We treat S1 as a vector space but make sure to wrap residuals between -pi and
pi. 

Also a 2D rotation matrix can be made like this...
And a 3D rotation matrix like this..


\section{Estimation} \label{sec:estimation}
% !TEX root=./root.tex

As mentioned in~\secref{sec:BLAH}, the propsed estimator estimates both the state of
the UAV and the state of the landing vehicle as well as the positions of visual
landmarks on the landing vehicle.
% in the same Error-State Kalman
% filter~\cite{sola2017quaternion}.
We express the state of the combined estimated system as the tuple
\begin{equation}
  \hat{\x} =
  \begin{pmatrix}
    \hat{\x}_{\text{UAV}}, \hat{\x}_{\text{Goal}}, \hat{\x}_{\text{Landmarks}}
  \end{pmatrix}
\end{equation}
with the components defined as
\begin{align}
  \hat{\vect{x}}_{\text{UAV}} &=
  \begin{pmatrix}
    \hat{\vect{p}}_{b/I}^{I}, \hat{\vect{q}}_I^{b}, \hat{\vect{v}}_{b/I}^b,
    \hat{\vect{\beta}}_a,
    \hat{\vect{\beta}}_{\omega}
  \end{pmatrix}
  \in \mathbb{R}^3 \times S^3 \times \mathbb{R}^3 \times \mathbb{R}^3 \times
    \mathbb{R}^3  \\
  % \x_{\text{UAV}} &=
  % \begin{bmatrix}
    % \vect{p}_{b/I}^I &
    % \phi & \theta & \psi &
    % \vect{v}_{b/I}^b &
    % \mu & \vect{\beta}_a & \vect{\beta}_\omega
  % \end{bmatrix}^\transpose \\
    \hat{\x}_{\text{Goal}} & =
    \begin{pmatrix}
      \hat{\vect{p}}_{g/b}^{v}, \hat{\vect{v}}_{g/I}^{g}, \theta_{I}^{g},
      \omega_{g/I}^{g}
    \end{pmatrix}
    \in \mathbb{R}^3 \times \mathbb{R}^2 \times S^1 \times \mathbb{R}^1
    \\
    \hat{\x}_{\text{Landmarks}} & =
    \begin{pmatrix}
      \hat{\vect{r}}_{1/g}^{g}, \dots \hat{\vect{r}}_{n/g}^{g}
    \end{pmatrix}
    \in \mathbb{R}^3 \dots \mathbb{R}^3.
\end{align}
Note that $\hat{\vect{x}}_{\text{UAV}}$ contains the same states mentioned previously
in~\secref{sec:UAV_dynamics} with the addition of $\hat{\vect{\beta}}_a$ and
$\hat{\vect{\beta}}_\omega$, the estimated bias vectors for the acclerometer and
gyroscope sensors. On the
other hand, $\hat{\vect{x}}_{\text{Goal}}$ varies 
from the landing vehicle states mentioned in~\secref{sec:landing_veh_dynamics}
by containing $\hat{\vect{p}}_{g/b}^v$ instead of $\hat{\vect{p}}_{b/I}^I$.
% as well as $\hat{\vect{r}}_{1/g}^{g} \dots \hat{\vect{r}}_{n/g}^{g}$.
We estimate the relative state, $\hat{\vect{p}}_{g/b}^v$, instead
of the global state, $\hat{\vect{p}}_{g/I}^I$, as the relative state is observable
even with poor estimates of the UAV's global position, $\hat{\vect{p}}_{b/I}^I$.
The estimated vectors $\hat{\vect{r}}_{1/g}^{g} \dots \hat{\vect{r}}_{n/g}^{g}$ represent the
locations of visual landmarks $1 \dots n$ which are rigidly attached to the
landing vehicle. We show in our simulation and hardware experiments, that the
addition of these visual landmarks to the estimated state allows the estimator
to maintain accurate and consistent estimates of $\hat{\x}_{\text{Goal}}$ even
while the fiducial landing marker is not detected for long periods of time. 
% As mentioned in~\secref{sec:intro??} the estimation of these
% visual landmarks allows the estimator to maintain accurate and consistent
% estimates of the landing vehicle even while the fiducial landing marker is not
% detected for long periods of time.

The inputs to the system are given by
\begin{equation*}
  \vect{u} = \begin{pmatrix} \bar{\vect{a}}_{b/I}^I, \bar{\vect{\omega}}_{b/I}^b \end{pmatrix} \in
        \mathbb{R}^3 \times \mathbb{R}^3,
\end{equation*}
which are directly measured from the inertial measurement unit (IMU) on the UAV.

As the estimated state is not a vector, but rather a tuple of Lie groups, we
employ the Error-State Kalman Filter (ESKF) as described in~\cite{BLAH}. We note
that the estimated state, however, is of dynamic size. As visual landmarks are
detected they are added to the state vector until a maximum size of the state
vector is reached. As the visual landmarks leave the field of view of the camera
or are otherwise no longer tracked, they are removed from the estimated state
vector, making room for new visual landmarks to be added. In the following
subsections, we describe the dynamic equations of the estimated state, the
initialization of certain states, and the
measurement models used to update the filter.

% !TEX root=./root.tex

\subsection{Propagation Model}
We use common rigid body kinematics to model the dynamics of
the UAV given by
\begin{align}
  \dot{\vect{p}}_{b/I}^I
  &=
  \left( R_I^b \right)^\transpose \vect{v}_{b/I}^b
  \label{eq:uav_dynamics}
  \\
  \dot{\vect{q}}_{I}^{b} 
	&= 	
  \q_I^b \otimes \begin{pmatrix} 0 \\ \frac{1}{2}
    \left( \bar{\vect{\omega}}_{b/I}^b - \vect{\beta}_\omega - \vect{\upsilon}_\omega \right)
\end{pmatrix} \nonumber \\
  \dot{\vect{v}}_{b/I}^b 
  &=
  R_I^b \vect{g}
  +
  \skewmat{\vect{v}_{b/I}^b}
  \left( \bar{\vect{\omega}}_{b/I}^b - \vect{\beta}_\omega -
  \vect{\upsilon}_\omega \right)
  +
  \left( \bar{\vect{a}} - \vect{\beta}_a - \vect{\upsilon}_a \right) \nonumber
  % \vect{\eta}_v 
  \\
  \dot{\vect{\beta}}_a &= \vect{\eta}_{\beta_a} \nonumber
  \\
  \dot{\vect{\beta}}_\omega &= \vect{\eta}_{\beta_\omega} \nonumber
\end{align}
% $\vect{\eta}_v$, 
where $\vect{\eta}_{\beta_a}$, $\vect{\eta}_{\beta_\omega}$,
$\vect{\upsilon}_\omega$, and $\vect{\upsilon}_a$
are zero-mean Gaussian noise processes.

We model the motion of the landing vehicle
with a constant velocity and constant
angular velocity motion model such that
% Though this is a very simplified model, we show
% in~\secref{sec:simulation} and~\secref{sec:hardware} that it is sufficient for
% cases that do not perfectly match this model. The dynamics of $\x_{\text{Goal}}$ are expressed as
% \begin{align}
  % \dot{\hat{\vect{p}}}_{g/b}^{v} &= \left( \hat{R}_{I}^{g} \right)^\transpose
  % \hat{\vect{v}}_{g/I}^{g} - \left( \hat{R}_{I}^{b} \right)^\transpose
  % \hat{\vect{v}}_{b/I}^{b} \label{eq:goal_dynamics} \\
  % \dot{\hat{\vect{v}}}_{g/I}^{g} &= \vect{0} \nonumber \\
  % \dot{\hat{\theta}}_{I}^{g} &= \hat{\omega}_{g/I}^g \nonumber \\
  % \dot{\hat{\omega}}_{g/I}^{g} &= 0. \nonumber
% \end{align}
\begin{align}
  \dot{\vect{p}}_{g/b}^{v} &= \left( R_{I}^{g} \right)^\transpose
  \vect{v}_{g/I}^{g} - \left( R_{I}^{b} \right)^\transpose
  \vect{v}_{b/I}^{b} \label{eq:goal_dynamics} \\
  \dot{\vect{v}}_{g/I}^{g} &= \vect{\eta}_{gv} \nonumber \\
  \dot{\psi}_{I}^{g} &= \omega_{g/I}^g \nonumber \\
  \dot{\omega}_{g/I}^{g} &= \eta_{g\omega} \nonumber
\end{align}
where $\vect{\eta}_{gv}$ and $\eta_{g\omega}$ are zero-mean Gaussian noise
processes. Though this is a very simplified motion model for the landing
vehicle, we show in~\secref{sec:simulation} and~\secref{sec:hardware} that is is
satisfactory for our experiments. We intend the motion model of the landing
vehicle to be easily modified for landing vehicles with more complex motion such
as a boat at sea.

As mentioned previously, we assume the tracked visual features are rigidly
attached to the landing vehicle such that
% there are no dynamics 
% of their associated states
\begin{equation}
  \dot{\vect{r}}_{i/g}^g = \vect{0}. \label{eq:feature_dynamics}
  % \dot{\hat{\vect{r}}}_{i/g}^g = \vect{0}. \label{eq:feature_dynamics}
\end{equation}

In the ESKF, the estimated state is propagated independently of the filter using
the expected value of the modeled dynamics. We use the expected values
of~\eqref{eq:uav_dynamics}, \eqref{eq:goal_dynamics}, and
\eqref{eq:feature_dynamics} given by
% The estimated state is propagated
\begin{align}
  \dot{\hat{\vect{p}}}_{b/I}^I
  &=
  \left( \hat{R}_I^b \right)^\transpose \hat{\vect{v}}_{b/I}^b
  \label{eq:estimated_dynamics}
  \\
  \dot{\hat{\vect{q}}}_{I}^{b} 
  &= 	
  \hat{\q}_I^b \otimes \begin{pmatrix} 0 \\ \frac{1}{2}
  \left( \bar{\vect{\omega}}_{b/I}^b - \hat{\vect{\beta}}_\omega \right)
\end{pmatrix} \nonumber \\
  \dot{\hat{\vect{v}}}_{b/I}^b 
  &=
  \hat{R}_I^b \vect{g}
  +
  \skewmat{\hat{\vect{v}}_{b/I}^b}
  \left( \bar{\vect{\omega}}_{b/I}^b - \hat{\vect{\beta}}_\omega \right)
  +
  \left( \bar{\vect{a}} - \hat{\vect{\beta}}_a \right) \nonumber
  \\
  \dot{\hat{\vect{\beta}}}_a &= \vect{0} \nonumber
  \\
  \dot{\hat{\vect{\beta}}}_\omega &= \vect{0} \nonumber
  \\
  \dot{\hat{\vect{p}}}_{g/b}^{v} &= \left( \hat{R}_{I}^{g} \right)^\transpose
  \hat{\vect{v}}_{g/I}^{g} - \left( \hat{R}_{I}^{b} \right)^\transpose
  \hat{\vect{v}}_{b/I}^{b} \nonumber \\
  \dot{\hat{\vect{v}}}_{g/I}^{g} &= \vect{0} \nonumber \\
  \dot{\hat{\psi}}_{I}^{g} &= \hat{\omega}_{g/I}^g \nonumber \\
  \dot{\hat{\omega}}_{g/I}^{g} &= 0 \nonumber \\
  \dot{\hat{\vect{r}}}_{i/g}^g &= \vect{0}. \nonumber
\end{align}

The error-state dynamics used to propagate the filter are found by relating
the modeled true-state dynamics from~\eqref{eq:uav_dynamics},
\eqref{eq:goal_dynamics}, and \eqref{eq:feature_dynamics}
with~\eqref{eq:estimated_dynamics} using the error-state definitions
from~\eqref{eq:est_paper_vector_error_state} and~\eqref{eq:quat_error_state}. The
first-order approximation of the error-state dynamics are given by \MF{TODO put
the right err state dynamics here}
\begin{align}
  \dot{\hat{\vect{p}}}_{b/I}^I
  &=
  \left( \hat{R}_I^b \right)^\transpose \hat{\vect{v}}_{b/I}^b
  \\
  \dot{\hat{\vect{q}}}_{I}^{b} 
  &= 	
  \hat{\q}_I^b \otimes \begin{pmatrix} 0 \\ \frac{1}{2}
  \left( \bar{\vect{\omega}}_{b/I}^b - \hat{\vect{\beta}}_\omega \right)
\end{pmatrix} \nonumber \\
  \dot{\hat{\vect{v}}}_{b/I}^b 
  &=
  \hat{R}_I^b \vect{g}
  +
  \skewmat{\hat{\vect{v}}_{b/I}^b}
  \left( \bar{\vect{\omega}}_{b/I}^b - \hat{\vect{\beta}}_\omega \right)
  +
  \left( \bar{\vect{a}} - \hat{\vect{\beta}}_a \right) \nonumber
  \\
  \dot{\hat{\vect{\beta}}}_a &= \vect{0} \nonumber
  \\
  \dot{\hat{\vect{\beta}}}_\omega &= \vect{0} \nonumber
  \\
  \dot{\hat{\vect{p}}}_{g/b}^{v} &= \left( \hat{R}_{I}^{g} \right)^\transpose
  \hat{\vect{v}}_{g/I}^{g} - \left( \hat{R}_{I}^{b} \right)^\transpose
  \hat{\vect{v}}_{b/I}^{b} \nonumber \\
  \dot{\hat{\vect{v}}}_{g/I}^{g} &= \vect{0} \nonumber \\
  \dot{\hat{\psi}}_{I}^{g} &= \hat{\omega}_{g/I}^g \nonumber \\
  \dot{\hat{\omega}}_{g/I}^{g} &= 0 \nonumber \\
  \dot{\hat{\vect{r}}}_{i/g}^g &= \vect{0}. \nonumber
\end{align}
The derivation of these error-state dynamics can be found in the continuing
subsections.

\subsubsection{UAV Position Error-State Dynamics}
To derive the error-state dynamics for the UAV position state, we start by
differentiating the error-state definition given
in~\eqref{eq:est_paper_uav_pos_err_state} with respect to time to yield
\begin{equation}
  \dot{\tilde{\vect{p}}}_{b/I}^I = \dot{\vect{p}}_{b/I}^I - \dot{\hat{\vect{p}}}_{b/I}^I.
\end{equation}
We then substitue in the corresponding dynamics from~\eqref{eq:uav_dynamics}
and~\eqref{eq:estimated_dynamics}
\begin{align}
  \dot{\tilde{\vect{p}}}_{b/I}^I
  &=
  \left( R_I^b \right)^\transpose \vect{v}_{b/I}^b
  - \left( \hat{R}_I^b \right)^\transpose \hat{\vect{v}}_{b/I}^b.
\end{align}
We expand this equation using the approximation for $\left( R_I^b \right)^\transpose$ given
in~\eqref{eq:est_paper_RTapprox} and the error-state definition given
in~\eqref{eq:est_paper_vector_error_state} resulting in
\begin{align}
  \dot{\tilde{\vect{p}}}_{b/I}^I
  &=
  \left( \hat{R}_I^b \right)^\transpose
  \left( I + \skewmat{\tilde{\vect{\theta}}_I^b} \right)
  \left( \hat{\vect{v}}_{b/I}^b + \tilde{\vect{v}}_{b/I}^b \right)
  - \left( \hat{R}_I^b \right)^\transpose \hat{\vect{v}}_{b/I}^b \\
  &=
  \left( \hat{R}_I^b \right)^\transpose
  \left( \hat{\vect{v}}_{b/I}^b + \tilde{\vect{v}}_{b/I}^b
  + \skewmat{\tilde{\vect{\theta}}_I^b} \hat{\vect{v}}_{b/I}^b
  + \skewmat{\tilde{\vect{\theta}}_I^b} \tilde{\vect{v}}_{b/I}^b \right)
  - \left( \hat{R}_I^b \right)^\transpose \hat{\vect{v}}_{b/I}^b \\
  &=
  \left( \hat{R}_I^b \right)^\transpose \tilde{\vect{v}}_{b/I}^b
  + \left( \hat{R}_I^b \right)^\transpose \skewmat{\tilde{\vect{\theta}}_I^b} \hat{\vect{v}}_{b/I}^b
  + \left( \hat{R}_I^b \right)^\transpose \skewmat{\tilde{\vect{\theta}}_I^b} \tilde{\vect{v}}_{b/I}^b.
\end{align}
We assume the error-state components to be small, neglecting the higher-order
terms to get the final expression
\begin{align}
  \dot{\tilde{\vect{p}}}_{b/I}^I
  &=
  \left( \hat{R}_I^b \right)^\transpose \tilde{\vect{v}}_{b/I}^b
  - \left( \hat{R}_I^b \right)^\transpose \skewmat{\hat{\vect{v}}_{b/I}^b}
  \tilde{\vect{\theta}}_I^b.
\end{align}

\subsubsection{UAV Attitude Error-State Dynamics}
To derive the error-state dynamics for the UAV attitude state, we
follow~\cite{koch2017relative}, starting
with~\eqref{eq:quat_true_state} and letting $\tilde{\q}_I^b = \exp_{\q} \left(
  \tilde{\vect{\theta}}_I^b \right)$ such that
\begin{equation}
  \q_I^b  = \hat{\q}_I^b \otimes \tilde{\q}_I^b
  \right).
\end{equation}
Differentiating with respect to time results in
\begin{equation}
  \dot{\q}_I^b  = \dot{\hat{\q}}_I^b \otimes \tilde{\q}_I^b + 
  \hat{\q}_I^b \otimes \dot{\tilde{\q}}_I^b
  \right)
\end{equation}
which we multiply all terms on the left by $\left( \hat{\q}_I^b \right)^{-1}$
and simplify to yield
\begin{align}
  \left( \hat{\q}_I^b \right)^{-1} \otimes \dot{\q}_I^b  &= \left( \hat{\q}_I^b
  \right)^{-1} \otimes \dot{\hat{\q}}_I^b \otimes \tilde{\q}_I^b + 
  \left( \hat{\q}_I^b \right)^{-1} \otimes \hat{\q}_I^b \otimes \dot{\tilde{\q}}_I^b
  \right) \\
  \left( \hat{\q}_I^b \right)^{-1} \otimes \dot{\q}_I^b  &= \left( \hat{\q}_I^b
  \right)^{-1} \otimes \dot{\hat{\q}}_I^b \otimes \tilde{\q}_I^b + 
  \dot{\tilde{\q}}_I^b
  \right).
\end{align}
We then rearrange the above equation and simplify using the true-state and
estimated-state dynamics of the UAV attitude state given
in~\eqref{eq:uav_dynamics} and~\eqref{eq:estimated_dynamics} along with the
error-state definition from~\eqref{eq:quat_error_state}:
\begin{align}
  \dot{\tilde{\q}}_I^b &=
  \left( \hat{\q}_I^b \right)^{-1} \otimes \dot{\q}_I^b  - \left( \hat{\q}_I^b
  \right)^{-1} \otimes \dot{\hat{\q}}_I^b \otimes \tilde{\q}_I^b
  \right) \\
  &=
  \left( \hat{\q}_I^b \right)^{-1} \otimes \q_I^b \otimes
  \begin{pmatrix}
    0 \\
    \frac{1}{2} \left( \bar{\vect{\omega}}_{b/I}^b - \vect{\beta}_\omega
    -\vect{\upsilon}_\omega \right)
  \end{pmatrix}
  - \left( \hat{\q}_I^b
  \right)^{-1} \otimes \hat{\q}_I^b \otimes
  \begin{pmatrix}
    0 \\
    \frac{1}{2} \left( \bar{\vect{\omega}}_{b/I}^b - \hat{\vect{\beta}}_\omega
  \end{pmatrix}
  \otimes \tilde{\q}_I^b
  \right) \\
  &=
  \label{eq:intermed_eq1}
  \frac{1}{2} \tilde{\q}_I^b \otimes
  \begin{pmatrix}
    0 \\
    \left( \bar{\vect{\omega}}_{b/I}^b - \vect{\beta}_\omega -
    \vect{\upsilon}_\omega \right)
  \end{pmatrix}
  - \frac{1}{2}
  \begin{pmatrix}
    0 \\
    \left( \bar{\vect{\omega}}_{b/I}^b - \hat{\vect{\beta}}_\omega
  \end{pmatrix}
  \otimes \tilde{\q}_I^b
  \right). 
\end{align}
Applying~\eqref{eq:quat_otimes_product} and noting that the quaternion group
operator $\otimes$ can also be written as
\begin{equation}
	\q^a \otimes \q^b = \begin{pmatrix} q_0^b & \left(-\bar{\q}^{b}\right)^\transpose \\ \bar{\q}^b & q^b_0 I - \skewmat{\bar{\q}^b} \end{pmatrix}
	\begin{pmatrix} q^a \\ \bar{\q}^a \end{pmatrix},
\end{equation}
we can expand~\eqref{eq:intermed_eq1} as matrix-like products
\begin{align}
  \dot{\tilde{\q}}_I^b &=
  \frac{1}{2}
  \begin{pmatrix}
    0 & -\left( \bar{\vect{\omega}}_{b/I}^b - \vect{\beta}_\omega -
      \vect{\upsilon}_\omega
    \right)^\transpose \\
      \left( \bar{\vect{\omega}}_{b/I}^b - \vect{\beta}_\omega -
    \vect{\upsilon}_\omega \right) &
    -\skewmat{\bar{\vect{\omega}}_{b/I}^b - \vect{\beta}_\omega -
    \vect{\upsilon}_\omega }
  \end{pmatrix}
  \tilde{\q}_I^b \\
                       & \qquad - \frac{1}{2}
  \begin{pmatrix}
    0 & -\left( \bar{\vect{\omega}}_{b/I}^b - \hat{\vect{\beta}}_\omega
    \right)^\transpose \\
      \left( \bar{\vect{\omega}}_{b/I}^b - \hat{\vect{\beta}}_\omega \right) &
      \skewmat{\bar{\vect{\omega}}_{b/I}^b - \hat{\vect{\beta}}_\omega }
  \end{pmatrix}
  \tilde{\q}_I^b
  \right). \nonumber
\end{align}
We then simplify the previous equation using the error-state definition
$\tilde{\vect{\beta}}_\omega \triangleq \vect{\beta}_\omega -
\hat{\vect{\beta}}_\omega$ giving
\begin{align}
  \dot{\tilde{\q}}_I^b &=
  \frac{1}{2}
  \begin{pmatrix}
    0 & -\left( - \tilde{\vect{\beta}}_\omega -
      \vect{\upsilon}_\omega
    \right)^\transpose \\
    \left( - \tilde{\vect{\beta}}_\omega -
    \vect{\upsilon}_\omega \right) &
    \skewmat{ -2\bar{\vect{\omega}}_{b/I}^b + 2\hat{\vect{\beta}}_\omega
      + \tilde{\vect{\beta}}_\omega + \vect{\upsilon}_\omega }
  \end{pmatrix}
  \tilde{\q}_I^b
\end{align}
which implies that
\begin{align}
  \begin{pmatrix}
    1 \\
    \frac{1}{2} \dot{\tilde{\vect{\theta}}}_I^b
  \end{pmatrix}
  &=
  \frac{1}{2}
  \begin{pmatrix}
    0 & -\left( - \tilde{\vect{\beta}}_\omega -
      \vect{\upsilon}_\omega
    \right)^\transpose \\
    \left( - \tilde{\vect{\beta}}_\omega -
    \vect{\upsilon}_\omega \right) &
    \skewmat{ -2\bar{\vect{\omega}}_{b/I}^b + 2\hat{\vect{\beta}}_\omega
      + \tilde{\vect{\beta}}_\omega + \vect{\upsilon}_\omega }
  \end{pmatrix}
  \begin{pmatrix}
    1 \\
    \frac{1}{2} \tilde{\vect{\theta}}_I^b
  \end{pmatrix}.
\end{align}
To get the final expression, we drop the scalar equation and neglect
second-order terms to yield
\begin{align}
  \dot{\tilde{\vect{\theta}}}_I^b
  &=
    \left( - \tilde{\vect{\beta}}_\omega -
    \vect{\upsilon}_\omega \right) 
    + \frac{1}{2}
    \skewmat{ -2\bar{\vect{\omega}}_{b/I}^b + 2\hat{\vect{\beta}}_\omega
      + \tilde{\vect{\beta}}_\omega + \vect{\upsilon}_\omega }
    \tilde{\vect{\theta}}_I^b \\
  &\approx
  -\skewmat{ \bar{\vect{\omega}}_{b/I}^b - \hat{\vect{\beta}}_\omega}
    \tilde{\vect{\theta}}_I^b
    - \tilde{\vect{\beta}}_\omega -
    \vect{\upsilon}_\omega .
\end{align}

\subsubsection{UAV Velocity Error-State Dynamics}
Similar to the UAV position error-state derivation, to derive the error-state
dynamics for the UAV velocity state, we begin with the time derivative of the
error-state definition
\begin{equation}
  \dot{\tilde{\vect{v}}}_{b/I}^I = \dot{\vect{v}}_{b/I}^I -
  \dot{\hat{\vect{v}}}_{b/I}^I.
\end{equation}
We then substitue in the dynamics from~\eqref{eq:uav_dynamics}
and~\eqref{eq:estimated_dynamics} and expand using the error-state definitions
and approximations
\begin{align}
  \dot{\tilde{\vect{v}}}_{b/I}^I
  =&
  R_I^b \vect{g}
  +
  \skewmat{\vect{v}_{b/I}^b}
  \left( \bar{\vect{\omega}}_{b/I}^b - \vect{\beta}_\omega -
  \vect{\upsilon}_\omega \right)
  +
  \left( \bar{\vect{a}} - \vect{\beta}_a - \vect{\upsilon}_a \right) \\
                                  & \qquad -
                                  \left( \hat{R}_I^b \vect{g}
  +
  \skewmat{\hat{\vect{v}}_{b/I}^b}
  \left( \bar{\vect{\omega}}_{b/I}^b - \hat{\vect{\beta}}_\omega \right)
  +
\left( \bar{\vect{a}} - \hat{\vect{\beta}}_a \right) \right) \nonumber \\
  \approx&
  \left( I - \skewmat{\tilde{\vect{\theta}}_I^b }\right) \hat{R}_I^b \vect{g}
  +
  \skewmat{ \hat{\vect{v}}_{b/I}^b + \tilde{\vect{v}}_{b/I}^b } 
  \left( \bar{\vect{\omega}}_{b/I}^b - \hat{\vect{\beta}}_\omega -
    \tilde{\vect{\beta}}_\omega -
  \vect{\upsilon}_\omega \right) \\
                                  & \qquad 
  + \left( \bar{\vect{a}} - \hat{\vect{\beta}}_a - \tilde{\vect{\beta}}_a - \vect{\upsilon}_a \right)
  - \left( \hat{R}_I^b \vect{g}
  +
  \skewmat{\hat{\vect{v}}_{b/I}^b}
  \left( \bar{\vect{\omega}}_{b/I}^b - \hat{\vect{\beta}}_\omega \right)
  +
\left( \bar{\vect{a}} - \hat{\vect{\beta}}_a \right) \right) \nonumber
\end{align}
which we can simplify and neglect high-order terms to yield the final
expression:
\begin{align}
  \dot{\tilde{\vect{v}}}_{b/I}^I
  \approx&
  \skewmat{ \hat{R}_I^b \vect{g} } \tilde{\vect{\theta}}_I^b 
  -
  \skewmat{ \hat{\vect{v}}_{b/I}^b } \tilde{\vect{\beta}}_\omega
  -
  \skewmat{ \hat{\vect{v}}_{b/I}^b } \vect{\upsilon}_\omega
  -
  \skewmat{ \bar{\omega}_{b/I}^b - \hat{\vect{\beta}}_\omega }
  \tilde{\vect{v}}_{b/I}^b
  -
  \tilde{\vect{\beta}}_a
  -
  \vect{\upsilon}_a.
\end{align}

\subsubsection{UAV Bias Error-State Dynamics}
\subsubsection{Landing Vehicle Position Error-State Dynamics}
\subsubsection{Landing Vehicle Velocity Error-State Dynamics}
\subsubsection{Landing Vehicle Attitude Error-State Dynamics}
\subsubsection{Landing Vehicle Angular Rate Error-State Dynamics}
\subsubsection{Visual Feature Vector Error-State Dynamics}


% !TEX root=./root.tex

\subsection{Measurement Models}

\subsubsection{Global UAV Position Measurement}
We assume to have some measurement of the position of the UAV with respect to
the inertial
frame. This measurement may come from a sensor such as GPS or a motion capture system.
The measurement model and its estimate can
be written as
\begin{align}
  h_{\text{pos}} \left( \x \right) &= \vect{p}_{b/I}^I +
  \vect{\eta}_{\text{pos}} \\
  h_{\text{pos}} \left( \hat{\x} \right) &= \hat{\vect{p}}_{b/I}^I.
\end{align}
For a given measurement of position, $\vect{z}_{\text{pos}}$, the residual is
\begin{equation*}
  \vect{r}_{\text{pos}} = \vect{z}_{\text{pos}} - h_{\text{pos}} \left( \hat{\x}
  \right),
\end{equation*}
which for the error-state Kalman filter is modeled as
\begin{align*}
  \vect{r}_{\text{pos}} &=  h_{\text{pos}} \left( \x \right) - h_{\text{pos}} \left( \hat{\x}
  \right), \\
                        &= \vect{p}_{b/I}^I + \eta_{\text{pos}} -
                        \hat{\vect{p}}_{b/I}^I \\
                        &= \tilde{\vect{p}}_{b/I}^I + \eta_{\text{pos}}.
\end{align*}
This results in the measurement jacobian
\begin{equation*}
  H_{\text{pos}} =
  \begin{bmatrix}
    I_{3 \times 3} & \vect{0} & \vect{0} & \dots & \vect{0}
  \end{bmatrix}.
\end{equation*}

% and the non-zero component of the jacobian of the measurement model as
% \begin{equation}
  % \frac{\partial \hat{\vect{p}}_{b/I}^I}{\partial \vect{p}_{b/I}^I} = I_{3
  % \times 3}.
% \end{equation}

\subsubsection{Global UAV Attitude Measurement}
Similar to the position measurement above, we assume to have some measurement of
the attitude of the UAV with respect to the inertial frame. This measurement may
come from an attitude and heading reference system (AHRS) or from a motion
capture system. 
The measurement model and its estimate can
be written as
\begin{align}
  h_{\text{att}} \left( \x \right) &= \vect{q}_{I}^b \otimes \exp_{\q} \left(
  \vect{\eta}_{\text{att}} \right) \\
    h_{\text{att}} \left( \hat{\x} \right) &= \hat{\vect{q}}_{I}^b.
\end{align}
For a given measurement of attitude, $\vect{z}_{\text{att}}$, the residual is
given by
\begin{equation*}
  \vect{r}_{\text{att}} = \log_{\q} \left(  h_{\text{att}} \left(
  \hat{\x} \right)^{-1} \otimes \vect{z}_{\text{att}} \right)
  % \vect{r}_{\text{att}} = \vect{z}_{\text{att}} - h_{\text{att}} \left( \hat{\x}
\end{equation*}
which is modeled and simplified using~\eqref{eq:quat_true_state}
\begin{align*}
  \vect{r}_{\text{att}} &= \log_{\q} \left(  h_{\text{att}} \left(
  \hat{\x} \right)^{-1} \otimes h_{\text{att}} \left( \x \right) \right) \\
                        &= \log_{\q} \left(  \left(
  \hat{\vect{q}}_{I}^b \right)^{-1} \otimes \vect{q}_{I}^b \otimes \exp_{\q} \left(
  \vect{\eta}_{\text{att}} \right)\right) \\
                        &= \log_{\q} \left(  \left(
                        \hat{\vect{q}}_{I}^b \right)^{-1} \otimes
                        \hat{\vect{q}}_{I}^b \otimes \exp_{\q} \left(
                      \tilde{\vect{\theta}}_I^b \right) \right)
                          + \vect{\eta}_{\text{att}}  \\
                        &= \tilde{\vect{\theta}}_I^b
                          + \vect{\eta}_{\text{att}}. 
\end{align*}
This results in the measurement jacobian
\begin{equation*}
  H_{\text{pos}} =
  \begin{bmatrix}
    \vect{0} & I_{3 \times 3} & \vect{0} & \dots & \vect{0}
  \end{bmatrix}.
\end{equation*}

\subsubsection{Fiducial Translation Measurement}
We assume that a known fiducial marker serves as the desired landing position
for the multirotor UAV on the landing vehicle. The goal frame is therefore
located at the center of the fiducial marker. This means that every detection of the fiducial
marker yields a measurement of the relative translation and rotation from the
camera frame to the goal frame as noted in~\eqref{eq:fiducial_meas}.
The measurement model and its estimate
therefore can be written as
\begin{align*}
  h_{\text{ft}} \left( \x \right) &=
  \vect{p}_{g/c}^c + \vect{\eta}_{\text{ft}} \\
  &= R_b^c \left( R_I^b \vect{p}_{g/b}^v -
  \vect{p}_{c/b}^b \right) + \vect{\eta}_{\text{ft}} \\
  h_{\text{ft}} \left( \hat{\x} \right) &=
    \hat{\vect{p}}_{g/c}^c \\
  &= R_b^c \left( \hat{R}_I^b \hat{\vect{p}}_{g/b}^v -
    \vect{p}_{c/b}^b \right). 
  % \hat{\vect{q}}_{c}^a  &= R_g^a \hat{R}_I^g \hat{R}_b^I R_c^b
\end{align*}
For a given measurement of the relative translation to the fiducial marker,
$\vect{z}_{\text{ft}}$, the residual is given by
\begin{equation*}
  \vect{r}_{\text{ft}} = \vect{z}_{\text{ft}} - h_{\text{ft}} \left( \hat{\x}
  \right),
\end{equation*}
which is modeled as
\begin{align*}
  \vect{r}_{\text{ft}} &=  h_{\text{ft}} \left( \x \right) - h_{\text{ft}} \left( \hat{\x}
  \right), \\
                       &= R_b^c \left( R_I^b \vect{p}_{g/b}^v -
                         \vect{p}_{c/b}^b \right)  +\eta_{\text{rt}} - R_b^c \left( \hat{R}_I^b \hat{\vect{p}}_{g/b}^v -
    \vect{p}_{c/b}^b \right)  \\
                       &= R_b^c R_I^b \vect{p}_{g/b}^v 
                          - R_b^c \hat{R}_I^b \hat{\vect{p}}_{g/b}^v +
                          \eta_{\text{rt}} \\
                       &= R_b^c \left( I - \skewmat{\tilde{\vect{\theta}}_I^b }\right) \hat{R}_I^b \left( \hat{\vect{p}}_{g/b}^v + \tilde{\vect{p}}_{g/b}^v \right) 
                          - R_b^c \hat{R}_I^b \hat{\vect{p}}_{g/b}^v +
                          \eta_{\text{rt}}.
\end{align*}
Expanding and removing second-order terms,
\begin{align*}
  \vect{r}_{\text{ft}} &\approx R_b^c \hat{R}_I^b \tilde{\vect{p}}_{g/b}^v -
  R_b^c \skewmat{\tilde{\vect{\theta}}_I^b } \hat{R}_I^b \hat{\vect{p}}_{g/b}^v 
      + \eta_{\text{rt}} \\
&= R_b^c \hat{R}_I^b \tilde{\vect{p}}_{g/b}^v + R_b^c \skewmat{\hat{R}_I^b \hat{\vect{p}}_{g/b}^v} \tilde{\vect{\theta}}_I^b 
      + \eta_{\text{rt}}.
\end{align*}
This results in the measurement jacobian
\begin{equation*}
  H_{\text{pos}} =
  \begin{bmatrix}
    \vect{0} & R_b^c \skewmat{ \hat{R}_I^b \hat{\vect{p}}_{g/b}^v } & \vect{0} &
    \vect{0} & \vect{0} & R_b^c \hat{R}_I^b & \vect{0} & \dots & \vect{0}
  \end{bmatrix}.
\end{equation*}


% The non-zero components of the jacobian of the measurement model used for the
% Kalman filter update are

% \begin{align}
  % \frac{\partial \hat{\vect{p}}_{g/c}^c}{\partial \vect{q}_I^b} =& R_b^c 
  % \hat{R}_I^b \skewmat{\hat{\vect{p}}_{g/b}^v} \\
    % \frac{\partial \hat{\vect{p}}_{g/c}^c}{\partial \vect{p}_{g/b}^v} =& R_b^c
    % \hat{R}_I^b.
% \end{align}

\subsubsection{Fidual Rotation Measurement}
We use the relative rotation measurement that results from a detection of the
fiducial marker
to create a pseudo measurement of the orientation of the goal frame. This pseudo
measurement is created with
\begin{equation}
  \bar{\theta}_I^g = yaw \left( \bar{R}_c^g R_b^c \hat{R}_I^b \right).
\end{equation}
The measurement model and its estimate can
be written as
\begin{align}
  h_{\text{fr}} \left( \x \right) &= \theta_I^g + \eta_{\text{fr}} \\
  h_{\text{fr}} \left( \hat{\x} \right) &= \hat{\theta}_I^g.
\end{align}
For a given measurement of the rotation of the fiducial marker, $z_{\text{fr}}$,
the residual is given by
\begin{equation}
  r_{\text{fr}} = z_{\text{fr}} - h_{\text{fr}} \left( \hat{\x} \right),
\end{equation}
which is modeled as
\begin{align}
  r_{\text{fr}} &= h_{\text{fr}} \left( \x \right) - h_{\text{fr}} \left( \hat{\x} \right) \\
                &= \theta_I^g + \eta_{\text{fr}} - \hat{\theta}_I^g \\
                &= \tilde{\theta}_I^g + \eta_{\text{fr}}.
\end{align}
This results in the measurement jacobian
\begin{equation*}
  H_{\text{pos}} =
  \begin{bmatrix}
    \vect{0} & \vect{0} & \vect{0} &
    \vect{0} & \vect{0} & \vect{0} & \vect{0} & 1 & \vect{0} & \dots & \vect{0}
  \end{bmatrix}.
\end{equation*}

% and the non-zero jacobians of the measurement model as
% \begin{equation}
  % \frac{\partial \hat{\theta}_I^g}{\partial \theta_I^g} = 1.
% \end{equation}

\subsubsection{Landmark Pixel Measurement}
As described in~\eqref{eq:pixel_meas}, the estimator receives measurements of
the location of each visual landmark in the camera image. We assume that the
pixel measurements received have already been corrected for lens distortion.
Using the pinhole camera model, we therefore express the pixel locations as
\begin{align}
  % \hat{h} &=
  % \begin{bmatrix}
    % \hat{p}_x & \hat{p}_y
  % \end{bmatrix}^\transpose \\
  \begin{bmatrix}
    p_x \\ p_y \\ 1
  \end{bmatrix} &= \frac{1}{\e_3^\transpose \vect{p}_{i/c}^c} K
  \vect{p}_{i/c}^c,
  % \vect{z} &=
  % \begin{bmatrix}
    % f_x \frac{\e_1 \vect{p}_{i/c}^c}{\e_3 \vect{p}_{i/c}^c} + c_x \\
    % f_y \frac{\e_2 \vect{p}_{i/c}^c}{\e_3 \vect{p}_{i/c}^c} + c_y
  % \end{bmatrix},
\end{align}
where K is the camera intrisic matrix and 
\begin{align}
  \vect{p}_{i/c}^c = R_b^c \left( R_I^b \left( R_I^g \right)^\transpose
  \vect{r}_{i/g}^g + R_I^b \vect{p}_{g/b}^v - \vect{p}_{c/b}^b \right).
  \label{eq:p_i_c_c}
\end{align}
The measurement model and its estimate can be written as
\begin{align}
  h_{\text{pix}} \left( \x \right)
  &= \begin{bmatrix} p_x & p_y \end{bmatrix}^\transpose \\
  &= \frac{1}{\e_3^\transpose \vect{p}_{i/c}^c} I_{2 \times 3} K
  \vect{p}_{i/c}^c + \vect{\eta}_{\text{pix}} \\
  h_{\text{pix}} \left( \hat{\x} \right)
  &= \begin{bmatrix} \hat{p}_x & \hat{p}_y \end{bmatrix}^\transpose \\
  &= \frac{1}{\e_3^\transpose \hat{\vect{p}}_{i/c}^c} I_{2 \times 3} K
  \hat{\vect{p}}_{i/c}^c
\end{align}
where
\begin{align}
  \hat{\vect{p}}_{i/c}^c = R_b^c \left( \hat{R}_I^b \left( \hat{R}_I^g \right)^\transpose
  \hat{\vect{r}}_{i/g}^g + \hat{R}_I^b \hat{\vect{p}}_{g/b}^v - \vect{p}_{c/b}^b
\right).
\end{align}
For a given measurement of the pixel location of a landmark,
$\vect{z}_{\text{pix}}$, the residual is given by
\begin{equation}
  \vect{r}_{\text{pix}} = \vect{z}_{\text{pix}} - h_{\text{pix}} \left( \hat{\x}
    \right),
\end{equation}
which is modeled as
\begin{align}
  \vect{r}_{\text{pix}} &= h_{\text{pix}} \left( \x \right) - h_{\text{pix}} \left( \hat{\x}
    \right) \\
  &= \frac{1}{\e_3^\transpose \vect{p}_{i/c}^c} I_{2 \times 3} K
  \vect{p}_{i/c}^c + \vect{\eta}_{\text{pix}} - \frac{1}{\e_3^\transpose \hat{\vect{p}}_{i/c}^c} I_{2 \times 3} K
  \hat{\vect{p}}_{i/c}^c.
\end{align}
To compute the measurement jacobian, we first expand~\eqref{eq:p_i_c_c}
\MF{This measurement jac is very crazy and the derivation may not be 100~\% correct}
\begin{align}
  \vect{p}_{i/c}^c &= R_b^c \left( \left( I - \skewmat{\tilde{\vect{\theta}}_I^b
    } \right) \hat{R}_I^b \left( I + \skewmat{\tilde{\vect{\theta}}_I^g }
  \right) \left( \hat{R}_I^g \right)^\transpose
\left( \hat{\vect{r}}_{i/g}^g  + \tilde{\vect{r}}_{i/g}^g \right) + \left( I - \skewmat{\tilde{\vect{\theta}}_I^b
} \right) \hat{R}_I^b \left( \hat{\vect{p}}_{g/b}^v + \tilde{\vect{p}}_{g/b}^v \right) - \vect{p}_{c/b}^b \right).
\end{align}
By simplifying and ignoring secord-order terms,
\begin{align*}
  \vect{p}_{i/c}^c &= R_b^c \left( \hat{R}_I^b \left( \hat{R}_I^g \right)^\transpose
  \hat{\vect{r}}_{i/g}^g + \hat{R}_I^b \hat{\vect{p}}_{g/b}^v - \vect{p}_{c/b}^b
  - \skewmat{\tilde{\vect{\theta}}_I^b} \hat{R}_I^b \left( \hat{R}_I^g
    \right)^\transpose \hat{\vect{r}}_{i/g}^g + \hat{R}_I^b
    \skewmat{\tilde{\vect{\theta}}_I^g} \left( \hat{R}_I^g
  \right)^\transpose \hat{\vect{r}}_{i/g}^g +
\hat{R}_I^b \left( \hat{R}_I^g
    \right)^\transpose \tilde{\vect{r}}_{i/g}^g \right. \\
                   &- \left. \skewmat{\tilde{\vect{\theta}}_I^b} \hat{R}_I^b \hat{\vect{p}}_{g/b}^v
    + \hat{R}_I^b \tilde{\vect{p}}_{g/b}^v
   \right). \\
    &= \hat{\vect{p}}_{i/c}^c
    + R_b^c \left( 
  - \skewmat{\tilde{\vect{\theta}}_I^b} \hat{R}_I^b \left( \hat{R}_I^g
    \right)^\transpose \hat{\vect{r}}_{i/g}^g + \hat{R}_I^b
    \skewmat{\tilde{\vect{\theta}}_I^g} \left( \hat{R}_I^g
  \right)^\transpose \hat{\vect{r}}_{i/g}^g +
\hat{R}_I^b \left( \hat{R}_I^g
    \right)^\transpose \tilde{\vect{r}}_{i/g}^g
                   - \skewmat{\tilde{\vect{\theta}}_I^b} \hat{R}_I^b \hat{\vect{p}}_{g/b}^v
    + \hat{R}_I^b \tilde{\vect{p}}_{g/b}^v
   \right), 
\end{align*}
which we define as
\begin{equation*}
  \vect{p}_{i/c}^c = \hat{\vect{p}}_{i/c}^c + \tilde{\vect{p}}_{i/c}^c .
\end{equation*}


% where $f_x$, $f_y$, $c_x$, and $c_y$ are constant parameters of the camera. We
% can then express the position of the landmark $i$ with respect to the camera in the
% camera frame as
% \begin{align}
  % \vect{p}_{i/c}^c &= R_b^c R_v^b \left(\vect{p}_{i/v}^v -
    % \vect{p}_{c/v}^v\right) \\
    % \vect{p}_{i/c}^c &= R_b^c \left(R_v^b \vect{p}_{i/v}^v -
    % \vect{p}_{c/b}^b\right)
% \end{align}
% This is actually a little bit weird because of the state parameters, but
% \begin{equation}
  % \vect{p}_{i/v}^v =
  % \begin{bmatrix}
    % \e_1^\transpose \left( R_g^v \vect{p}_{i/g}^g + \vect{p}_{g/v}^v \right) \\
    % \e_2^\transpose \left( R_g^v \vect{p}_{i/g}^g + \vect{p}_{g/v}^v \right) \\
    % \e_3^\transpose \left( \vect{p}_{i/g}^g - \vect{p}_{b/I}^I \right)
  % \end{bmatrix}.
% \end{equation}

% Let us derive the measurement jacobians.
% \begin{align}
  % \frac{\partial}{\partial \x} \hat{h} &= \frac{\e_3^\transpose \hat{\vect{p}}_{i/c}^c
  % \frac{\partial}{\partial \x} K \hat{\vect{p}}_{i/c}^c - K \hat{\vect{p}}_{i/c}^c
  % \frac{\partial}{\partial \x} \e_3^\transpose \hat{\vect{p}}_{i/c}^c}{\left(
% \e_3^\transpose \hat{\vect{p}}_{i/c}^c \right)^2 } \\
  % \frac{\partial}{\partial \x} \hat{h} &= \frac{\e_3^\transpose \hat{\vect{p}}_{i/c}^c
  % K \frac{\partial}{\partial \x} \hat{\vect{p}}_{i/c}^c - K \hat{\vect{p}}_{i/c}^c
  % \e_3^\transpose \frac{\partial}{\partial \x} \hat{\vect{p}}_{i/c}^c}{\left(
% \e_3^\transpose \hat{\vect{p}}_{i/c}^c \right)^2 }
% \end{align}
% \begin{align}
  % \frac{\partial}{\partial \vect{p}_{g/b}^v} \hat{\vect{p}}_{i/c}^c &=
  % R_b^c \hat{R}_I^b \\
  % \frac{\partial}{\partial \vect{q}_{I}^b} \hat{\vect{p}}_{i/c}^c &=
  % -R_b^c \hat{R}_I^b \skewmat{ \left( R_I^g \right)^\transpose
  % \hat{\vect{r}}_{i/g}^g + \hat{\vect{p}}_{g/b}^v } \\
  % \frac{\partial}{\partial \theta_I^g} \hat{\vect{p}}_{i/c}^c &=
  % R_b^c \hat{R}_I^b \\
% \end{align}

% % \subsubsection{Measurement Jacobian}
% For the purpose of deriving the jacobians, we will just look at the jacobians
% with respect to $p_x$, the pixel location along the $x$ axis of the camera
% frame. From above we have that
% \begin{align}
  % p_x &= f_x \frac{\e_1^\transpose \vect{p}_{i/c}^c}{\e_3^\transpose \vect{p}_{i/c}^c} + c_x \\ 
  % p_x &= f_x \frac{\e_1^\transpose R_b^c \left(R_v^b \vect{p}_{i/v}^v -
      % \vect{p}_{c/b}^b\right) }{\e_3^\transpose R_b^c \left(R_v^b \vect{p}_{i/v}^v -
  % \vect{p}_{c/b}^b\right) } + c_x 
% \end{align}
% Note that all values are constants except for $\vect{p}_{i/v}^v$ and $R_v^b$.
% The individual parts of the jacobian are given here:
% \begin{equation}
  % \frac{\partial p_x}{\partial \vect{p}_{g/v}^v} =
  % \frac{f_x \e_1 R_b^c R_v^b}
    % {\left(\e_3 R_b^c \left(R_v^b \vect{p}_{i/v}^v -
    % \vect{p}_{c/b}^b\right)\right)}
    % - \frac{\left(\e_3 R_b^c R_v^b \right) f_x \left(\e_1 R_b^c \left(R_v^b \vect{p}_{i/v}^v -
        % \vect{p}_{c/b}^b\right)\right)} {\left(\e_3 R_b^c \left(R_v^b \vect{p}_{i/v}^v -
  % \vect{p}_{c/b}^b\right)\right)^2}
% \end{equation}
% where only the first two (of three) entries of the vector are used. The third
% entry is used for the jacobian w.r.t. $\vect{p}_{b/I}^I(2)$ as
% \begin{equation}
  % \frac{\partial p_x}{\partial \vect{p}_{b/I}^I(2)} = -\frac{\partial p_x}{\partial
  % \vect{p}_{g/v}^v}\left(2\right).
% \end{equation}
% The derivatives for the attitude representation are less straight forward. From
% "Micro Lie Theory" we use the fact that the jacobian of the rotation action can
% be shown to be
% \begin{equation}
  % J_R^{R \cdot v} = -R \skewmat{v}.
% \end{equation}
% \begin{equation}
  % \frac{\partial p_x}{\partial \q} =
  % \frac{-f_x \e_1 R_b^c R_v^b \skewmat{\vect{p}_{i/v}^v}}
    % {\left(\e_3 R_b^c \left(R_v^b \vect{p}_{i/v}^v -
    % \vect{p}_{c/b}^b\right)\right)}
    % - \frac{\left(-\e_3 R_b^c R_v^b \skewmat{\vect{p}_{i/v}^v} \right) f_x \left(\e_1 R_b^c \left(R_v^b \vect{p}_{i/v}^v -
        % \vect{p}_{c/b}^b\right)\right)} {\left(\e_3 R_b^c \left(R_v^b \vect{p}_{i/v}^v -
  % \vect{p}_{c/b}^b\right)\right)^2}
% \end{equation}
% where the jacobians for the other angles are similar, just substituting in.
% The derivative for the goal angle, $\theta_g$ is given by using
% \begin{equation}
  % \frac{\partial \vect{p}_{i/v}^v}{\partial \theta_g} =
  % \begin{bmatrix}
    % \e_1^\transpose \frac{\partial R_g^v}{\partial \theta_g} \vect{p}_{i/g}^g \\
    % \e_2^\transpose \frac{\partial R_g^v}{\partial \theta_g} \vect{p}_{i/g}^g \\
    % 0
  % \end{bmatrix}
% \end{equation}
% in the jacobian given by
% \begin{equation}
  % \frac{\partial p_x}{\partial \theta_g} =
  % \frac{f_x \e_1 R_b^c R_v^b \frac{\partial \vect{p}_{i/v}^v}{\partial \theta_g}}
    % {\left(\e_3 R_b^c \left(R_v^b \vect{p}_{i/v}^v -
    % \vect{p}_{c/b}^b\right)\right)}
    % - \frac{\left(\e_3 R_b^c R_v^b \frac{\partial \vect{p}_{i/v}^v}{\partial \theta_g} \right) f_x \left(\e_1 R_b^c \left(R_v^b \vect{p}_{i/v}^v -
        % \vect{p}_{c/b}^b\right)\right)} {\left(\e_3 R_b^c \left(R_v^b \vect{p}_{i/v}^v -
  % \vect{p}_{c/b}^b\right)\right)^2}
% \end{equation}
% Similarly, the derivative for the landmark offset, $\vect{r}_i$ or
% $\vect{p}_{i/g}^g$ is given by using
% \begin{equation}
  % \frac{\partial \vect{p}_{i/v}^v}{\partial \vect{p}_{i/g}^g} =
  % \begin{bmatrix}
    % \e_1^\transpose R_g^v \\
    % \e_2^\transpose R_g^v \\
    % \e_3^\transpose
  % \end{bmatrix}
% \end{equation}
% in the jacobian given by
% \begin{equation}
  % \frac{\partial p_x}{\partial \vect{p}_{i/g}^g} =
  % \frac{f_x \e_1 R_b^c R_v^b \frac{\partial \vect{p}_{i/v}^v}{\partial \vect{p}_{i/g}^g}}
    % {\left(\e_3 R_b^c \left(R_v^b \vect{p}_{i/v}^v -
    % \vect{p}_{c/b}^b\right)\right)}
    % - \frac{\left(\e_3 R_b^c R_v^b \frac{\partial \vect{p}_{i/v}^v}{\partial \vect{p}_{i/g}^g} \right) f_x \left(\e_1 R_b^c \left(R_v^b \vect{p}_{i/v}^v -
        % \vect{p}_{c/b}^b\right)\right)} {\left(\e_3 R_b^c \left(R_v^b \vect{p}_{i/v}^v -
  % \vect{p}_{c/b}^b\right)\right)^2}
% \end{equation}
% The jacobians for the y pixel measurement, $p_y$ are similar to the ones derived
% above, just changing $f_y$ in place of $f_x$ and $\e_2$ instead of $\e_1$.


% !TEX root=./root.tex

\subsection{Goal and Landmark Initialization}
The goal states are initialized upon receiving the first measurement from
the detection fiducial landing marker. We assume that we can measure the
relative translation and rotation from the camera image to the fiducial marker
at each detection. This measurement,
\begin{equation}
  \bar{\vect{z}} =
  \begin{bmatrix}
    \bar{\vect{p}}_{g/c}^c & \bar{\vect{q}}_c^g
  \end{bmatrix}
\end{equation}
is used to initialize the goal states with
\begin{align}
  \hat{\vect{p}}_{g/b}^v &= \left( \hat{R}_I^b \right)^\transpose \left( \left ( R_b^c
  \right)^\transpose \bar{\vect{p}}_{g/c}^c + \vect{p}_{c/b}^b
\right)   \\
      \hat{\theta_I^g} &= yaw \left( \bar{R}_c^g R_b^c \hat{R}_I^b \right)
  % quat::Quatd q_I2g_meas = x().q * q_b2c_ * z.q_c2a * q_a2g;
  % const double yaw_meas = q_I2g_meas.euler()(2);
\end{align}
where $R_b^c$ and $\vect{p}_{c/b}^b$ are assumed to be known constants and
$yaw()$ is a function that extracts the yaw Euler angle from a rotation matrix.
The linear and angular velocity of the landing vehicle are initialized to values
of zero with large enough covariances for the specific use case.

Similar to the manner in which the goal states are intialized, each time a new
landmark is added to the estimated vector, we initialize its state based on the
first measurement received. 
Each time a new landmark is acquired and added to the estimated vector, we
should initialize its location to something intelligent. Theoretically, we could
just initialize the landmark's location to zero with a large enough associated
covariance. However, by initializing the landmark to the location based on its
initial pixel measurement, we can initialize the landmark with a smaller
covariance.

When we first acquire a landmark, the only information we have about it is a
measurement of its pixel location in the image frame. This measurement, denoted
by
\begin{equation}
  h \left( \x \right) = 
  \begin{bmatrix}
    p_x & p_y
  \end{bmatrix}^\transpose
\end{equation}
can be used to deduce the landmark's relative position vector that is estimated.
The easiest way to see this is by first creating a virtual image plane. This
virtual image plane projects the landmark pixel location into the image plane as
if the camera were perfectly aligned with the inertial, vehicle frame. We can
derive this using the pin hole camera model where
\begin{align}
  \begin{bmatrix}
    p_x \\ p_y \\ 1
  \end{bmatrix} &= K
  \begin{bmatrix}
    X^c / Z^c \\
    Y^c / Z^c \\
    1
  \end{bmatrix} \\
  \begin{bmatrix}
    X^c / Z^c \\
    Y^c / Z^c \\
    1
  \end{bmatrix}
   &= K^{-1}
  \begin{bmatrix}
    p_x \\ p_y \\ 1
  \end{bmatrix} \\
  \begin{bmatrix}
    X^c / Z^c \\
    Y^c / Z^c \\
    1
  \end{bmatrix}^v
   &= R_b^v R_c^b K^{-1}
  \begin{bmatrix}
    p_x \\ p_y \\ 1
  \end{bmatrix} \\
\end{align}
The resulting vector, $
  \begin{bmatrix}
    X^c / Z^c \\
    Y^c / Z^c \\
    1
  \end{bmatrix}^v$
  is the vector $\vect{p}_{i/c}^v$ up to a scale factor. The vector can be scaled by assuming that the altitude of the landmark is equal to the
altitude of the goal. To get the expected altitude, we solve for
\begin{align}
  \e_3^\transpose \vect{p}_{g/c}^v &= \e_3^\transpose \vect{p}_{g/b}^v - \e_3^\transpose \vect{p}_{c/b}^v \\
  \e_3^\transpose \vect{p}_{g/c}^v &= \e_3^\transpose \vect{p}_{g/b}^v -
  \e_3^\transpose R_b^I \vect{p}_{c/b}^b \\
  \e_3^\transpose \vect{p}_{g/c}^v &= \frac{1}{\rho_g} -
  \e_3^\transpose R_b^I \vect{p}_{c/b}^b \\
\end{align}

This gives us the vector, $\vect{p}_{i/c}^v$.
With this vector, we can then reach the estimated state vector,
$\vect{p}_{i/g}^g$ with the following
\begin{align}
  \vect{p}_{i/v}^v &= \vect{p}_{i/c}^v + R_b^I \vect{p}_{c/b}^b \\
  \vect{p}_{i/g}^g &= R_v^g \left( \vect{p}_{i/v}^v - \vect{p}_{g/v}^v \right).
\end{align}



\section{Simulation} \label{sec:est_paper_simulation}
% !TEX root=../root.tex

In addition to the hardware results presented in the following section, we
present simulation results to show the effectiveness of the proposed estimation
algorithm.
A multirotor UAV is simulated using the dynamics presented
in~\eqref{eq:uav_dynamics}.
Additionally, a landing vehicle is simulated using the dynamics presented
in~\eqref{eq:goal_dynamics}. The standard deviations used for the zero-mean
Guassian noise processes in the dynamics are found
in~\tabref{tab:sim_process_noises}.
As the estimator depends on
tracking and estimating the positions of visual landmarks that are rigidly
attached to the landing vehicle, landmark positions on the landing vehicle
are also simulated.
Positions for each landmark in the goal frame are
randomly sampled from the
uniform distribution
\begin{equation}
  \vect{r}_{i/g}^g = \mathcal{U}
  \left( \begin{bmatrix} -2 \\ -2 \\ -1 \end{bmatrix},
  \begin{bmatrix} 2 \\ 2 \\ 1 \end{bmatrix} \right)
  \label{eq:uniform_dist}
\end{equation}
At each time step of the simulation, landmarks are given a one percent chance
of disappearing. When a landmark disappears, a new landmark is randomly
generated by sampling from~\eqref{eq:uniform_dist}.
A specified probability determines

The sensor measurements that are provided to the estimator are simulated using the true
state of the simulation. These sensor measurements include accelerometer,
gyroscope, global position of the UAV, global attitude 
of the UAV, relative position and attitude from the fiducial landing marker, and
landmark locations in the camera frame. All measurements are
corrupted with white Gaussian noise. The sensor noise parameters along with the
rate at which each sensor is sampled are found
in~\tabref{tab:sim_meas_noise}.

The landing vehicle was initialized with initial conditions of $v_x = 0.5
m/s$, $v_y = 0 m/s$, $\theta_I^g = 1.5 rad$ and $\omega = 0.5 rad/s$. The
simulated camera was oriented at a yaw angle of $\pi/2$ with respect to the body
frame and $p_{c/b}^b = \begin{bmatrix}0.25 & -0.2 &
0.4\end{bmatrix}^\transpose$.

The results for simulations where the estimator uses ten visual landmarks and
zero visual landmarks are seen in~\ref{fig:BLAH} and~\ref{fig:BLAH}
respectively. We do not include the plots of the UAV states,
$\vect{x}_{\text{UAV}}$, as it is well known that these states are easily
estimated with the given measurements. In the experiments shown in the plots,
the measurements from the fiducial landing marker are not used after $t = 5 s$
in the simulation to demonstrate the performance of the proposed estimation
algorithm. The figures clearly show that when ten visual landmarks are used by
the estimator, the estimates of the goal states remain accurate and consistent
throughout the duration of the simulation. However, when no visual landmarks are
used by the estimator, the estimate of the goal states quickly become incorrect.
We note that the $z$ direction of the goal position is the exception to this
because the landing vehicle is constrained to move only in the $xy$ plane.

To further demonstrate this performance, 100 simulation runs were performed
both using ten visual landmarks and using zero visual landmarks. The L2 norm of
the errors in the $x$ and $y$ direction of the goal position state are plotted
with respect to time in~\ref{fig:BLAH}. While the error when using ten landmarks
almost entirely remains under 1 m for all 100 simulations, the error when using
no landmarks quickly grows larger that 1 m. The error for several of the
simulations reaches over 10 m after 30s.

These simulation results clearly demonstrate the ability to maintain accurate
and consistent estimates of the goal states when the fiducial landing marker is
not detected for significant periods of time.


\begin{table}[h!]
  \begin{center}
    \caption{Simulation Motion Model Parameters.}
    \label{tab:sim_process_noises}
    \begin{tabular}{l|l}
      \textbf{Parameter} & \textbf{Std. Deviation} \\
      \hline
      $\vect{\eta}_{\beta_a}$ & 0.05 $m/s^2$ \\
      $\vect{\eta}_{\beta_\omega}$ & 0.01 $rad/s$ \\
      $\vect{\eta}_{gv}$ & 5 m/s \\
      $\vect{\eta}_{g\omega}$ & 5 rad/s \\
      % v & 5 m/s \\
      % landmark $x_{\text{min}}$ & -2 m \\
      % landmark $x_{\text{max}}$ & 2 m \\
      % landmark $y_{\text{min}}$ & -2 m \\
      % landmark $y_{\text{max}}$ & 2 m \\
      % landmark $z_{\text{min}}$ & -1 m \\
      % landmark $z_{\text{max}}$ & 1 m \\
      % landmark disappear prob. & 1\% \\
    \end{tabular}
  \end{center}
\end{table}

\begin{table}[h!]
  \begin{center}
    \caption{Simulation Sensor Noise Characteristics.}
    \label{tab:sim_meas_noise}
    \begin{tabular}{l|l|l}
      \textbf{Measurement} & \textbf{Std Dev} & \textbf{Rate} \\
      \hline
      Accelerometer & 0.2 $m/s^2$ & 250 Hz \\
      % walk & 0.05 $m/s^2$  \\
      % init & 0.01 $m/s^2$ \\
      Gyroscope & 0.1 $rad/s$ & 250 Hz \\
      % walk & 0.01 $rad/s$  \\
      % init & 0.01 $rad/s$ \\
      UAV global position & 0.01 $m$ & 10 Hz \\
      UAV global attitude & 0.001 $rad$ & 10 Hz \\
      Fiducial marker position & 0.1 $m$ & 30 Hz \\
      Fiducial marker attitude & 0.1 $rad$ & 30 Hz \\
      Landmark image point & 2.0 pixels & 30 Hz \\
    \end{tabular}
  \end{center}
\end{table}

\begin{figure}
  \centering
  \includegraphics[scale=0.5]{plots/with_lms_gp.png}
  \caption{Simulation results with estimator using a maximum of ten visual
  landmarks. Measurements from the fiducial marker are not used after $t$ = 5
$s$ to demonstrate the performance of the estimator.}
  \label{fig:with_lms_gp}
\end{figure}

\begin{figure}
  \centering
  \includegraphics[scale=0.5]{plots/with_lms_gv.png}
  \caption{Simulation results with estimator using a maximum of ten visual
  landmarks. Measurements from the fiducial marker are not used after $t$ = 5
$s$ to demonstrate the performance of the estimator.}
  \label{fig:with_lms_gv}
\end{figure}

\begin{figure}
  \centering
  \includegraphics[scale=0.5]{plots/with_lms_gatt.png}
  \caption{Simulation results with estimator using a maximum of ten visual
  landmarks. Measurements from the fiducial marker are not used after $t$ = 5
$s$ to demonstrate the performance of the estimator.}
  \label{fig:with_lms_gatt}
\end{figure}

\begin{figure}
  \centering
  \includegraphics[scale=0.5]{plots/no_lms_gp.png}
  \caption{Simulation results with estimator using no visual
  landmarks. Measurements from the fiducial marker are not used after $t$ = 5
$s$ to demonstrate the performance of the estimator.}
  \label{fig:no_lms_gp}
\end{figure}

\begin{figure}
  \centering
  \includegraphics[scale=0.5]{plots/no_lms_gv.png}
  \caption{Simulation results with estimator using no visual
  landmarks. Measurements from the fiducial marker are not used after $t$ = 5
$s$ to demonstrate the performance of the estimator.}
  \label{fig:no_lms_gv}
\end{figure}

\begin{figure}
  \centering
  \includegraphics[scale=0.5]{plots/no_lms_gatt.png}
  \caption{Simulation results with estimator using no visual
  landmarks. Measurements from the fiducial marker are not used after $t$ = 5
$s$ to demonstrate the performance of the estimator.}
  \label{fig:no_lms_gatt}
\end{figure}

\begin{figure}
  \centering
  \includegraphics[scale=0.5]{plots/mc_with_lms_xy_err.png}
  \caption{Simulation results with estimator using a maximum of ten visual
  landmarks. Measurements from the fiducial marker are not used after $t$ = 5
$s$ to demonstrate the performance of the estimator. The L2 norm of the error in
the x and y directions of the goal position is seen for 100 different simulation
runs.}
  \label{fig:mc_with_lms_xy_err}
\end{figure}

\begin{figure}
  \centering
  \includegraphics[scale=0.5]{plots/mc_no_lms_xy_err.png}
  \caption{Simulation results with estimator using no visual
  landmarks. Measurements from the fiducial marker are not used after $t$ = 5
$s$ to demonstrate the performance of the estimator. The L2 norm of the error in
the x and y directions of the goal position is seen for 100 different simulation
runs.}
  \label{fig:mc_no_lms_xy_err}
\end{figure}


\section{Hardware} \label{sec:est_paper_hardware}
% !TEX root=../root.tex

\subsection{Platform}
The proposed estimation and control schemes were implemented and flown in
hardware to further demonstrate the performance observed in the simulation
results. The multirotor used for experiments was built on a DJI 450 Flamewheel
frame. All computation was done onboard the UAV on an NVIDIA Jetson TX2 using
ROS. An ELP
USB Camera with a 2.1mm Lens was mounted to the bottom of the UAV such that the
camera faced downward in the UAV's bodyframe. Controller commands were sent from
the onboard computer to the CC3D Revolution 32bit F4 flight controller running
the ROSflight firmware using the ROSflight ROS node~\cite{jackson2016rosflight}.


% !TEX root=../root.tex

\subsection{Fiducial Marker}
We use an ArUco marker~\cite{romero2018speeded} as a fiducial landing marker
in our experiements. As pictured in the top, right corner of~\figref{fig:features_with_aruco}, an ArUco
marker is a square fiducial marker. When detected, a relative translation and
rotation from the camera to the marker can be estimated using the positions of
the corners of the marker in the camera image and the known size of the marker. 
The ArUco library only detects the marker when the entire marker is contained
unobstructed in the camera image. Due to these
conditions, it is not uncommon that the marker may be undetected for significant
periods of time during a landing maneuver.


% !TEX root=../root.tex

\subsection{Feature Tracking}
As mentioned in~\secref{sec:estimation}, in addition to using measurements from the fiducial landing
marker, the estimation algorithm uses measurements of visual features that are
rigidly attached to the landing vehicle. It is important to note that the
problem of determining which visual features in a camera image are attached to
the landing vehicle is not trivial. We leave this problem as future work and
circumvent this problem by flying low enough to the landing vehicle that the
landing vehicle takes up the entire field of view of the camera.

Visual features are first detected using a FAST feature
detector~\cite{rosten2006machine}. Features are then tracked from one frame to
the next using optical flow~\cite{bouguet2001pyramidal}. Periodically, we remove
features that have been tracked poorly by estimating the Essential matrix between
the current camera image and a stored keyframe image. Outliers to the found
Essential matrix are thrown out, and new FAST features are detected to replace
them. For our experiements, the feature tracker attempts to maintain 250
tracked features at all times. As the proposed estimator only uses measurements from a few
tracked features, a subset of the features which have been tracked the longest
are provided to the estimator. 

Each visual feature that is acquired and tracked is assigned a unique integer
identitication number. The estimator uses these identification numbers to
determine when a visual feature is no longer tracked and, therefore, should be removed from
the estimated state.
An example camera image from the UAV showing the
subset of tracked features provided to the estimator and their identificantion
numbers can be seen in~\figref{fig:features_with_aruco}.


\begin{figure}
  \centering
  \includegraphics[scale=0.5]{imgs/features_with_aruco.png}
  \caption{A processed camera image from the multirotor UAV's camera. The ArUco
  landing marker is pictured in the top, right corner of the image. Each green
circle shows the tracked location of a visual landmark being used by the
estimator. The red number associated with each visual landmark is the unique
integer ID assigned by the feature tracker.}
  \label{fig:features_with_aruco}
\end{figure}

\input{hardware/mocap}
% !TEX root=../root.tex

\subsection{Landing Vehicle}
As flight tests were conducted in a small indoor environement, the landing
vehicle was scaled down in an attempt to better extend to outdoor scenarios. A
landing vehicle was assembled as pictured in~\figref{fig:landing_vehicle}.
During the experiments, the landing vehicle was manually driven around the
motion capture room, following the perimeter of the room and turning at each
wall.

\begin{figure}
  \centering
  \includegraphics[scale=0.5]{imgs/landing_vehicle.png}
  \caption{Multirotor UAV shown in autonomous flight, tracking the landing
  vehicle.}
  \label{fig:landing_vehicle}
\end{figure}

% !TEX root=../root.tex

\subsection{Experiment Results}
During the flight experiments, the multirotor UAV was manually flown until the
fiducial landing marker was detected. Upon detection, the closed loop estimation
and control was given full control of the UAV. To demonstrate the ability of the
system to maintain good tracking of the landing vehicle even when the fiducial
marker is not detected for long periods of time, the fiducial marker detection
was turned off a few seconds after detection.

The results are shown in Figures~\ref{fig:hardware_gp}, \ref{fig:hardware_gv},
\ref{fig:hardware_gatt}. The fiducial landing marker is first detected at
$t$~=~20~$s$, when the plots begin. After 10 seconds, at $t$~=~30~$s$, the
fiducial marker detection is turned off for the remainder of the experiment.
Even though no measurements from the fiducial marker are used, it is clear that
the estimates of the position, velocitity, attitude and angular velocity of the
landing vehicle remain accurate and consistent for the duration of the
experiment. These accurate estimates allow
the UAV to continue to control relative to the landing vehicle, tracking closely
above the goal frame as the landing vehicle moves around the room. After the
landing vehicle has completed two full laps around the room,
at approximately $t$~=102~$s$, manual control of the UAV is resumed, ending the experiment.

A video of the flight experiment can be found at \MF{TODO make and add video}.

\begin{figure}
  \centering
  \includegraphics[scale=0.5]{plots/hardware_gp.png}
  \caption{Hardware results with the estimator using 10 visual
  features. The blue line represents the true state while the orange line
  represents the estimated state. The two grey lines show a 2 $\sigma$ bound for
  the estimate based on the estimated covariance. Measurements from the fiducial
  marker are not used after $t$~=~30~$s$ to demonstrate the performance of the estimator.}
  \label{fig:hardware_gp}
\end{figure}

\begin{figure}
  \centering
  \includegraphics[scale=0.5]{plots/hardware_gv.png}
  \caption{Hardware results with the estimator using 10 visual
  features. The blue line represents the true state while the orange line
  represents the estimated state. The two grey lines show a 2 $\sigma$ bound for
  the estimate based on the estimated covariance. Measurements from the fiducial
  marker are not used after $t$~=~30~$s$ to demonstrate the performance of the estimator.}
  \label{fig:hardware_gv}
\end{figure}

\begin{figure}
  \centering
  \includegraphics[scale=0.5]{plots/hardware_gatt.png}
  \caption{Hardware results with the estimator using 10 visual
  features. The blue line represents the true state while the orange line
  represents the estimated state. The two grey lines show a 2 $\sigma$ bound for
  the estimate based on the estimated covariance. Measurements from the fiducial
  marker are not used after $t$~=~30~$s$ to demonstrate the performance of the estimator.}
  \label{fig:hardware_gatt}
\end{figure}


\section{Conclusion} \label{sec:est_paper_conclusion}
The proposed ESKF provides a method for maintaining accurate and consistent
estimates of the state of a landing target vehicle when a fiducial landing marker is
not detected for significant periods of time. This improvement is achieved by
tracking and estimating the locations of unknown visual features on the target
vehicle.
The presented experiments demonstrated that the addition of just ten visual
feature positions to the estimated state
is sufficient to allow for reliable operation with respect to a target vehicle
for more than one
minute without detection of a fiducial landing marker.


% An example of a floating figure using the graphicx package.
% Note that \label must occur AFTER (or within) \caption.
% For figures, \caption should occur after the \includegraphics.
% Note that IEEEtran v1.7 and later has special internal code that
% is designed to preserve the operation of \label within \caption
% even when the captionsoff option is in effect. However, because
% of issues like this, it may be the safest practice to put all your
% \label just after \caption rather than within \caption{}.
%
% Reminder: the "draftcls" or "draftclsnofoot", not "draft", class
% option should be used if it is desired that the figures are to be
% displayed while in draft mode.
%
%\begin{figure}[!t]
%\centering
%\includegraphics[width=2.5in]{myfigure}
% where an .eps filename suffix will be assumed under latex, 
% and a .pdf suffix will be assumed for pdflatex; or what has been declared
% via \DeclareGraphicsExtensions.
%\caption{Simulation results for the network.}
%\label{fig_sim}
%\end{figure}

% Note that the IEEE typically puts floats only at the top, even when this
% results in a large percentage of a column being occupied by floats.


% An example of a double column floating figure using two subfigures.
% (The subfig.sty package must be loaded for this to work.)
% The subfigure \label commands are set within each subfloat command,
% and the \label for the overall figure must come after \caption.
% \hfil is used as a separator to get equal spacing.
% Watch out that the combined width of all the subfigures on a 
% line do not exceed the text width or a line break will occur.
%
%\begin{figure*}[!t]
%\centering
%\subfloat[Case I]{\includegraphics[width=2.5in]{box}%
%\label{fig_first_case}}
%\hfil
%\subfloat[Case II]{\includegraphics[width=2.5in]{box}%
%\label{fig_second_case}}
%\caption{Simulation results for the network.}
%\label{fig_sim}
%\end{figure*}
%
% Note that often IEEE papers with subfigures do not employ subfigure
% captions (using the optional argument to \subfloat[]), but instead will
% reference/describe all of them (a), (b), etc., within the main caption.
% Be aware that for subfig.sty to generate the (a), (b), etc., subfigure
% labels, the optional argument to \subfloat must be present. If a
% subcaption is not desired, just leave its contents blank,
% e.g., \subfloat[].


% An example of a floating table. Note that, for IEEE style tables, the
% \caption command should come BEFORE the table and, given that table
% captions serve much like titles, are usually capitalized except for words
% such as a, an, and, as, at, but, by, for, in, nor, of, on, or, the, to
% and up, which are usually not capitalized unless they are the first or
% last word of the caption. Table text will default to \footnotesize as
% the IEEE normally uses this smaller font for tables.
% The \label must come after \caption as always.
%
%\begin{table}[!t]
%% increase table row spacing, adjust to taste
%\renewcommand{\arraystretch}{1.3}
% if using array.sty, it might be a good idea to tweak the value of
% \extrarowheight as needed to properly center the text within the cells
%\caption{An Example of a Table}
%\label{table_example}
%\centering
%% Some packages, such as MDW tools, offer better commands for making tables
%% than the plain LaTeX2e tabular which is used here.
%\begin{tabular}{|c||c|}
%\hline
%One & Two\\
%\hline
%Three & Four\\
%\hline
%\end{tabular}
%\end{table}


% Note that the IEEE does not put floats in the very first column
% - or typically anywhere on the first page for that matter. Also,
% in-text middle ("here") positioning is typically not used, but it
% is allowed and encouraged for Computer Society conferences (but
% not Computer Society journals). Most IEEE journals/conferences use
% top floats exclusively. 
% Note that, LaTeX2e, unlike IEEE journals/conferences, places
% footnotes above bottom floats. This can be corrected via the
% \fnbelowfloat command of the stfloats package.








% if have a single appendix:
%\appendix[Proof of the Zonklar Equations]
% or
%\appendix  % for no appendix heading
% do not use \section anymore after \appendix, only \section*
% is possibly needed

% use appendices with more than one appendix
% then use \section to start each appendix
% you must declare a \section before using any
% \subsection or using \label (\appendices by itself
% starts a section numbered zero.)
%

% \appendices
% \include{appendix/est_paper_err_state_dynamics}
% \include{appendix/est_paper_visual_feature_pixel_measurement}

% \appendices
% \section{Proof of the First Zonklar Equation}
% Appendix one text goes here.

% you can choose not to have a title for an appendix
% if you want by leaving the argument blank
% \section{}
% Appendix two text goes here.


% use section* for acknowledgment
% \section*{Acknowledgment}


% The authors would like to thank...


% Can use something like this to put references on a page
% by themselves when using endfloat and the captionsoff option.
\ifCLASSOPTIONcaptionsoff
  \newpage
\fi



\bibliographystyle{IEEEtran}
\bibliography{abbrev,library}

% trigger a \newpage just before the given reference
% number - used to balance the columns on the last page
% adjust value as needed - may need to be readjusted if
% the document is modified later
%\IEEEtriggeratref{8}
% The "triggered" command can be changed if desired:
%\IEEEtriggercmd{\enlargethispage{-5in}}

% references section

% can use a bibliography generated by BibTeX as a .bbl file
% BibTeX documentation can be easily obtained at:
% http://mirror.ctan.org/biblio/bibtex/contrib/doc/
% The IEEEtran BibTeX style support page is at:
% http://www.michaelshell.org/tex/ieeetran/bibtex/
%\bibliographystyle{IEEEtran}
% argument is your BibTeX string definitions and bibliography database(s)
%\bibliography{IEEEabrv,../bib/paper}
%
% <OR> manually copy in the resultant .bbl file
% set second argument of \begin to the number of references
% (used to reserve space for the reference number labels box)
% \begin{thebibliography}{1}

% \bibitem{IEEEhowto:kopka}
% H.~Kopka and P.~W. Daly, \emph{A Guide to \LaTeX}, 3rd~ed.\hskip 1em plus
  % 0.5em minus 0.4em\relax Harlow, England: Addison-Wesley, 1999.

% \end{thebibliography}

% biography section
% 
% If you have an EPS/PDF photo (graphicx package needed) extra braces are
% needed around the contents of the optional argument to biography to prevent
% the LaTeX parser from getting confused when it sees the complicated
% \includegraphics command within an optional argument. (You could create
% your own custom macro containing the \includegraphics command to make things
% simpler here.)
%\begin{IEEEbiography}[{\includegraphics[width=1in,height=1.25in,clip,keepaspectratio]{mshell}}]{Michael Shell}
% or if you just want to reserve a space for a photo:

% \begin{IEEEbiography}{Michael D. Farrell}
% Biography text here.
% \end{IEEEbiography}
% \begin{IEEEbiographynophoto}{Michael D. Farrell}
\begin{IEEEbiography}[{\includegraphics[width=1in,height=1.25in,clip,keepaspectratio]{imgs/MF.jpg}}]{Michael
  D. Farrell}
  received the B.S. degree in mechanical engineering from Brigham Young
  University, Provo, UT, in 2017, where he is currently working toward the M.S.
  degree. His research focuses on autonomous landing of small UAVs using
  visual-inertial sensor fusion.
\end{IEEEbiography}
% \end{IEEEbiographynophoto}

% if you will not have a photo at all:
\begin{IEEEbiography}[{\includegraphics[width=1in,height=1.25in,clip,keepaspectratio]{imgs/TWM.jpg}}]{Timothy W. McLain}
  (S'91--M'93--SM'03)
is a professor of mechanical engineering at Brigham Young University. He
received BS and MS degrees from BYU and a PhD from Stanford University, all in
mechanical engineering. He joined BYU as a professor in 1995. During 1999 and 2000,
he was a visiting scientist at the Air Force Research Laboratory where he initiated
research in the guidance and control of unmanned aircraft systems. Since then, his UAS
research has attracted the support of the Air Force, the Army, DARPA, NASA, NSF, ONR,
NIST, and numerous companies. He is a co-author of the textbook Small Unmanned
Aircraft and the author of over 140 peer-reviewed articles. From 2012 to 2019, he was
the director of the Center for Unmanned Aircraft Systems sponsored by the National
Science Foundation. He currently serves as the associate dean of the Fulton College of
Engineering at BYU.
\end{IEEEbiography}

% insert where needed to balance the two columns on the last page with
% biographies
%\newpage

% \begin{IEEEbiographynophoto}{Jane Doe}
% Biography text here.
% \end{IEEEbiographynophoto}

% You can push biographies down or up by placing
% a \vfill before or after them. The appropriate
% use of \vfill depends on what kind of text is
% on the last page and whether or not the columns
% are being equalized.

%\vfill

% Can be used to pull up biographies so that the bottom of the last one
% is flush with the other column.
%\enlargethispage{-5in}


% that's all folks
\end{document}



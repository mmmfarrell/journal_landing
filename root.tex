\documentclass[journal,onecolumn]{IEEEtran}

\usepackage[hidelinks]{hyperref}
\usepackage{graphicx}
\usepackage{xcolor}
\usepackage{tabularx}
\usepackage{amsmath,amssymb,amsbsy,amsfonts}
\usepackage{accents}
\usepackage{nicefrac}
\usepackage{upgreek}
\usepackage[]{footmisc}

% !TEX root=root.tex

% vectors, quaternions, etc.
\newcommand*{\vect}[1]{\boldsymbol{\mathrm{#1}}}
\newcommand*{\quat}[1]{\boldsymbol{\mathrm{#1}}}

\newcommand*{\x}{\vect{x}}
\newcommand*{\xt}{\tilde{\vect{x}}}
\newcommand*{\xhat}{\hat{\vect{x}}}
\newcommand*{\xbar}{\bar{\vect{x}}}

\newcommand*{\z}{\vect{z}}
\newcommand*{\zhat}{\hat{\vect{z}}}
\newcommand*{\zbar}{\bar{\vect{z}}}

\newcommand*{\q}{\quat{q}}
\newcommand*{\e}{\vect{e}}
\newcommand*{\qt}{\tilde{\quat{q}}}
\newcommand*{\drag}{c_d}

% modifiers
% \newcommand*{\desired}[1]{\accentset{\circ}{#1}}
\newcommand*{\des}[1]{\check{#1}}

% norms, transpose, etc.
\newcommand*{\transpose}{\mathsf{T}}
\newcommand*{\skewmat}[1]{\left[ #1 \right]_{\times}}
\newcommand*{\norm}[1]{\left\Vert #1\right\Vert }
\newcommand*{\abs}[1]{\left\vert #1\right\vert }

% group/algebra names
\newcommand*{\SO}{\ensuremath{\mathit{SO}}}
\newcommand*{\so}{\ensuremath{\mathfrak{so}}}
\newcommand*{\SE}{\ensuremath{\mathit{SE}}}
\newcommand*{\se}{\ensuremath{\mathfrak{se}}}

% operators
\DeclareMathOperator*{\argmin}{arg\,min}
\DeclareMathOperator*{\Exp}{Exp}
\DeclareMathOperator*{\Log}{Log}

% reference commands
%\renewcommand*{\eqref}[1]{Eq.~\ref{#1}}
\renewcommand*{\eqref}[1]{(\ref{#1})}
\newcommand*{\tabref}[1]{Table~\ref{#1}}
\newcommand*{\secref}[1]{Sec.~\ref{#1}}
\newcommand*{\appxref}[1]{Appx.~\ref{#1}}

% comments
\definecolor{orange}{rgb}{1,0.5,0}
\definecolor{darkgreen}{rgb}{0,0.6,0}
\newcommand{\JJ}[1]{{\color{orange}{JJ: #1}}}
\newcommand{\RB}[1]{{\color{teal}{RB: #1}}}
\newcommand{\TM}[1]{{\color{purple}{TM: #1}}}
\newcommand{\DK}[1]{{\color{darkgreen}{DK: #1}}}
\newcommand{\JN}[1]{{\color{red}{JN: #1}}}
\newcommand{\MF}[1]{{\color{red}{MF: #1}}}

% theorem environments
\newtheorem{theorem}{Theorem}[subsection]
\newtheorem{corollary}{Corollary}[theorem]
\newtheorem{lemma}[theorem]{Lemma}



\begin{document}

\title{Visual Landing of UAV on Moving Vehicles}

\author{Michael~Farrell
        and~Tim~McLain
\thanks{Michael Farrell is a Masters student in the Department of Mechanical
Engineering, Brigham Young University, Provo, UT, 84602, USA.}% <-this % stops a space
\thanks{Tim McLain is a Professor in the Department of Mechanical
Engineering, Brigham Young University, Provo, UT, 84602, USA.}% <-this % stops a space
}

\maketitle

\begin{abstract}
The abstract goes here.
\end{abstract}

\section{Introduction}
% Big Picture
% Why important, what, why hard?
% Thecnical gaps
% Drones operating from static world
% Why drones operating/landing on moving vehicles
% !TEX root=../../master.tex

\IEEEPARstart{S}{mall} 
% Small
multirotor unmanned air vehicles (UAVs) have rapidly become popular platforms for
a variety of applications including
inspection, reconnaissance, and search and rescue.
% The ability of small multirotor UAVs to agily operate in confined spaces and to
% take off and land vertically give them a unique advantage over other robotic
% platforms.
For many of these use cases, UAVs are required to operate
autonomously, as skilled pilots are not feasible
since they are often unable to maintain direct line of sight to the UAV.
% due to
% limitations in line of sight. 
% A variety of emerging use cases
% require multirotor UAVs to operate from larger, mobile vehicles.
% a moving vehicle.
% instead of from the static world.
% These use
% cases include martitime surveillance, where the UAV must operate from
% a martitime vessel at sea,
% and package delivery, where the UAV must operate from a large truck in motion.
Newly emerging use cases such as maritime surveillance and package delivery
pose unique problems, requiring UAVs to operate autonomously from larger,
mobile vehicles instead of from a stationary base station.
% require UAVs to operate autonomously from larger, mobile vehicles.
% which carries the packages to be delivered.
% The ability to operate reliably from a
% moving vehicle is still a very active field of research, posing many unsolved
% problems.

% \cite{ling2014precision} AprilTag, kalman filter to predict
% \cite{araar2017vision} relies on a known map of many AprilTags fiducials.
% \cite{borowczyk2017autonomous} uses a single AprilTag, fast car
% \cite{baca2019autonomous} main MBZIRC 2017 paper (tag = square with x)
% \cite{falanga2017vision} MBZIRC, SVO + IMU
% \cite{beul2017fast} MBZIRC
% \cite{cantelli2017autonomous} MBZIRC
% \cite{marantos2018vision} helicopter, tag (Aruco/April)

% \cite{lee2012autonomous} IBVS
% \cite{wynn2019visual} IBVS - nested tag

Nearly all current approaches to the operation of multirotor UAVs with respect to moving
vehicles rely on the
% A variety of approaches to the operation of multirotor UAVs with respect to
% moving vehicles have been proposed in literature.
% Nearly all of these approaches rely on the
detection of a fiducial marker on the moving vehicle for relative pose
measurements. One of the earliest of these works
used a known configuration of infrared LEDs on the landing vehicle as a fiducial
marker~\cite{wenzel2011automatic}.
% relied on the detection of infrared LEDs that were
% illuminated in a known configuration on the landing
% vehicle~\cite{wenzel2011automatic}.
Since then, visual fiducial markers such as
AprilTags~\cite{olson2011tags} and ArUco markers~\cite{garrido2016generation}
have become more widely
used~\cite{ling2014precision,borowczyk2017autonomous,marantos2018vision,
araar2017vision}.
% In 2017, the Mohamed Bin Zayed International Robotics Challenge (MBZIRC)
% featured a stage that included landing a multirotor UAV on a visual fiducial
% marker atop a moving golf
% cart~\cite{baca2019autonomous,falanga2017vision,beul2017fast,cantelli2017autonomous}.
% While some landing methods employ image-based visual
% servoing~\cite{lee2012autonomous,wynn2019visual}, most methods
% require an estimate of the state of the target landing vehicle.
% While some landing methods control the UAV entirely based on the detections of the
% fiducial marker~\cite{lee2012autonomous,wynn2019visual}, more robust methods
% compute control based on an estimate of the state of the target landing
% vehicle~\cite{ling2014precision}.
Some landing methods employ image-based visual servoing techniques, depending
almost entirely on the detections of the fiducial
marker~\cite{lee2012autonomous,wynn2019visual}. These methods quickly fail when
the fiducial marker leaves the field of view of the camera or is otherwise not
detected. While recent research has worked toward ensuring the fiducial marker
remains detected during visual servoing~\cite{zheng2019towards}, more robust
methods compute control based on an estimate of the state of the target landing
vehicle~\cite{ling2014precision}.

% Make visual servoing more robust~\cite{zheng2019towards}.

% The Kalman filter~\cite{kalman} is a widely used algorithm 
% As we should not expect uninterrupted detection of the fiducial marker during
% landing, it is
% important to model the dynamics of the target vehicle and use them to propagate
% forward
As the UAV descends toward the landing target, it is common
that the fiducial marker remains undetected
% to not detect the fiducial marker
for periods of time due to poor lighting, occlusion, or extreme
motion.
% For this reason, it is important that a model of the dynamics of the target vehicle
For this reason, it is important that the dynamics of the target vehicle are
modeled and
used by the estimation algorithm to predict the state of the target vehicle
when measurements are not available.
The Kalman filter~\cite{kalman} has been frequently used
for this task, producing accurate estimates 
% showing good results 
when the fiducial marker is not detected for
short periods of time~\cite{baca2019autonomous}.
% n estimation method which models
% the dynamics of the target vehicle is employed to maintain accurate estimates in
% between measurements.
% Ling notes in~\cite{ling2014precision} that it is important to model the
% dynamics of the target vehicle so that they can be propagated forward for short
% periods of time when the fiducial marker is not
% detected.
Due to imperfect motion models, however,
all of the mentioned
approaches are likely to fail if the fiducial marker is not detected for
significant periods of time.
% due to imperfect motion models.
% because the target vehicle is likely to not
% perfectly follow the modeled motion.
% Even if the state of the target vehicle is estimated, the estimates are only as
% good as the
% motion model of the target vehicle when the fiducial marker is not detected.
% We propose an improvement to these methods: 
% To improve these methods, we propose an estimation algorithm that detects,
% tracks, and estimates the positions of unknown visual features on the target
% vehicle in addition to the fiducial marker.
To improve these methods, we propose an estimation algorithm that uses
measurements of unknown visual features on the target vehicle in addition to
measurements resulting from detections of a fiducial marker.
% to improve
% estimation accuracy when the fiducial marker is not detected for significant
% periods of time.

% Tracking and estimating the positions of visual features is
% a common technique used
% % similar to that commonly used
% in the field of visual odometry.
Many visual odometry methods such
as~\cite{qin2018vins,leutenegger2013keyframe,mourikis2007multi,mur2015orb,wang2018ego} use
detected and tracked visual features to aid in camera motion estimation.

% Track static features~\cite{wang2018ego}.

% as VINS-MONO, OKVIS, MSCKF, and ORB SLAM use an indirect approach
% \cite{qin2018vins} VINS-MONO, feature extraction, optical flow frame to frame
% \cite{leutenegger2013keyframe} OKVIS, BRISK features
% \cite{mourikis2007multi} MSCKF, uses SIFT features
% \cite{mur2015orb} ORB SLAM
In these methods, the tracked visual features are assumed to belong to the
static world.
Visual odometry techniques
% tracking static features
have also been previously applied to landing on
moving platforms~\cite{falanga2017vision}.
During landing, however, static features become more sparse as the dynamic target vehicle
occupies progessively more of the
field of view of the UAV's camera. This results in detoriating
estimates as the UAV approaches the landing target.
% During landing, however, it is common for the target vehicle to occupy the
% entire field of view of the UAV's camera, making it impossible to track static
% features throughout the duration of the flight.
% While visual odometry techniques have been
% previously applied to landing on
% moving platforms~\cite{falanga2017vision}, these methods still only tracked and
% estimated static features.
% During the landing phase, it is not uncommon, however, for the target vehicle
% to occupy the entire field of view of the UAV's camera, making it not possible
% to track static features throughout the duration of the flight.
For this reason, the proposed estimator, instead, tracks and estimates the
locations of
visual features that are rigidly attached to the target vehicle. These tracked
visual features provide information about the relative position of the target
vehicle
as long as the vehicle remains in the field of view of the camera.
% for the duration of the flight.
% movement of the UAV as well as
% information about the movement of the target vehicle.
We show in simulation and hardware experiments that the estimation of these tracked features allows
for accurate estimates of the state of the target vehicle even
when the fiducial marker is not detected for significant periods of time.


% Different approaches with fiducial markers - IBVS, Big Competition
% Visual Inertial Odometry - ROVIO, VINS-MONO, etc
% \input{introduction/related_work}
% % !TEX root=../root.tex

The outline of the paper is as follows.
\secref{sec:math_prelim} explains the mathematical notation and conventions used throughout the paper.
\secref{sec:estimation} presents the proposed estimation algorithm including the
state dynamics, state initialization and measurement models.
\secref{sec:est_paper_simulation} describes the simulation experiments conducted and
\secref{sec:est_paper_hardware} describes the hardware experiments conducted.
\secref{sec:est_paper_conclusion} provides concluding remarks.


This demo file is intended to serve as a ``starter file''
for IEEE journal papers produced under \LaTeX\ using
IEEEtran.cls version 1.8b and later.
I wish you the best of success. \cite{baldwin2007complementary}

\section{Model} \label{sec:model}
% !TEX root=../root.tex

\subsection{Notation}

Throughout the paper, we represent vectors with a bold letter (e.g., $\bf v$)
and matrices with a captial letter (e.g., $A$). Other common notation used
throughout the paper is contained below.
\begin{center}
\begin{tabularx}{\columnwidth}{lX}
$R_a^b$ & Rotation matrix from reference frame $a$ to $b$ \\
$\vect{v}_{a/b}^c$ & Vector state $\vect{v}$ of frame $a$ w.r.t.~frame $b$, expressed in frame $c$ \\
$\hat{a}$ & Estimate of true variable $a$ \\
$\bar{a}$ & Measurement of $a$ \\
% $\des{a}$ & Desired value of $a$ \\
$\dot{a}$ & Time derivative of $a$ \\
$\tilde{a}$ & Error of variable $a$, i.e., $\tilde{a} \triangleq a - \hat{a}$
\end{tabularx}
\end{center}
%
We also make use of the following coordinate frames:
\begin{center}
\begin{tabularx}{\columnwidth}{lX}
$I$ & The inertial coordinate frame in north-east-down\\
% $\ell$ & The aircraft's vehicle-1 (body-level) coordinate frame \\
$v$ & The aircraft's vehicle-1 (body-level) coordinate frame \\
$b$ & The aircraft's body-fixed coordinate frame \\
$c$ & The camera frame \\
$g$ & The landing vehicle's body-fixed coordinate frame located at the desired
landing location (goal) of the aircraft
\end{tabularx}
\end{center}
%
We use the standard basis vectors $\vect{e}_1, \vect{e}_2, \vect{e}_3$,
where $\vect{e}_1 = \begin{bmatrix} 1 & 0 & 0 \end{bmatrix}^\transpose$,
$\vect{e}_2 = \begin{bmatrix} 0 & 1 & 0 \end{bmatrix}^\transpose$,
and $\vect{e}_3 = \begin{bmatrix} 0 & 0 & 1 \end{bmatrix}^\transpose$.
We also use the skew-symmetric matrix operator
% We make frequent use of the skew-symmetric matrix operator defined by
\begin{equation}
  \skewmat{\vect{v}} \triangleq
  \begin{bmatrix}
  0 & -v_{3} & v_{2}\\
  v_{3} & 0 & -v_{1}\\
  -v_{2} & v_{1} & 0
  \end{bmatrix},
\end{equation}
which is related to the cross-product between two vectors as
\begin{equation}
  \vect{v}\times\vect{w}=\skewmat{\vect{v}}\vect{w}.
\end{equation}

% !TEX root=../root.tex

\subsection{UAV State Dynamics}
The state variables of the multirotor UAV are expressed as
\begin{equation}
  \x =
    \begin{bmatrix}
      \vect{p}_{b/I}^{I} & 
      \phi & \theta & \psi &
      \vect{v}_{b/I}^{b} 
    \end{bmatrix}^{\transpose}
\end{equation}
where $\phi$, $\theta$, and $\psi$ represent the Z-Y-X Euler angles known as
roll, pitch, and yaw. With these angles, we note that the rotation matrix
$R_I^b$ can be expressed as
\begin{equation*}
  R_I^b \left( \phi, \theta, \psi \right) =
  \begin{bmatrix}
    \ctheta \cpsi & \ctheta \spsi & -\stheta \\
    \sphi \stheta \cpsi - \cphi \spsi & \sphi \stheta \spsi + \cphi
    \cpsi & \sphi \ctheta \\
    \cphi \stheta \cpsi + \sphi \spsi & \cphi \stheta \spsi - \sphi
    \cpsi & \cphi \ctheta
  \end{bmatrix}.
\end{equation*}
We recognize that Euler angles do not form a proper vector space~\cite{??},
however, since we operate the UAV close to level flight, the approximation of
Euler angles as a vector space is satisfactory.

We use common UAV dynamics with the addition of a linear drag term as explained
in~\cite{leishman2014accel}. For this system, the inputs are given by
\begin{equation}
  \vect{u} =
    \begin{bmatrix}
      a_z & \vect{\omega}_{b/I}^b 
    \end{bmatrix}^\transpose
\end{equation}
and the dynamics given by
\begin{align}
  \dot{\vect{p}}_{b/I}^I
  &=
  \left( R_I^b \right)^\transpose \vect{v}_{b/I}^b 
  \\
  \begin{bmatrix}
    \dot{\phi} \\
    \dot{\theta} \\
    \dot{\psi}
  \end{bmatrix}
  &=
  \begin{bmatrix}
    1 & \sin\phi\tan\theta & \cos\phi\tan\theta \\
    0 & \cos\phi & -\sin\phi \\
    0 & \frac{\sin\phi}{\cos\theta} & \frac{\cos\phi}{\cos\theta}
  \end{bmatrix}
  \vect{\omega}_{b/I}^b
  \\
  \dot{\vect{v}}_{b/I}^b 
  &=
  R_I^b \vect{g}
  +
  \skewmat{\vect{v}_{b/I}^b} \vect{\omega}_{b/I}^b
  -
  a_z \e_3 \\
  &\qquad -
  \begin{bmatrix}
    \mu u \\
    \mu v \\
    0 \nonumber
  \end{bmatrix}.
\end{align}


% !TEX root=./root.tex

\subsection{Landing Vehicle Dynamics}
Without loss of generality, we assume that the motion of the landing vehicle can be modeled with a constant
velocity unicycle motion model. In our experiments we show that this very simple
model produces satisfactory results in a variety of scenarios. We point out,
however, that the motion model can easily be substituted for a more accurate
model for different use cases (i.e. a car model). For our case, the state of the landing vehicle is expressed as the tuple of
position, velocity, attitude and angular rotation rate
\begin{equation*}
  \x =
    \begin{pmatrix}
      \vect{p}_{g/I}^{I}, \vect{v}_{g/I}^{g}, \theta_{I}^{g},
      \omega_{g/I}^{g}
    \end{pmatrix}
    \in
    \mathbb{R}^2 \times \mathbb{R}^2 \times S^1 \times \mathbb{R}^1.
\end{equation*}
The value $\theta_{I}^g \in S^1$ represents the rotation from the inertial frame to the
goal frame about the inertial frame's $z$ axis. We note that the rotation matrix
derived from this rotation can be written as
\begin{equation}
  R \left( \theta \right) =
  \begin{bmatrix}
    \ctheta & -\stheta \\
    \stheta & \ctheta
  \end{bmatrix}.
\end{equation}

The constant velocity, unicyle motion model we use to describe the motion of the
landing vehicle can be written as
\begin{align}
  \dot{\vect{p}}_{g/I}^I
  &=
  \left( R_I^g \right)^\transpose \vect{v}_{g/I}^g  \\
  \dot{\vect{v}}_{g/I}^g
  &=
  \vect{0} + \vect{\eta}_v \\
  \dot{\theta}_{I}^g
  &=
  \omega_{g/I}^g \\
  \dot{\omega}_{g/I}^g
  &=
  0 +\eta_\omega,
\end{align}
where $\vect{\eta}_v$ and $\eta_\omega$ describe the random walk of the velocity
and rotational velocity of the landing vehicle.


\section{Estimation} \label{sec:estimation}
% !TEX root=./root.tex

As mentioned in~\secref{sec:TODO}, we estimate the state of the UAV and the
state of the landing vehicle in the same EKF.
The state of the combined system can be expressed as
\begin{equation}
  \x =
  \begin{bmatrix}
    \left(\x_{\text{UAV}}\right)^\transpose & \left(\x_{\text{Goal}}\right)^\transpose
  \end{bmatrix}^\transpose
\end{equation}
with the components defined as
\begin{align}
  \x_{\text{UAV}} &=
  \begin{bmatrix}
    \vect{p}_{b/I}^I &
    \phi & \theta & \psi &
    \vect{v}_{b/I}^b &
    \mu & \vect{\beta}_a & \vect{\beta}_\omega
  \end{bmatrix}^\transpose \\
  \x_{\text{Goal}} & =
    \begin{bmatrix}
      \vect{p}_{g/v}^{v} & \vect{v}_{g/I}^{g} & \theta_{I}^{g} &
      \omega_{g/I}^{g} &
      \vect{r}_{1/g}^{g} & \dots & \vect{r}_{n/g}^{g}
    \end{bmatrix}^{\transpose}.
\end{align}
The UAV states are self explanatory, covering position, attitude and velocity.
The Goal states, however, are a little more unique. They are described here:
\begin{itemize}
  \item $\vect{p}_{g/v}^{v}$ - The position of the goal frame w.r.t. vehicle
    frame, expressed in the vehicle frame. This position does not change as
    either the UAV or the goal rotates because the vehicle frame never rotates.
  \item $\rho$ - The inverse depth from the vehicle frame to the goal frame.
    Also note that rotation does not affect this state.
  \item $\vect{v}_{g/I}^{g}$ - The velocity of the goal frame w.r.t. inertial
    frame, expressed in the goal frame. The motion of the goal is most easily
    defined in the goal frame.
  \item $\theta_I^g$ - The angle that represents the rotation from the inertial
    frame to the goal frame. Note that this is only one value as the goal frame
    can only yaw while it is constrained to motion on the plane.
  \item $\omega_{g/I}^g$ - The angular rate of the goal frame w.r.t. inertial
    frame, expressed in the goal frame. This is the rate at which $\theta_I^g$
    changes.
  \item $\vect{r}_{i/g}^g$ - The location of landmark $i$ w.r.t. the goal frame,
    expressed in the goal frame. These landmarks are rigidly attached to the
    same landing vehicle as the goal frame, and therefore $\vect{r}_{i/g}^g$
    does not change, but is a constant.
\end{itemize}

\subsection{UAV State Dynamics}
We use common UAV dynamics with the addition of a linear drag term as explained
in~\cite{leishman2014accel}. The UAV dynamics are given by
\begin{align}
  \dot{\vect{p}}_{b/I}^I
  &=
  R_b^I \vect{v}_{b/I}^b 
  \\
  \begin{bmatrix}
    \dot{\phi} \\
    \dot{\theta} \\
    \dot{\psi}
  \end{bmatrix}
  &=
  \begin{bmatrix}
    1 & \sin\phi\tan\theta & \cos\phi\tan\theta \\
    0 & \cos\phi & -\sin\phi \\
    0 & \frac{\sin\phi}{\cos\theta} & \frac{\cos\phi}{\cos\theta}
  \end{bmatrix}
  \left( \bar{\vect{\omega}}_{b/I}^b - \vect{\beta}_\omega \right)
  \\
  \dot{\vect{v}}_{b/I}^b 
  &=
  R_I^b \vect{g}
  +
  \skewmat{\vect{v}_{b/I}^b}
  \left( \bar{\vect{\omega}}_{b/I}^b - \vect{\beta}_\omega \right)
  -
  \left( \bar{a}_z - \beta_{a_z} \right) \e_3
  -
  \begin{bmatrix}
    \mu u \\
    \mu v \\
    0
  \end{bmatrix}
  \\
  \dot{\mu} &= 0
  \\
  \dot{\vect{\beta}}_a &= \vect{0} + \vect{\eta}_{\beta_a}
  \\
  \dot{\vect{\beta}}_\omega &= \vect{0} + \vect{\eta}_{\beta_\omega}
\end{align}

\subsubsection{UAV State Jacobians}
\begin{equation}
  A =
  \begin{bmatrix}
    \vect{0} & \frac{\partial \dot{\vect{p}}}{\partial \vect{\theta}} &
    \frac{\partial \dot{\vect{p}}}{\partial \vect{v}} & \vect{0} & \vect{0} &
    \vect{0} \\
    \vect{0} & \frac{\partial \dot{\vect{\theta}}}{\partial \vect{\theta}} &
    \vect{0} & \vect{0} & \vect{0} &
    \frac{\partial \dot{\vect{\theta}}}{\partial \vect{\beta}_\omega} \\
    \vect{0} & \frac{\partial \dot{\vect{v}}}{\partial \vect{\theta}} &
    \frac{\partial \dot{\vect{v}}}{\partial \vect{v}} & \frac{\partial
    \dot{\vect{v}}}{\partial \mu} & 
    \frac{\partial \dot{\vect{v}}}{\partial \vect{\beta}_a} &
    \frac{\partial \dot{\vect{v}}}{\partial \vect{\beta}_\omega} \\
    \vect{0} & \vect{0} & \vect{0} & 0 & \vect{0} & \vect{0} \\
    \vect{0} & \vect{0} & \vect{0} & 0 & \vect{0} & \vect{0} \\
    \vect{0} & \vect{0} & \vect{0} & 0 & \vect{0} & \vect{0}
  \end{bmatrix}
\end{equation}
The partial jacobians are then given by:
\begin{align*}
  \dot{\vect{p}}_{b/I}^I &= R_b^I \vect{v}_{b/I}^b \\
  \frac{\partial \dot{\vect{p}}}{\partial \vect{\theta}} &= 
  \frac{\partial R_b^I}{\partial \vect{\theta}} \vect{v}_{b/I}^b \\
  \frac{\partial \dot{\vect{p}}}{\partial \vect{v}} &= R_b^I
\end{align*}

\begin{align*}
  \begin{bmatrix}
    \dot{\phi} \\
    \dot{\theta} \\
    \dot{\psi}
  \end{bmatrix}
  &=
  \begin{bmatrix}
    1 & \sin\phi\tan\theta & \cos\phi\tan\theta \\
    0 & \cos\phi & -\sin\phi \\
    0 & \frac{\sin\phi}{\cos\theta} & \frac{\cos\phi}{\cos\theta}
  \end{bmatrix}
  \left( \bar{\vect{\omega}}_{b/I}^b - \vect{\beta}_\omega \right) \\
  \frac{\partial \dot{\vect{\theta}}}{\partial \vect{\theta}} &= 
  \frac{\partial W_{\text{mat}}}{\partial \vect{\theta}} \left(
  \bar{\vect{\omega}}_{b/I}^b - \vect{\beta}_\omega \right) \\
  \frac{\partial \dot{\vect{\theta}}}{\partial \vect{\beta}_\omega} &= -
  \begin{bmatrix}
    1 & \sin\phi\tan\theta & \cos\phi\tan\theta \\
    0 & \cos\phi & -\sin\phi \\
    0 & \frac{\sin\phi}{\cos\theta} & \frac{\cos\phi}{\cos\theta}
  \end{bmatrix}
\end{align*}

\begin{align*}
  \dot{\vect{v}}_{b/I}^b 
  &=
  R_I^b \vect{g}
  +
  \skewmat{\vect{v}_{b/I}^b}
  \left( \bar{\vect{\omega}}_{b/I}^b - \vect{\beta}_\omega \right)
  -
  \left( \bar{a}_z - \beta_{a_z} \right) \e_3
  -
  \begin{bmatrix}
    \mu u \\
    \mu v \\
    0
  \end{bmatrix} \\
  \frac{\partial \dot{\vect{v}}}{\partial \vect{\theta}} &= 
  \frac{\partial R_I^b}{\partial \vect{\theta}} \vect{g} \\
  \frac{\partial \dot{\vect{v}}}{\partial \vect{v}} &= 
  -\skewmat{\bar{\vect{\omega}}_{b/I}^b - \vect{\beta}_\omega} -
  \begin{bmatrix}
    \mu & 0 & 0 \\
    0 & \mu & 0 \\
    0 & 0 & 0
  \end{bmatrix}\\
  \frac{\partial \dot{\vect{v}}}{\partial \mu} &= -
  \begin{bmatrix}
    u \\ v \\ 0
  \end{bmatrix}\\
  \\
  \frac{\partial \dot{\vect{v}}}{\partial \vect{\beta}_a} &= 
  \begin{bmatrix}
    0 & 0 & 0 \\
    0 & 0 & 0 \\
    0 & 0 & 1
  \end{bmatrix}\\
  \frac{\partial \dot{\vect{v}}}{\partial \vect{\beta}_\omega} &= 
  - \skewmat{\vect{v}_{b/I}^b}
\end{align*}

\subsubsection{Input Jacobians}
\begin{equation}
  B =
  \begin{bmatrix}
    0 & \vect{0} \\
    0 & \frac{\partial \dot{\vect{\theta}}}{\partial \vect{\omega}} \\
    \frac{\partial \dot{\vect{v}}}{\partial a_z} & \frac{\partial
      \dot{\vect{v}}}{\partial \vect{\omega}} \\
    0 & \vect{0} \\
    0 & \vect{0} \\
    0 & \vect{0}
  \end{bmatrix}
\end{equation}

\begin{align*}
  \frac{\partial \dot{\vect{\theta}}}{\partial \vect{\omega}} &=
  \begin{bmatrix}
    1 & \sin\phi\tan\theta & \cos\phi\tan\theta \\
    0 & \cos\phi & -\sin\phi \\
    0 & \frac{\sin\phi}{\cos\theta} & \frac{\cos\phi}{\cos\theta}
  \end{bmatrix} \\
  \frac{\partial \dot{\vect{v}}}{\partial a_z} &= 
  \begin{bmatrix}
    0 & 0 & -1
  \end{bmatrix}^\transpose \\
  \frac{\partial \dot{\vect{v}}}{\partial \vect{\omega}} &=
  \skewmat{\vect{v}_{b/I}^b}
\end{align*}

\subsection{Goal State Dynamics}
We assume that the landing vehicle moves with a constant velocity and a constant
angular velocity. The dynamics of the goal states are expressed as
\begin{align}
  \dot{\vect{p}}_{g/v}^{v} &= \left(R_{v}^{g}\right)^\transpose
  \vect{v}_{g/I}^{g} - I_{2 \times 3}\left(R_{v}^{b}\right)^\transpose \vect{v}_{b/I}^{b} \\
  \dot{\rho} &= \rho^{2} \vect{e}_{3}^\transpose \left(R_{v}^{b}\right)^\transpose \vect{v}_{b/I}^{b} \\
  \dot{\vect{v}}_{g/I}^{g} &= \vect{0} + \eta_{v} \\
  \dot{\theta}_{I}^{g} &= \omega_{g/I}^g \\
  \dot{\omega}_{g/I}^{g} &= 0 + \eta_{\omega} \\
  \dot{\vect r}_{i}^{b} &= \vect{0}.
\end{align}.

Note that the rotation from the vehicle frame to the goal frame is given by
\begin{equation}
  R_v^g =
  \begin{bmatrix}
    \ctheta & \stheta \\
    -\stheta & \ctheta
  \end{bmatrix}.
\end{equation}

\subsection{Motion Model Jacobians}
The full state jacobian is given by
\begin{equation}
  A =
  \begin{bmatrix}
    \frac{\partial \dot{\x}_{\text{UAV}}}{\partial \x_{\text{UAV}}} &
    \frac{\partial \dot{\x}_{\text{UAV}}}{\partial \x_{\text{Goal}}} \\
    \frac{\partial \dot{\x}_{\text{Goal}}}{\partial \x_{\text{UAV}}} &
    \frac{\partial \dot{\x}_{\text{Goal}}}{\partial \x_{\text{Goal}}} 
  \end{bmatrix}.
\end{equation}
We have defined the first term, $\frac{\partial \dot{\x}_{\text{UAV}}}{\partial
\x_{\text{UAV}}}$ above in the previous section. We also note that
\begin{equation}
  \frac{\partial \dot{\x}_{\text{UAV}}}{\partial \x_{\text{Goal}}} = \vect{0}
\end{equation}
as the UAV dynamics do not depend on the landing vehicle dynamics. We define the
remaining terms here below.

\subsubsection{Jacobian w.r.t. UAV State}
\begin{equation}
  \frac{\partial \dot{\x}_{\text{Goal}}}{\partial \x_{\text{UAV}}}
  =
  \begin{bmatrix}
    \vect{0} & \frac{\partial \dot{\vect{p}}_{\text{Goal}}}{\partial
      \vect{\theta}_{\text{UAV}}} & \frac{\partial
      \dot{\vect{p}}_{\text{Goal}}}{\partial \vect{v}_{\text{UAV}}} & 0 \\
    \vect{0} & \frac{\partial \dot{\rho}_{\text{Goal}}}{\partial
      \vect{\theta}_{\text{UAV}}} & \frac{\partial
      \dot{\rho}_{\text{Goal}}}{\partial \vect{v}_{\text{UAV}}} & 0 \\
      \vect{0} & 0 & \vect{0} & 0 \\
    \vect{0} & \vect{0} & \vect{0} & 0 \\
    \vect{0} & \vect{0} & \vect{0} & 0 \\
    \vect{0} & \vect{0} & \vect{0} & 0 \\
  \end{bmatrix}
\end{equation}
\begin{align}
    \frac{\partial \dot{\vect{p}}_{\text{Goal}}}{\partial
      \vect{\theta}_{\text{UAV}}}
      &=
      - I_{2 \times 3} \frac{\partial}{\partial \vect{\theta}_{\text{UAV}}}
      \left(R_{v}^{b}\right)^\transpose \vect{v}_{b/I}^{b}
      \\
    \frac{\partial \dot{\vect{p}}_{\text{Goal}}}{\partial \vect{v}_{\text{UAV}}}
      &=
      - I_{2 \times 3}\left(R_{v}^{b}\right)^\transpose
      \\
    \frac{\partial \dot{\rho}_{\text{Goal}}}{\partial
      \vect{\theta}_{\text{UAV}}}
      &=
      \rho^{2} \vect{e}_{3}^\transpose
        \frac{\partial}{\partial \vect{\theta}_{\text{UAV}}}
        \left(R_{v}^{b}\right)^\transpose \vect{v}_{b/I}^{b}
      \\
    \frac{\partial \dot{\rho}_{\text{Goal}}}{\partial \vect{v}_{\text{UAV}}}
      &=
      \rho^{2} \vect{e}_{3}^\transpose \left(R_{v}^{b}\right)^\transpose
      \\
\end{align}

\subsubsection{Jacobian w.r.t. Goal State}
\begin{equation}
  \frac{\partial \dot{\x}_{\text{Goal}}}{\partial \x_{\text{Goal}}}
  =
  \begin{bmatrix}
    \vect{0} & 0 & \frac{\partial \dot{\vect{p}}_{\text{Goal}}}{\partial
      \vect{v}_{\text{Goal}}} & \frac{\partial
      \dot{\vect{p}}_{\text{Goal}}}{\partial \theta_{\text{Goal}}} & 0 & \vect{0} & 0 \\
    \vect{0} & \frac{\partial \dot{\rho}}{\partial \rho} & \vect{0} & 0 & 0
             & \vect{0} & 0 \\
    \vect{0} & 0 & \vect{0} & 0 & 0 & \vect{0} & 0 \\
    \vect{0} & 0 & \vect{0} & 0 & \frac{\partial
      \dot{\theta}_{\text{Goal}}}{\partial \omega_{Goal}} & \vect{0} & 0 \\
    \vect{0} & 0 & \vect{0} & 0 & 0 & \vect{0} & 0 \\
    \vect{0} & 0 & \vect{0} & 0 & 0 & \vect{0} & 0
  \end{bmatrix}
\end{equation}
\begin{align}
    \frac{\partial \dot{\vect{p}}_{\text{Goal}}}{\partial
      \vect{v}_{\text{Goal}}}
      &=
      \left(R_{v}^{g}\right)^\transpose
      \\
    \frac{\partial \dot{\vect{p}}_{\text{Goal}}}{\partial \theta_{\text{Goal}}}
      &=
      \frac{\partial}{\partial \theta_{\text{Goal}}} \left(R_{v}^{g}\right)^\transpose \vect{v}_{g/I}^{g}
      \\
    \frac{\partial \dot{\rho}}{\partial \rho}
      &=
      2 \rho \vect{e}_{3}^\transpose \left(R_{v}^{b}\right)^\transpose \vect{v}_{b/I}^{b}
      \\
    \frac{\partial \dot{\theta}_{\text{Goal}}}{\partial \omega_{Goal}}
      &=
      1
\end{align}

\subsection{Accelerometer Measurement Update}
As in~\cite{leishman2014accel} we use the accelerometer measurements from the
$x$ and $y$ axes as a measurement update. The expected measurement due to drag
is expressed as
\begin{align}
  \bar{a}_x &= -\mu u + \beta_{a_x} \\
  \bar{a}_y &= -\mu v + \beta_{a_y}.
\end{align}
The measurement jacobian therefore is given by
\begin{align}
  \frac{\partial h_x}{\partial u} &= -\mu \\
  \frac{\partial h_x}{\partial \mu} &= -u \\
  \frac{\partial h_x}{\partial \beta_x} &= 1 \\
  \frac{\partial h_y}{\partial v} &= -\mu \\
  \frac{\partial h_y}{\partial \mu} &= -v \\
  \frac{\partial h_y}{\partial \beta_y} &= 1.
\end{align}

\subsection{Goal ArUco Measurement}
\subsubsection{Measurement model}
In the case that and ArUco marker is placed as the landing pad, and therefore
defines the goal frame, we get a measurement of relative transfrom from the
camera to the ArUco frame. If we assume that there is a known rotation offset
between the ArUco frame and the goal frame, the measurements can be expressed as
\begin{equation}
  h \left( \x \right) = 
  \begin{bmatrix}
    \vect{p}_{a/c}^c \\
    \vect{q}_c^a,
  \end{bmatrix}
\end{equation}
where $F^a$ is the ArUco frame.
We can expand this measurement model using our estimated state
\begin{align}
  \hat{h} \left( \x \right) &=
  \begin{bmatrix}
    \hat{\vect{p}}_{a/c}^c \\
    \hat{\vect{q}}_c^a,
  \end{bmatrix} \\
  \hat{\vect{p}}_{a/c}^c  &= R_b^c \left( \hat{R}_I^b \hat{\vect{p}}_{g/v}^v -
    \vect{p}_{c/v}^b \right) \\
  \hat{\vect{q}}_{c}^a  &= R_g^a \hat{R}_I^g \hat{R}_b^I R_c^b
\end{align}

\subsubsection{Measurement Jacobians}
\begin{align}
  \frac{\partial \hat{\vect{p}}_{a/c}^c}{\partial \phi} =& R_b^c \frac{\partial
  R_I^b}{\partial \phi} \hat{\vect{p}}_{g/v}^v \\
  \frac{\partial \hat{\vect{p}}_{a/c}^c}{\partial \theta} =& R_b^c \frac{\partial
  R_I^b}{\partial \theta} \hat{\vect{p}}_{g/v}^v \\
  \frac{\partial \hat{\vect{p}}_{a/c}^c}{\partial \psi} =& R_b^c \frac{\partial
  R_I^b}{\partial \psi} \hat{\vect{p}}_{g/v}^v \\
    \frac{\partial \hat{\vect{p}}_{a/c}^c}{\partial \vect{p}_{g/v}^v} =& R_b^c \hat{R}_I^b
\end{align}

\subsection{Goal Pixel Measurement}
\subsubsection{Measurement model}
We assume that the UAV is equiped with a downward facing camera that is rigidly
attached to the UAV. From this camera, we can get a measurement of the pixel
location of the goal landing location in the image frame
\begin{equation}
  \vect{z} =
  \begin{bmatrix}
    p_x & p_y
  \end{bmatrix}^\transpose.
\end{equation}
We can expand this measurements using our estimated states
\begin{align}
  \vect{z} &=
  \begin{bmatrix}
    p_x & p_y
  \end{bmatrix}^\transpose \\
  \vect{z} &=
  \begin{bmatrix}
    f_x \frac{\e_1 \vect{p}_{g/c}^c}{\e_3 \vect{p}_{g/c}^c} + c_x \\
    f_y \frac{\e_2 \vect{p}_{g/c}^c}{\e_3 \vect{p}_{g/c}^c} + c_y
  \end{bmatrix},
\end{align}
where $f_x$, $f_y$, $c_x$, and $c_y$ are constant parameters of the camera. We
can then express the position of the goal with respect to the camera in the
camera frame 
\begin{align}
  \vect{p}_{g/c}^c &= R_b^c R_v^b \left(\vect{p}_{g/v}^v -
    \vect{p}_{c/v}^v\right) \\
    \vect{p}_{g/c}^c &= R_b^c \left(R_v^b \vect{p}_{g/v}^v -
    \vect{p}_{c/b}^b\right)
\end{align}
where $R_b^c$ and $p_{c/b}^b$ are constant, known values and
\begin{equation}
  \vect{p}_{g/v}^v =
    \begin{bmatrix}
      \vect{p}_{g/v}^v(0) \\
      \vect{p}_{g/v}^v(1) \\
      1 / \rho
    \end{bmatrix}
\end{equation}
Therefore, the measurement model expanded looks like this
\begin{equation}
  \vect{z} =
  \begin{bmatrix}
    f_x \frac{\e_1 R_b^c \left(R_v^b \vect{p}_{g/v}^v - \vect{p}_{c/b}^b\right)}{\e_3 R_b^c \left(R_v^b \vect{p}_{g/v}^v - \vect{p}_{c/b}^b\right)} + c_x \\
    f_y \frac{\e_2 R_b^c \left(R_v^b \vect{p}_{g/v}^v - \vect{p}_{c/b}^b\right)}{\e_3 R_b^c \left(R_v^b \vect{p}_{g/v}^v - \vect{p}_{c/b}^b\right)} + c_y
  \end{bmatrix}.
\end{equation}

\subsubsection{Jacobian}
To derive  the jacobian of this measurement model, we first look at just the
first measurement, $p_x$ which can be written as
\begin{equation}
    p_x = f_x \left(\e_1 R_b^c \left(R_v^b \vect{p}_{g/v}^v -
      \vect{p}_{c/b}^b\right)\right) \left(\e_3 R_b^c \left(R_v^b \vect{p}_{g/v}^v -
      \vect{p}_{c/b}^b\right)\right)^{-1} + c_x.
\end{equation}
Note that all values are constants except for $\vect{p}_{g/v}^v$ and $R_v^b$.
The individual parts of the jacobian are given here:
\begin{equation}
  \frac{\partial p_x}{\partial \vect{p}_{g/v}^v} =
  \frac{f_x \e_1 R_b^c R_v^b}
    {\left(\e_3 R_b^c \left(R_v^b \vect{p}_{g/v}^v -
    \vect{p}_{c/b}^b\right)\right)}
    - \frac{\left(\e_3 R_b^c R_v^b \right) f_x \left(\e_1 R_b^c \left(R_v^b \vect{p}_{g/v}^v -
        \vect{p}_{c/b}^b\right)\right)} {\left(\e_3 R_b^c \left(R_v^b \vect{p}_{g/v}^v -
  \vect{p}_{c/b}^b\right)\right)^2}
\end{equation}
where only the first two (of three) entries of the vector are used. The third
entry is used for the jacobian w.r.t. $\rho$ as
\begin{equation}
  \frac{\partial p_x}{\partial \rho} = -\frac{1}{\rho^2} \frac{\partial p_x}{\partial
  \vect{p}_{g/v}^v}\left(2\right).
\end{equation}
The derivatives for the euler angles are given by
\begin{equation}
  \frac{\partial p_x}{\partial \phi} =
  \frac{f_x \e_1 R_b^c \frac{\partial R_v^b}{\partial \phi} \vect{p}_{g/v}^v}
    {\left(\e_3 R_b^c \left(R_v^b \vect{p}_{g/v}^v -
    \vect{p}_{c/b}^b\right)\right)}
    - \frac{\left(\e_3 R_b^c \frac{\partial R_v^b}{\partial \phi} \vect{p}_{g/v}^v \right) f_x \left(\e_1 R_b^c \left(R_v^b \vect{p}_{g/v}^v -
        \vect{p}_{c/b}^b\right)\right)} {\left(\e_3 R_b^c \left(R_v^b \vect{p}_{g/v}^v -
  \vect{p}_{c/b}^b\right)\right)^2}
\end{equation}
where the jacobians for the other angles are similar, just substituting in. The
jacobians for the y pixel measurement, $p_y$ are similar to the ones derived
above, just changing $f_y$ in place of $f_x$ and $\e_2$ instead of $\e_1$.


% Python code
%d1 = -np.matmul(E3, np.matmul(RBC, R_I_b)) * FX * (p_g_c_c[0] / p_g_c_c[2] /
        %p_g_c_c[2]) + FX * np.matmul(E1, np.matmul(RBC, R_I_b)) \
                %/ p_g_c_c[2]
%d2 = -np.matmul(E3, np.matmul(RBC, R_I_b)) * FY * (p_g_c_c[1] / p_g_c_c[2] /
        %p_g_c_c[2]) + FY * np.matmul(E2, np.matmul(RBC, R_I_b)) \
                %/ p_g_c_c[2]
%jac[0, 10:12] = d1[0:2]
%jac[1, 10:12] = d2[0:2]

%# d / d rho
%jac[0, 12] = -d1[2] / rho_g / rho_g
%jac[1, 12] = -d2[2] / rho_g / rho_g

%d1dphi = -np.matmul(E3, np.matmul(RBC, np.matmul(dRdPhi, p_g_v_v))) * FX * (p_g_c_c[0] / p_g_c_c[2] /
        %p_g_c_c[2]) + FX * np.matmul(E1, np.matmul(RBC, np.matmul(dRdPhi,
            %p_g_v_v))) / p_g_c_c[2]
%d1dtheta = -np.matmul(E3, np.matmul(RBC, np.matmul(dRdTheta, p_g_v_v))) * FX * (p_g_c_c[0] / p_g_c_c[2] /
        %p_g_c_c[2]) + FX * np.matmul(E1, np.matmul(RBC, np.matmul(dRdTheta,
            %p_g_v_v))) / p_g_c_c[2]
%d1dpsi = -np.matmul(E3, np.matmul(RBC, np.matmul(dRdPsi, p_g_v_v))) * FX * (p_g_c_c[0] / p_g_c_c[2] /
        %p_g_c_c[2]) + FX * np.matmul(E1, np.matmul(RBC, np.matmul(dRdPsi,
            %p_g_v_v))) / p_g_c_c[2]

%jac[0, 3] = d1dphi
%jac[0, 4] = d1dtheta
%jac[0, 5] = d1dpsi

\subsection{Goal Depth Measurement}
We assume that the camera also provides a measurement of the distance to the
goal. This is given in the form of the z coordinate of the goal location in the
camera frame.

\subsubsection{Measurement Model}
We can express the goal depth measurement as
\begin{equation}
  \vect{z} = \e_3^\transpose \vect{p}_{g/c}^c.
\end{equation}
We can expand this to be in terms of our estimated state as
\begin{align}
  \vect{z} &= \e_3^\transpose \vect{p}_{g/c}^c \\
  \vect{z} &= \e_3^\transpose R_b^c R_v^b \vect{p}_{g/c}^v \\
  %\vect{z} &= \e_3^\transpose R_b^c R_v^b \left( \vect{p}_{g/v}^v \right) \\
  \vect{z} &= \e_3 R_b^c \left(R_v^b \vect{p}_{g/v}^v -
    \vect{p}_{c/b}^b\right).
\end{align}

\subsubsection{Measurement Jacobian}
As seen in the measurement model above, the measurement is dependent on the UAV
attitude as well as the goal position. The jacobian terms for these parts are
given by
\begin{align}
  \frac{\partial \vect{z}}{\partial \phi} &= \e_3^\transpose R_b^c \frac{
    \partial R_v^b}{\partial \phi} \left( \vect{p}_{g/v}^v \right) \\
  \frac{\partial \vect{z}}{\partial \theta} &= \e_3^\transpose R_b^c \frac{
    \partial R_v^b}{\partial \theta} \left( \vect{p}_{g/v}^v \right) \\
  \frac{\partial \vect{z}}{\partial \psi} &= \e_3^\transpose R_b^c \frac{
    \partial R_v^b}{\partial \psi} \left( \vect{p}_{g/v}^v \right) \\
    \frac{\partial \vect{z}}{\partial \vect{p}_{g/v}^v} &= \e_3^\transpose R_b^c
    R_v^b \\
  \frac{\partial \vect{z}}{\partial \rho} &= -\frac{1}{\rho^2}
    \frac{\partial \vect{z}}{\partial \vect{p}_{g/v}^v}\left(2\right).
\end{align}

\subsection{Landmark Pixel Measurement}
We assume that landmark locations that are rigidly attached to the landing
vehicle are tracked in the camera image. For now we will just have a fixed
number of landmarks that are tracked for the entire duration of the flight, even
when they leave the FOV of the camera and reenter. In real life, however, the
landmarks must be dynamically added and removed from the estimated states vector
as they leave the FOV of the camera and more are acquired.

\subsubsection{Measurement Model}
The measurement is a pixel location of each landmark in the camera image
\begin{equation}
  \vect{z} =
  \begin{bmatrix}
    p_x & p_y
  \end{bmatrix}^\transpose.
\end{equation}
We can expand this measurements using our estimated states
\begin{align}
  \vect{z} &=
  \begin{bmatrix}
    p_x & p_y
  \end{bmatrix}^\transpose \\
  \vect{z} &=
  \begin{bmatrix}
    f_x \frac{\e_1 \vect{p}_{i/c}^c}{\e_3 \vect{p}_{i/c}^c} + c_x \\
    f_y \frac{\e_2 \vect{p}_{i/c}^c}{\e_3 \vect{p}_{i/c}^c} + c_y
  \end{bmatrix},
\end{align}
where $f_x$, $f_y$, $c_x$, and $c_y$ are constant parameters of the camera. We
can then express the position of the landmark $i$ with respect to the camera in the
camera frame as
\begin{align}
  \vect{p}_{i/c}^c &= R_b^c R_v^b \left(\vect{p}_{i/v}^v -
    \vect{p}_{c/v}^v\right) \\
    \vect{p}_{i/c}^c &= R_b^c \left(R_v^b \vect{p}_{i/v}^v -
    \vect{p}_{c/b}^b\right)
\end{align}
This is actually a little bit weird because of the state parameters, but
\begin{equation}
  \vect{p}_{i/v}^v =
  \begin{bmatrix}
    \e_1^\transpose \left( R_g^v \vect{p}_{i/g}^g + \vect{p}_{g/v}^v \right) \\
    \e_2^\transpose \left( R_g^v \vect{p}_{i/g}^g + \vect{p}_{g/v}^v \right) \\
    \left( \e_3^\transpose \vect{p}_{i/g}^g + 1 / \rho \right)
  \end{bmatrix}.
\end{equation}

\subsubsection{Measurement Jacobian}
For the purpose of deriving the jacobians, we will just look at the jacobians
with respect to $p_x$, the pixel location along the $x$ axis of the camera
frame. From above we have that
\begin{align}
  p_x &= f_x \frac{\e_1^\transpose \vect{p}_{i/c}^c}{\e_3^\transpose \vect{p}_{i/c}^c} + c_x \\ 
  p_x &= f_x \frac{\e_1^\transpose R_b^c \left(R_v^b \vect{p}_{i/v}^v -
      \vect{p}_{c/b}^b\right) }{\e_3^\transpose R_b^c \left(R_v^b \vect{p}_{i/v}^v -
  \vect{p}_{c/b}^b\right) } + c_x 
\end{align}
Note that all values are constants except for $\vect{p}_{i/v}^v$ and $R_v^b$.
The individual parts of the jacobian are given here:
\begin{equation}
  \frac{\partial p_x}{\partial \vect{p}_{g/v}^v} =
  \frac{f_x \e_1 R_b^c R_v^b}
    {\left(\e_3 R_b^c \left(R_v^b \vect{p}_{i/v}^v -
    \vect{p}_{c/b}^b\right)\right)}
    - \frac{\left(\e_3 R_b^c R_v^b \right) f_x \left(\e_1 R_b^c \left(R_v^b \vect{p}_{i/v}^v -
        \vect{p}_{c/b}^b\right)\right)} {\left(\e_3 R_b^c \left(R_v^b \vect{p}_{i/v}^v -
  \vect{p}_{c/b}^b\right)\right)^2}
\end{equation}
where only the first two (of three) entries of the vector are used. The third
entry is used for the jacobian w.r.t. $\rho$ as
\begin{equation}
  \frac{\partial p_x}{\partial \rho} = -\frac{1}{\rho^2} \frac{\partial p_x}{\partial
  \vect{p}_{g/v}^v}\left(2\right).
\end{equation}
The derivatives for the euler angles are given by
\begin{equation}
  \frac{\partial p_x}{\partial \phi} =
  \frac{f_x \e_1 R_b^c \frac{\partial R_v^b}{\partial \phi} \vect{p}_{i/v}^v}
    {\left(\e_3 R_b^c \left(R_v^b \vect{p}_{i/v}^v -
    \vect{p}_{c/b}^b\right)\right)}
    - \frac{\left(\e_3 R_b^c \frac{\partial R_v^b}{\partial \phi} \vect{p}_{i/v}^v \right) f_x \left(\e_1 R_b^c \left(R_v^b \vect{p}_{i/v}^v -
        \vect{p}_{c/b}^b\right)\right)} {\left(\e_3 R_b^c \left(R_v^b \vect{p}_{i/v}^v -
  \vect{p}_{c/b}^b\right)\right)^2}
\end{equation}
where the jacobians for the other angles are similar, just substituting in.
The derivative for the goal angle, $\theta_g$ is given by using
\begin{equation}
  \frac{\partial \vect{p}_{i/v}^v}{\partial \theta_g} =
  \begin{bmatrix}
    \e_1^\transpose \frac{\partial R_g^v}{\partial \theta_g} \vect{p}_{i/g}^g \\
    \e_2^\transpose \frac{\partial R_g^v}{\partial \theta_g} \vect{p}_{i/g}^g \\
    0
  \end{bmatrix}
\end{equation}
in the jacobian given by
\begin{equation}
  \frac{\partial p_x}{\partial \theta_g} =
  \frac{f_x \e_1 R_b^c R_v^b \frac{\partial \vect{p}_{i/v}^v}{\partial \theta_g}}
    {\left(\e_3 R_b^c \left(R_v^b \vect{p}_{i/v}^v -
    \vect{p}_{c/b}^b\right)\right)}
    - \frac{\left(\e_3 R_b^c R_v^b \frac{\partial \vect{p}_{i/v}^v}{\partial \theta_g} \right) f_x \left(\e_1 R_b^c \left(R_v^b \vect{p}_{i/v}^v -
        \vect{p}_{c/b}^b\right)\right)} {\left(\e_3 R_b^c \left(R_v^b \vect{p}_{i/v}^v -
  \vect{p}_{c/b}^b\right)\right)^2}
\end{equation}
Similarly, the derivative for the landmark offset, $\vect{r}_i$ or
$\vect{p}_{i/g}^g$ is given by using
\begin{equation}
  \frac{\partial \vect{p}_{i/v}^v}{\partial \vect{p}_{i/g}^g} =
  \begin{bmatrix}
    \e_1^\transpose R_g^v \\
    \e_2^\transpose R_g^v \\
    \e_3^\transpose
  \end{bmatrix}
\end{equation}
in the jacobian given by
\begin{equation}
  \frac{\partial p_x}{\partial \vect{p}_{i/g}^g} =
  \frac{f_x \e_1 R_b^c R_v^b \frac{\partial \vect{p}_{i/v}^v}{\partial \vect{p}_{i/g}^g}}
    {\left(\e_3 R_b^c \left(R_v^b \vect{p}_{i/v}^v -
    \vect{p}_{c/b}^b\right)\right)}
    - \frac{\left(\e_3 R_b^c R_v^b \frac{\partial \vect{p}_{i/v}^v}{\partial \vect{p}_{i/g}^g} \right) f_x \left(\e_1 R_b^c \left(R_v^b \vect{p}_{i/v}^v -
        \vect{p}_{c/b}^b\right)\right)} {\left(\e_3 R_b^c \left(R_v^b \vect{p}_{i/v}^v -
  \vect{p}_{c/b}^b\right)\right)^2}
\end{equation}
The jacobians for the y pixel measurement, $p_y$ are similar to the ones derived
above, just changing $f_y$ in place of $f_x$ and $\e_2$ instead of $\e_1$.

\subsection{Landmark Pixel Measurement 2.0}

\subsubsection{Measurement Model}
The measurement is a pixel location of each landmark in the camera image
\begin{equation}
  \vect{z} =
  \begin{bmatrix}
    p_x & p_y
  \end{bmatrix}^\transpose.
\end{equation}
We can expand this measurements using our estimated states
\begin{align}
  \vect{z} &=
  \begin{bmatrix}
    p_x & p_y
  \end{bmatrix}^\transpose \\
  \vect{z} &=
  \begin{bmatrix}
    f_x \frac{\e_1 \vect{p}_{i/c}^c}{\e_3 \vect{p}_{i/c}^c} + c_x \\
    f_y \frac{\e_2 \vect{p}_{i/c}^c}{\e_3 \vect{p}_{i/c}^c} + c_y
  \end{bmatrix},
\end{align}
where $f_x$, $f_y$, $c_x$, and $c_y$ are constant parameters of the camera. We
can then express the position of the landmark $i$ with respect to the camera in the
camera frame as
\begin{align}
  \vect{p}_{i/c}^c &= R_b^c R_v^b \left(\vect{p}_{i/v}^v -
    \vect{p}_{c/v}^v\right) \\
    \vect{p}_{i/c}^c &= R_b^c \left(R_v^b \vect{p}_{i/v}^v -
    \vect{p}_{c/b}^b\right)
\end{align}
This is actually a little bit weird because of the state parameters, but
\begin{equation}
  \vect{p}_{i/v}^v =
  \begin{bmatrix}
    \e_1^\transpose \left( R_g^v \vect{p}_{i/g}^g + \vect{p}_{g/v}^v \right) \\
    \e_2^\transpose \left( R_g^v \vect{p}_{i/g}^g + \vect{p}_{g/v}^v \right) \\
    \e_3^\transpose \left( \vect{p}_{i/g}^g - \vect{p}_{b/I}^I \right)
  \end{bmatrix}.
\end{equation}

\subsubsection{Measurement Jacobian}
For the purpose of deriving the jacobians, we will just look at the jacobians
with respect to $p_x$, the pixel location along the $x$ axis of the camera
frame. From above we have that
\begin{align}
  p_x &= f_x \frac{\e_1^\transpose \vect{p}_{i/c}^c}{\e_3^\transpose \vect{p}_{i/c}^c} + c_x \\ 
  p_x &= f_x \frac{\e_1^\transpose R_b^c \left(R_v^b \vect{p}_{i/v}^v -
      \vect{p}_{c/b}^b\right) }{\e_3^\transpose R_b^c \left(R_v^b \vect{p}_{i/v}^v -
  \vect{p}_{c/b}^b\right) } + c_x 
\end{align}
Note that all values are constants except for $\vect{p}_{i/v}^v$ and $R_v^b$.
The individual parts of the jacobian are given here:
\begin{equation}
  \frac{\partial p_x}{\partial \vect{p}_{g/v}^v} =
  \frac{f_x \e_1 R_b^c R_v^b}
    {\left(\e_3 R_b^c \left(R_v^b \vect{p}_{i/v}^v -
    \vect{p}_{c/b}^b\right)\right)}
    - \frac{\left(\e_3 R_b^c R_v^b \right) f_x \left(\e_1 R_b^c \left(R_v^b \vect{p}_{i/v}^v -
        \vect{p}_{c/b}^b\right)\right)} {\left(\e_3 R_b^c \left(R_v^b \vect{p}_{i/v}^v -
  \vect{p}_{c/b}^b\right)\right)^2}
\end{equation}
where only the first two (of three) entries of the vector are used. The third
entry is used for the jacobian w.r.t. $\vect{p}_{b/I}^I(2)$ as
\begin{equation}
  \frac{\partial p_x}{\partial \vect{p}_{b/I}^I(2)} = -\frac{\partial p_x}{\partial
  \vect{p}_{g/v}^v}\left(2\right).
\end{equation}
The derivatives for the attitude representation are less straight forward. From
"Micro Lie Theory" we use the fact that the jacobian of the rotation action can
be shown to be
\begin{equation}
  J_R^{R \cdot v} = -R \skewmat{v}.
\end{equation}
\begin{equation}
  \frac{\partial p_x}{\partial \q} =
  \frac{-f_x \e_1 R_b^c R_v^b \skewmat{\vect{p}_{i/v}^v}}
    {\left(\e_3 R_b^c \left(R_v^b \vect{p}_{i/v}^v -
    \vect{p}_{c/b}^b\right)\right)}
    - \frac{\left(-\e_3 R_b^c R_v^b \skewmat{\vect{p}_{i/v}^v} \right) f_x \left(\e_1 R_b^c \left(R_v^b \vect{p}_{i/v}^v -
        \vect{p}_{c/b}^b\right)\right)} {\left(\e_3 R_b^c \left(R_v^b \vect{p}_{i/v}^v -
  \vect{p}_{c/b}^b\right)\right)^2}
\end{equation}
where the jacobians for the other angles are similar, just substituting in.
The derivative for the goal angle, $\theta_g$ is given by using
\begin{equation}
  \frac{\partial \vect{p}_{i/v}^v}{\partial \theta_g} =
  \begin{bmatrix}
    \e_1^\transpose \frac{\partial R_g^v}{\partial \theta_g} \vect{p}_{i/g}^g \\
    \e_2^\transpose \frac{\partial R_g^v}{\partial \theta_g} \vect{p}_{i/g}^g \\
    0
  \end{bmatrix}
\end{equation}
in the jacobian given by
\begin{equation}
  \frac{\partial p_x}{\partial \theta_g} =
  \frac{f_x \e_1 R_b^c R_v^b \frac{\partial \vect{p}_{i/v}^v}{\partial \theta_g}}
    {\left(\e_3 R_b^c \left(R_v^b \vect{p}_{i/v}^v -
    \vect{p}_{c/b}^b\right)\right)}
    - \frac{\left(\e_3 R_b^c R_v^b \frac{\partial \vect{p}_{i/v}^v}{\partial \theta_g} \right) f_x \left(\e_1 R_b^c \left(R_v^b \vect{p}_{i/v}^v -
        \vect{p}_{c/b}^b\right)\right)} {\left(\e_3 R_b^c \left(R_v^b \vect{p}_{i/v}^v -
  \vect{p}_{c/b}^b\right)\right)^2}
\end{equation}
Similarly, the derivative for the landmark offset, $\vect{r}_i$ or
$\vect{p}_{i/g}^g$ is given by using
\begin{equation}
  \frac{\partial \vect{p}_{i/v}^v}{\partial \vect{p}_{i/g}^g} =
  \begin{bmatrix}
    \e_1^\transpose R_g^v \\
    \e_2^\transpose R_g^v \\
    \e_3^\transpose
  \end{bmatrix}
\end{equation}
in the jacobian given by
\begin{equation}
  \frac{\partial p_x}{\partial \vect{p}_{i/g}^g} =
  \frac{f_x \e_1 R_b^c R_v^b \frac{\partial \vect{p}_{i/v}^v}{\partial \vect{p}_{i/g}^g}}
    {\left(\e_3 R_b^c \left(R_v^b \vect{p}_{i/v}^v -
    \vect{p}_{c/b}^b\right)\right)}
    - \frac{\left(\e_3 R_b^c R_v^b \frac{\partial \vect{p}_{i/v}^v}{\partial \vect{p}_{i/g}^g} \right) f_x \left(\e_1 R_b^c \left(R_v^b \vect{p}_{i/v}^v -
        \vect{p}_{c/b}^b\right)\right)} {\left(\e_3 R_b^c \left(R_v^b \vect{p}_{i/v}^v -
  \vect{p}_{c/b}^b\right)\right)^2}
\end{equation}
The jacobians for the y pixel measurement, $p_y$ are similar to the ones derived
above, just changing $f_y$ in place of $f_x$ and $\e_2$ instead of $\e_1$.

\subsection{Landmark Initialization}
Each time a new landmark is acquired and added to the estimated vector, we
should initialize its location to something intelligent. Theoretically, we could
just initialize the landmark's location to zero with a large enough associated
covariance. However, by initializing the landmark to the location based on its
initial pixel measurement, we can initialize the landmark with a smaller
covariance.

When we first acquire a landmark, the only information we have about it is a
measurement of its pixel location in the image frame. This measurement, denoted
by
\begin{equation}
  h \left( \x \right) = 
  \begin{bmatrix}
    p_x & p_y
  \end{bmatrix}^\transpose
\end{equation}
can be used to deduce the landmark's relative position vector that is estimated.
The easiest way to see this is by first creating a virtual image plane. This
virtual image plane projects the landmark pixel location into the image plane as
if the camera were perfectly aligned with the inertial, vehicle frame. We can
derive this using the pin hole camera model where
\begin{align}
  \begin{bmatrix}
    p_x \\ p_y \\ 1
  \end{bmatrix} &= K
  \begin{bmatrix}
    X^c / Z^c \\
    Y^c / Z^c \\
    1
  \end{bmatrix} \\
  \begin{bmatrix}
    X^c / Z^c \\
    Y^c / Z^c \\
    1
  \end{bmatrix}
   &= K^{-1}
  \begin{bmatrix}
    p_x \\ p_y \\ 1
  \end{bmatrix} \\
  \begin{bmatrix}
    X^c / Z^c \\
    Y^c / Z^c \\
    1
  \end{bmatrix}^v
   &= R_b^v R_c^b K^{-1}
  \begin{bmatrix}
    p_x \\ p_y \\ 1
  \end{bmatrix} \\
\end{align}
The resulting vector, $
  \begin{bmatrix}
    X^c / Z^c \\
    Y^c / Z^c \\
    1
  \end{bmatrix}^v$
  is the vector $\vect{p}_{i/c}^v$ up to a scale factor. The vector can be scaled by assuming that the altitude of the landmark is equal to the
altitude of the goal. To get the expected altitude, we solve for
\begin{align}
  \e_3^\transpose \vect{p}_{g/c}^v &= \e_3^\transpose \vect{p}_{g/b}^v - \e_3^\transpose \vect{p}_{c/b}^v \\
  \e_3^\transpose \vect{p}_{g/c}^v &= \e_3^\transpose \vect{p}_{g/b}^v -
  \e_3^\transpose R_b^I \vect{p}_{c/b}^b \\
  \e_3^\transpose \vect{p}_{g/c}^v &= \frac{1}{\rho_g} -
  \e_3^\transpose R_b^I \vect{p}_{c/b}^b \\
\end{align}

This gives us the vector, $\vect{p}_{i/c}^v$.
With this vector, we can then reach the estimated state vector,
$\vect{p}_{i/g}^g$ with the following
\begin{align}
  \vect{p}_{i/v}^v &= \vect{p}_{i/c}^v + R_b^I \vect{p}_{c/b}^b \\
  \vect{p}_{i/g}^g &= R_v^g \left( \vect{p}_{i/v}^v - \vect{p}_{g/v}^v \right).
\end{align}


% % !TEX root=./root.tex

\section{Estimator using normal depth}
\subsection{Goal State Dynamics}
We assume that the landing vehicle moves with a constant velocity and a constant
angular velocity. The dynamics of the goal states are expressed as
\begin{align}
  \dot{\vect{p}}_{g/v}^{v} &= \left(R_{v}^{g}\right)^\transpose
  \vect{v}_{g/I}^{g} - \left(R_{v}^{b}\right)^\transpose \vect{v}_{b/I}^{b} \\
  \dot{\vect{v}}_{g/I}^{g} &= \vect{0} + \eta_{v} \\
  \dot{\theta}_{I}^{g} &= \omega_{g/I}^g \\
  \dot{\omega}_{g/I}^{g} &= 0 + \eta_{\omega} \\
  \dot{\vect r}_{i}^{b} &= \vect{0}.
\end{align}.

Note that the rotation from the vehicle frame to the goal frame is given by
\begin{equation}
  R_v^g =
  \begin{bmatrix}
    \ctheta & \stheta \\
    -\stheta & \ctheta
  \end{bmatrix}.
\end{equation}

\subsection{Motion Model Jacobians}
The full state jacobian is given by
\begin{equation}
  A =
  \begin{bmatrix}
    \frac{\partial \dot{\x}_{\text{UAV}}}{\partial \x_{\text{UAV}}} &
    \frac{\partial \dot{\x}_{\text{UAV}}}{\partial \x_{\text{Goal}}} \\
    \frac{\partial \dot{\x}_{\text{Goal}}}{\partial \x_{\text{UAV}}} &
    \frac{\partial \dot{\x}_{\text{Goal}}}{\partial \x_{\text{Goal}}} 
  \end{bmatrix}.
\end{equation}
We have defined the first term, $\frac{\partial \dot{\x}_{\text{UAV}}}{\partial
\x_{\text{UAV}}}$ above in the previous section. We also note that
\begin{equation}
  \frac{\partial \dot{\x}_{\text{UAV}}}{\partial \x_{\text{Goal}}} = \vect{0}
\end{equation}
as the UAV dynamics do not depend on the landing vehicle dynamics. We define the
remaining terms here below.

\subsubsection{Jacobian w.r.t. UAV State}
\begin{equation}
  \frac{\partial \dot{\x}_{\text{Goal}}}{\partial \x_{\text{UAV}}}
  =
  \begin{bmatrix}
    \vect{0} & \frac{\partial \dot{\vect{p}}_{\text{Goal}}}{\partial
      \vect{\theta}_{\text{UAV}}} & \frac{\partial
      \dot{\vect{p}}_{\text{Goal}}}{\partial \vect{v}_{\text{UAV}}} & 0 \\
    \vect{0} & \vect{0} & \vect{0} & 0 \\
    \vect{0} & \vect{0} & \vect{0} & 0 \\
    \vect{0} & \vect{0} & \vect{0} & 0 \\
  \end{bmatrix}
\end{equation}
\begin{align}
    \frac{\partial \dot{\vect{p}}_{\text{Goal}}}{\partial
      \vect{\theta}_{\text{UAV}}}
      &=
      - \frac{\partial}{\partial \vect{\theta}_{\text{UAV}}}
      \left(R_{v}^{b}\right)^\transpose \vect{v}_{b/I}^{b}
      \\
    \frac{\partial \dot{\vect{p}}_{\text{Goal}}}{\partial \vect{v}_{\text{UAV}}}
      &=
      - \left(R_{v}^{b}\right)^\transpose
      \\
\end{align}

\subsubsection{Jacobian w.r.t. Goal State}
\begin{equation}
  \frac{\partial \dot{\x}_{\text{Goal}}}{\partial \x_{\text{Goal}}}
  =
  \begin{bmatrix}
    \vect{0} & 0 & \frac{\partial \dot{\vect{p}}_{\text{Goal}}}{\partial
      \vect{v}_{\text{Goal}}} & \frac{\partial
      \dot{\vect{p}}_{\text{Goal}}}{\partial \theta_{\text{Goal}}} & 0 & \vect{0} & 0 \\
    \vect{0} & 0 & \vect{0} & 0 & 0 & \vect{0} & 0 \\
    \vect{0} & 0 & \vect{0} & 0 & \frac{\partial
      \dot{\theta}_{\text{Goal}}}{\partial \omega_{Goal}} & \vect{0} & 0 \\
    \vect{0} & 0 & \vect{0} & 0 & 0 & \vect{0} & 0 \\
    \vect{0} & 0 & \vect{0} & 0 & 0 & \vect{0} & 0
  \end{bmatrix}
\end{equation}
\begin{align}
    \frac{\partial \dot{\vect{p}}_{\text{Goal}}}{\partial
      \vect{v}_{\text{Goal}}}
      &=
      \left(R_{v}^{g}\right)^\transpose
      \\
    \frac{\partial \dot{\vect{p}}_{\text{Goal}}}{\partial \theta_{\text{Goal}}}
      &=
      \frac{\partial}{\partial \theta_{\text{Goal}}} \left(R_{v}^{g}\right)^\transpose \vect{v}_{g/I}^{g}
      \\
    \frac{\partial \dot{\theta}_{\text{Goal}}}{\partial \omega_{Goal}}
      &=
      1
\end{align}

\subsection{Goal ArUco Measurement}
\subsubsection{Measurement model}
In the case that and ArUco marker is placed as the landing pad, and therefore
defines the goal frame, we get a measurement of relative transfrom from the
camera to the ArUco frame. If we assume that there is a known rotation offset
between the ArUco frame and the goal frame, the measurements can be expressed as
\begin{equation}
  h \left( \x \right) = 
  \begin{bmatrix}
    \vect{p}_{a/c}^c \\
    \vect{q}_c^a,
  \end{bmatrix}
\end{equation}
where $F^a$ is the ArUco frame.
We can expand this measurement model using our estimated state
\begin{align}
  \hat{h} \left( \x \right) &=
  \begin{bmatrix}
    \hat{\vect{p}}_{a/c}^c \\
    \hat{\vect{q}}_c^a,
  \end{bmatrix} \\
  \hat{\vect{p}}_{a/c}^c  &= R_b^c \left( \hat{R}_I^b \hat{\vect{p}}_{g/v}^v -
    \vect{p}_{c/v}^b \right) \\
  \hat{\vect{q}}_{c}^a  &= R_g^a \hat{R}_I^g \hat{R}_b^I R_c^b
\end{align}

\subsubsection{Measurement Jacobians}
\begin{align}
  \frac{\partial \hat{\vect{p}}_{a/c}^c}{\partial \phi} =& R_b^c \frac{\partial
  R_I^b}{\partial \phi} \hat{\vect{p}}_{g/v}^v \\
  \frac{\partial \hat{\vect{p}}_{a/c}^c}{\partial \theta} =& R_b^c \frac{\partial
  R_I^b}{\partial \theta} \hat{\vect{p}}_{g/v}^v \\
  \frac{\partial \hat{\vect{p}}_{a/c}^c}{\partial \psi} =& R_b^c \frac{\partial
  R_I^b}{\partial \psi} \hat{\vect{p}}_{g/v}^v \\
    \frac{\partial \hat{\vect{p}}_{a/c}^c}{\partial \vect{p}_{g/v}^v} =& R_b^c \hat{R}_I^b
\end{align}

\subsection{Landmark Pixel Measurement}
We assume that landmark locations that are rigidly attached to the landing
vehicle are tracked in the camera image. For now we will just have a fixed
number of landmarks that are tracked for the entire duration of the flight, even
when they leave the FOV of the camera and reenter. In real life, however, the
landmarks must be dynamically added and removed from the estimated states vector
as they leave the FOV of the camera and more are acquired.

\subsubsection{Measurement Model}
The measurement is a pixel location of each landmark in the camera image
\begin{equation}
  \vect{z} =
  \begin{bmatrix}
    p_x & p_y
  \end{bmatrix}^\transpose.
\end{equation}
We can expand this measurements using our estimated states
\begin{align}
  \vect{z} &=
  \begin{bmatrix}
    p_x & p_y
  \end{bmatrix}^\transpose \\
  \vect{z} &=
  \begin{bmatrix}
    f_x \frac{\e_1 \vect{p}_{i/c}^c}{\e_3 \vect{p}_{i/c}^c} + c_x \\
    f_y \frac{\e_2 \vect{p}_{i/c}^c}{\e_3 \vect{p}_{i/c}^c} + c_y
  \end{bmatrix},
\end{align}
where $f_x$, $f_y$, $c_x$, and $c_y$ are constant parameters of the camera. We
can then express the position of the landmark $i$ with respect to the camera in the
camera frame as
\begin{align}
  \vect{p}_{i/c}^c &= R_b^c R_v^b \left(\vect{p}_{i/v}^v -
    \vect{p}_{c/v}^v\right) \\
    \vect{p}_{i/c}^c &= R_b^c \left(R_v^b \vect{p}_{i/v}^v -
    \vect{p}_{c/b}^b\right)
\end{align}
This is actually a little bit weird because of the state parameters, but
\begin{equation}
  \vect{p}_{i/v}^v = R_g^v \vect{p}_{i/g}^g + \vect{p}_{g/v}^v
  %\begin{bmatrix}
    %\e_1^\transpose \left( R_g^v \vect{p}_{i/g}^g + \vect{p}_{g/v}^v \right) \\
    %\e_2^\transpose \left( R_g^v \vect{p}_{i/g}^g + \vect{p}_{g/v}^v \right) \\
    %\left( \e_3^\transpose \vect{p}_{i/g}^g + \vect{p}_{g/v}^v \right)
  %\end{bmatrix}.
\end{equation}

\subsubsection{Measurement Jacobian}
For the purpose of deriving the jacobians, we will just look at the jacobians
with respect to $p_x$, the pixel location along the $x$ axis of the camera
frame. From above we have that
\begin{align}
  p_x &= f_x \frac{\e_1^\transpose \vect{p}_{i/c}^c}{\e_3^\transpose \vect{p}_{i/c}^c} + c_x \\ 
  p_x &= f_x \frac{\e_1^\transpose R_b^c \left(R_v^b \vect{p}_{i/v}^v -
      \vect{p}_{c/b}^b\right) }{\e_3^\transpose R_b^c \left(R_v^b \vect{p}_{i/v}^v -
  \vect{p}_{c/b}^b\right) } + c_x 
\end{align}
Note that all values are constants except for $\vect{p}_{i/v}^v$ and $R_v^b$.
The individual parts of the jacobian are given here:
\begin{equation}
  \frac{\partial p_x}{\partial \vect{p}_{g/v}^v} =
  \frac{f_x \e_1 R_b^c R_v^b}
    {\left(\e_3 R_b^c \left(R_v^b \vect{p}_{i/v}^v -
    \vect{p}_{c/b}^b\right)\right)}
    - \frac{\left(\e_3 R_b^c R_v^b \right) f_x \left(\e_1 R_b^c \left(R_v^b \vect{p}_{i/v}^v -
        \vect{p}_{c/b}^b\right)\right)} {\left(\e_3 R_b^c \left(R_v^b \vect{p}_{i/v}^v -
  \vect{p}_{c/b}^b\right)\right)^2}
\end{equation}
The derivatives for the euler angles are given by
\begin{equation}
  \frac{\partial p_x}{\partial \phi} =
  \frac{f_x \e_1 R_b^c \frac{\partial R_v^b}{\partial \phi} \vect{p}_{i/v}^v}
    {\left(\e_3 R_b^c \left(R_v^b \vect{p}_{i/v}^v -
    \vect{p}_{c/b}^b\right)\right)}
    - \frac{\left(\e_3 R_b^c \frac{\partial R_v^b}{\partial \phi} \vect{p}_{i/v}^v \right) f_x \left(\e_1 R_b^c \left(R_v^b \vect{p}_{i/v}^v -
        \vect{p}_{c/b}^b\right)\right)} {\left(\e_3 R_b^c \left(R_v^b \vect{p}_{i/v}^v -
  \vect{p}_{c/b}^b\right)\right)^2}
\end{equation}
where the jacobians for the other angles are similar, just substituting in.
The derivative for the goal angle, $\theta_g$ is given by using
\begin{equation}
  \frac{\partial \vect{p}_{i/v}^v}{\partial \theta_g} =
  \frac{\partial R_g^v}{\partial \theta_g} \vect{p}_{i/g}^g
  %\begin{bmatrix}
    %\e_1^\transpose \frac{\partial R_g^v}{\partial \theta_g} \vect{p}_{i/g}^g \\
    %\e_2^\transpose \frac{\partial R_g^v}{\partial \theta_g} \vect{p}_{i/g}^g \\
    %0
  %\end{bmatrix}
\end{equation}
in the jacobian given by
\begin{equation}
  \frac{\partial p_x}{\partial \theta_g} =
  \frac{f_x \e_1 R_b^c R_v^b \frac{\partial \vect{p}_{i/v}^v}{\partial \theta_g}}
    {\left(\e_3 R_b^c \left(R_v^b \vect{p}_{i/v}^v -
    \vect{p}_{c/b}^b\right)\right)}
    - \frac{\left(\e_3 R_b^c R_v^b \frac{\partial \vect{p}_{i/v}^v}{\partial \theta_g} \right) f_x \left(\e_1 R_b^c \left(R_v^b \vect{p}_{i/v}^v -
        \vect{p}_{c/b}^b\right)\right)} {\left(\e_3 R_b^c \left(R_v^b \vect{p}_{i/v}^v -
  \vect{p}_{c/b}^b\right)\right)^2}
\end{equation}
Similarly, the derivative for the landmark offset, $\vect{r}_i$ or
$\vect{p}_{i/g}^g$ is given by using
\begin{equation}
  \frac{\partial \vect{p}_{i/v}^v}{\partial \vect{p}_{i/g}^g} = R_g^v 
  %\begin{bmatrix}
    %\e_1^\transpose R_g^v \\
    %\e_2^\transpose R_g^v \\
    %\e_3^\transpose
  %\end{bmatrix}
\end{equation}
in the jacobian given by
\begin{equation}
  \frac{\partial p_x}{\partial \vect{p}_{i/g}^g} =
  \frac{f_x \e_1 R_b^c R_v^b \frac{\partial \vect{p}_{i/v}^v}{\partial \vect{p}_{i/g}^g}}
    {\left(\e_3 R_b^c \left(R_v^b \vect{p}_{i/v}^v -
    \vect{p}_{c/b}^b\right)\right)}
    - \frac{\left(\e_3 R_b^c R_v^b \frac{\partial \vect{p}_{i/v}^v}{\partial \vect{p}_{i/g}^g} \right) f_x \left(\e_1 R_b^c \left(R_v^b \vect{p}_{i/v}^v -
        \vect{p}_{c/b}^b\right)\right)} {\left(\e_3 R_b^c \left(R_v^b \vect{p}_{i/v}^v -
  \vect{p}_{c/b}^b\right)\right)^2}
\end{equation}
The jacobians for the y pixel measurement, $p_y$ are similar to the ones derived
above, just changing $f_y$ in place of $f_x$ and $\e_2$ instead of $\e_1$.

\subsection{Landmark Initialization}
Each time a new landmark is acquired and added to the estimated vector, we
should initialize its location to something intelligent. Theoretically, we could
just initialize the landmark's location to zero with a large enough associated
covariance. However, by initializing the landmark to the location based on its
initial pixel measurement, we can initialize the landmark with a smaller
covariance.

When we first acquire a landmark, the only information we have about it is a
measurement of its pixel location in the image frame. This measurement, denoted
by
\begin{equation}
  h \left( \x \right) = 
  \begin{bmatrix}
    p_x & p_y
  \end{bmatrix}^\transpose
\end{equation}
can be used to deduce the landmark's relative position vector that is estimated.
The easiest way to see this is by first creating a virtual image plane. This
virtual image plane projects the landmark pixel location into the image plane as
if the camera were perfectly aligned with the inertial, vehicle frame. We can
derive this using the pin hole camera model where
\begin{align}
  \begin{bmatrix}
    p_x \\ p_y \\ 1
  \end{bmatrix} &= K
  \begin{bmatrix}
    X^c / Z^c \\
    Y^c / Z^c \\
    1
  \end{bmatrix} \\
  \begin{bmatrix}
    X^c / Z^c \\
    Y^c / Z^c \\
    1
  \end{bmatrix}
   &= K^{-1}
  \begin{bmatrix}
    p_x \\ p_y \\ 1
  \end{bmatrix} \\
  \begin{bmatrix}
    X^c / Z^c \\
    Y^c / Z^c \\
    1
  \end{bmatrix}^v
   &= R_b^v R_c^b K^{-1}
  \begin{bmatrix}
    p_x \\ p_y \\ 1
  \end{bmatrix} \\
\end{align}
The resulting vector, $
  \begin{bmatrix}
    X^c / Z^c \\
    Y^c / Z^c \\
    1
  \end{bmatrix}^v$
  is the vector $\vect{p}_{i/c}^v$ up to a scale factor. The vector can be scaled by assuming that the altitude of the landmark is equal to the
altitude of the goal. To get the expected altitude, we solve for
\begin{align}
  \e_3^\transpose \vect{p}_{g/c}^v &= \e_3^\transpose \vect{p}_{g/b}^v - \e_3^\transpose \vect{p}_{c/b}^v \\
  \e_3^\transpose \vect{p}_{g/c}^v &= \e_3^\transpose \vect{p}_{g/b}^v -
  \e_3^\transpose R_b^I \vect{p}_{c/b}^b
  %\e_3^\transpose \vect{p}_{g/c}^v &= \frac{1}{\rho_g} -
  %\e_3^\transpose R_b^I \vect{p}_{c/b}^b \\
\end{align}

This gives us the vector, $\vect{p}_{i/c}^v$.
With this vector, we can then reach the estimated state vector,
$\vect{p}_{i/g}^g$ with the following
\begin{align}
  \vect{p}_{i/v}^v &= \vect{p}_{i/c}^v + R_b^I \vect{p}_{c/b}^b \\
  \vect{p}_{i/g}^g &= R_v^g \left( \vect{p}_{i/v}^v - \vect{p}_{g/v}^v \right).
\end{align}



\section{Control} \label{sec:control}

\section{Simulation} \label{sec:simulation}

\section{Hardware} \label{sec:hardware}

\subsection{Platform}
Describe indoor and outdoor. Multirotor. Cart. Truck.

\subsubsection{Feature Tracking}
FAST Features, Optical Flow, Space them out, Keep track of longest tracked.

\subsubsection{Indoor Motion Capture}

\subsubsection{Outdoor}

\section{Conclusion}
The conclusion goes here.

% \appendices
% \section{Proof of the First Zonklar Equation}
% Appendix one text goes here.

% \section{}
% Appendix two text goes here.

% \section*{Acknowledgment}
% The authors would like to thank...

\bibliographystyle{IEEEtran}
\bibliography{abbrev,library}

\end{document}



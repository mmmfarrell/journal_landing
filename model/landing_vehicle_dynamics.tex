% !TEX root=./root.tex

\subsection{Landing Vehicle Dynamics}
Without loss of generality, we assume that the motion of the landing vehicle can be modeled with a constant
velocity unicycle motion model. In our experiments we show that this very simple
model produces satisfactory results in a variety of scenarios. We point out,
however, that the motion model can easily be substituted for a more accurate
model for different use cases (i.e. a car model). For our case, the state of the landing vehicle is expressed as the tuple of
position, velocity, attitude and angular rotation rate
\begin{equation*}
  \x =
    \begin{pmatrix}
      \vect{p}_{g/I}^{I}, \vect{v}_{g/I}^{g}, \theta_{I}^{g},
      \omega_{g/I}^{g}
    \end{pmatrix}
    \in
    \mathbb{R}^2 \times \mathbb{R}^2 \times S^1 \times \mathbb{R}^1.
\end{equation*}
The constant velocity, unicyle motion model we use to describe the motion of the
landing vehicle can be written as
\begin{align}
  \dot{\vect{p}}_{g/I}^I
  &=
  \left( R_I^g \right)^\transpose \vect{v}_{g/I}^g  \\
  \dot{\vect{v}}_{g/I}^g
  &=
  \vect{0} + \vect{\eta}_v \\
  \dot{\theta}_{I}^g
  &=
  \omega_{g/I}^g \\
  \dot{\omega}_{g/I}^g
  &=
  0 +\eta_\omega,
\end{align}
where $\vect{\eta}_v$ and $\eta_\omega$ describe the random walk of the velocity
and rotational velocity of the landing vehicle.

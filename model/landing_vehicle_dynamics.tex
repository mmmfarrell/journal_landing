% !TEX root=./root.tex

\subsection{Landing Vehicle Dynamics}
Without loss of generality, we assume that the motion of the landing vehicle can be modeled with a constant
velocity unicycle motion model. In our experiments we show that this very simple
model produces satisfactory results in a variety of scenarios. We point out,
however, that the motion model can easily be substituted for a more accurate
model for different use cases. Some specific motion models that may be of
particular use include a car model and a 3d ???.

For our use case, the state variables of the landing vehicle are expressed as
\begin{equation}
  \x =
    \begin{bmatrix}
      \vect{p}_{g/I}^{I} & \vect{v}_{g/I}^{g} & \theta_{I}^{g} &
      \omega_{g/I}^{g}
    \end{bmatrix}^{\transpose}
\end{equation}
where $\vect{p}_{g/I}^{I}$ represents the two-dimensional position of the
landing vehicle w.r.t. the inertial frame, $\vect{v}_{g/I}^{g}$ the
two-dimensional velocity, $\theta_{I}^{g}$ the rotation angle about the
inertial $z$ axis, and $\omega_{g/I}^{g}$ the rotational velocity about the
inertial $z$ axis.


The constant velocity unicyle motion model used in this paper can be written as

\begin{align}
  \dot{\vect{p}}_{g/I}^I
  &=
  \left( R_I^g \right)^\transpose \vect{v}_{g/I}^g  \\
  \dot{\vect{v}}_{g/I}^g
  &=
  \vect{0} \\
  \dot{\theta}_{I}^g
  &=
  \omega_{g/I}^g \\
  \dot{\omega}_{g/I}^g
  &=
  0.
\end{align}

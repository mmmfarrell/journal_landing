% TEX root=../root.tex

\subsection{2D Rotation Group}
We parameterize the 2D rotation group as an angle, $\psi$, that respresents
the angle of rotation about a given axis. We treat $\psi$ as a vector space
that uses the common addition and subtraction operators such that
\begin{align}
  \psi_a^c &= \psi_a^b + \psi_b^c \\
  \psi_a^b &= \psi_a^c - \psi_b^c.
\end{align}
We note, however, that with this formulation, all addition and subtraction
operations must be wrapped such
that the resultant angle $\psi \in \left[ \right. -\pi, \pi \left. \right)$. The
passive 2D
rotation matrix can be created from any $\psi$ as
\begin{equation}
  R \left( \psi \right) = \begin{bmatrix} \cos \psi & -\sin \psi \\ \sin \psi &
  \cos \psi \end{bmatrix}.
\end{equation}
If $\psi$ represents the angle of rotation about the $z$ axis of a frame, then
the corresponding passive 3D rotation matrix is given by
\begin{equation}
  R \left( \psi \right) = \begin{bmatrix}
    \cos \psi & -\sin \psi & 0 \\
    \sin \psi & \cos \psi & 0 \\
    0 & 0 & 1
  \end{bmatrix}.
\end{equation}
% We treat S1 as a vector space but make sure to wrap residuals between -pi and
% pi. 
% Also a 2D rotation matrix can be made like this...
% And a 3D rotation matrix like this..

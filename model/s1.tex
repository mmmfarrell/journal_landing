% TEX root=../root.tex

\subsection{2D Rotation Group}
We parameterize the 2D rotation group as an angle, $\theta$, that respresents
the angle of rotation about a given axis. We treat $\theta$ as a vector space
that uses the common addition and subtraction operators such that
\begin{align}
  \theta_I^g &= \theta_I^b + \theta_b^g \\
  \theta_I^b &= \theta_I^g - \theta_b^g.
\end{align}
We note, however, that all addition and subtraction operations are wrapped such
that the resultant angle $\in \left[ \right. -\pi, \pi \left. \right)$. The 2D
rotation matrix can be created from any $\theta$ as
\begin{equation}
  R \left( \theta \right) = \begin{bmatrix} \ctheta & -\stheta \\ \stheta &
  \ctheta \end{bmatrix}.
\end{equation}
If $\theta$ represents the angle of rotation about the $z$ axis of a frame, then
the corresponding 3D rotation matrix is given by
\begin{equation}
  R \left( \theta \right) = \begin{bmatrix}
    \ctheta & -\stheta & 0 \\
    \stheta & \ctheta & 0 \\
    0 & 0 & 1
  \end{bmatrix}.
\end{equation}
% We treat S1 as a vector space but make sure to wrap residuals between -pi and
% pi. 
% Also a 2D rotation matrix can be made like this...
% And a 3D rotation matrix like this..

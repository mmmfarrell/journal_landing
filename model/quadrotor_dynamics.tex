% !TEX root=../root.tex

\subsection{UAV State Dynamics}
The state variables of the multirotor UAV are expressed as
\begin{equation}
  \x =
    \begin{bmatrix}
      \vect{p}_{b/I}^{I} & 
      \phi & \theta & \psi &
      \vect{v}_{b/I}^{b} 
    \end{bmatrix}^{\transpose}
\end{equation}
where $\phi$, $\theta$, and $\psi$ represent the Z-Y-X Euler angles known as
roll, pitch, and yaw. With these angles, we note that the rotation matrix
$R_I^b$ can be expressed as
\begin{equation*}
  R_I^b \left( \phi, \theta, \psi \right) =
  \begin{bmatrix}
    \ctheta \cpsi & \ctheta \spsi & -\stheta \\
    \sphi \stheta \cpsi - \cphi \spsi & \sphi \stheta \spsi + \cphi
    \cpsi & \sphi \ctheta \\
    \cphi \stheta \cpsi + \sphi \spsi & \cphi \stheta \spsi - \sphi
    \cpsi & \cphi \ctheta
  \end{bmatrix}.
\end{equation*}
We recognize that Euler angles do not form a proper vector space~\cite{??},
however, since we operate the UAV close to level flight, the approximation of
Euler angles as a vector space is satisfactory.

We use common UAV dynamics with the addition of a linear drag term as explained
in~\cite{leishman2014accel}. For this system, the inputs are given by
\begin{equation}
  \vect{u} =
    \begin{bmatrix}
      a_z & \vect{\omega}_{b/I}^b 
    \end{bmatrix}^\transpose
\end{equation}
and the dynamics given by
\begin{align}
  \dot{\vect{p}}_{b/I}^I
  &=
  \left( R_I^b \right)^\transpose \vect{v}_{b/I}^b 
  \\
  \begin{bmatrix}
    \dot{\phi} \\
    \dot{\theta} \\
    \dot{\psi}
  \end{bmatrix}
  &=
  \begin{bmatrix}
    1 & \sin\phi\tan\theta & \cos\phi\tan\theta \\
    0 & \cos\phi & -\sin\phi \\
    0 & \frac{\sin\phi}{\cos\theta} & \frac{\cos\phi}{\cos\theta}
  \end{bmatrix}
  \vect{\omega}_{b/I}^b
  \\
  \dot{\vect{v}}_{b/I}^b 
  &=
  R_I^b \vect{g}
  +
  \skewmat{\vect{v}_{b/I}^b} \vect{\omega}_{b/I}^b
  -
  a_z \e_3 \\
  &\qquad -
  \begin{bmatrix}
    \mu u \\
    \mu v \\
    0 \nonumber
  \end{bmatrix}.
\end{align}


% !TEX root=../root.tex

\subsection{Notation}

Throughout the paper, we represent vectors with a bold letter (e.g., $\bf v$)
and matrices with a captial letter (e.g., $A$). Other common notation used
throughout the paper is contained below.
\begin{center}
\begin{tabularx}{\columnwidth}{lX}
$R_a^b$ & Rotation matrix from reference frame $a$ to $b$ \\
$\vect{v}_{a/b}^c$ & Vector state $\vect{v}$ of frame $a$ w.r.t.~frame $b$, expressed in frame $c$ \\
$\hat{a}$ & Estimate of true variable $a$ \\
$\bar{a}$ & Measurement of $a$ \\
% $\des{a}$ & Desired value of $a$ \\
$\dot{a}$ & Time derivative of $a$ \\
$\tilde{a}$ & Error of variable $a$, i.e., $\tilde{a} \triangleq a - \hat{a}$
\end{tabularx}
\end{center}
%
We also make use of the following coordinate frames:
\begin{center}
\begin{tabularx}{\columnwidth}{lX}
$I$ & The inertial coordinate frame in north-east-down\\
% $\ell$ & The aircraft's vehicle-1 (body-level) coordinate frame \\
$v$ & The aircraft's vehicle-1 (body-level) coordinate frame \\
$b$ & The aircraft's body-fixed coordinate frame \\
$c$ & The camera frame \\
$g$ & The landing vehicle's body-fixed coordinate frame located at the desired
landing location (goal) of the aircraft
\end{tabularx}
\end{center}
%
We use the standard basis vectors $\vect{e}_1, \vect{e}_2, \vect{e}_3$,
where $\vect{e}_1 = \begin{bmatrix} 1 & 0 & 0 \end{bmatrix}^\transpose$,
$\vect{e}_2 = \begin{bmatrix} 0 & 1 & 0 \end{bmatrix}^\transpose$,
and $\vect{e}_3 = \begin{bmatrix} 0 & 0 & 1 \end{bmatrix}^\transpose$.
We use the skew-symmetric matrix operator
% We make frequent use of the skew-symmetric matrix operator defined by
\begin{equation}
  \skewmat{\vect{v}} \triangleq
  \begin{bmatrix}
  0 & -v_{3} & v_{2}\\
  v_{3} & 0 & -v_{1}\\
  -v_{2} & v_{1} & 0
  \end{bmatrix},
\end{equation}
which is related to the cross-product between two vectors as
\begin{equation}
  \vect{v}\times\vect{w}=\skewmat{\vect{v}}\vect{w}
\end{equation}
and the skew-symmetric identity that
\begin{equation}
  \skewmat{\vect{v}} \vect{w} = -\skewmat{\vect{w}} \vect{v}.
\end{equation}
We also use the 2D skew-symmetric operator which acts on a scalar
\begin{equation}
  \skewmat{a} \triangleq
  \begin{bmatrix}
  0 & -a \\
  a & 0
  \end{bmatrix}.
\end{equation}
We use the identity matrix $I$ as well as submatricies of $I$ denoted with
subscripts such that
\begin{equation}
  I_{2 \times 3} =
  \begin{bmatrix}
    1 & 0 & 0 \\
    0 & 1 & 0
  \end{bmatrix}.
\end{equation}

% !TEX root=../root.tex

\subsection{Notation}

We define some common notation used throughout the paper, first noting that
vectors are represented with a bold letter (e.g., $\bf v$) and matrices with a
capital letter (e.g., $A$).
\begin{center}
\begin{tabularx}{\columnwidth}{lX}
$R_a^b$ & Rotation matrix from reference frame $a$ to $b$ \\
$\vect{v}_{a/b}^c$ & Vector state $\vect{v}$ of frame $a$ w.r.t.~frame $b$, expressed in frame $c$ \\
$\hat{a}$ & Estimate of true variable $a$ \\
$\bar{a}$ & Measurement of $a$ \\
$\des{a}$ & Desired value of $a$ \\
$\dot{a}$ & Time derivative of $a$ \\
$\tilde{a}$ & Error of variable $a$, i.e., $\tilde{a} \triangleq a - \des{a}$
\end{tabularx}
\end{center}
%
We also define the following coordinate frames:
\begin{center}
\begin{tabularx}{\columnwidth}{lX}
$I$ & The inertial coordinate frame in north-east-down\\
$\ell$ & The aircraft's vehicle-1 (body-level) coordinate frame \\
$b$ & The aircraft's body-fixed coordinate frame \\
$c$ & The camera frame \\
$g$ & The goal frame, the landing vehicle's body-fixed coordinate frame located at the desired landing location of the aircraft
\end{tabularx}
\end{center}

% We make frequent use of the skew-symmetric matrix operator defined by
% \begin{equation}
  % \skewmat{\vect{v}} \triangleq
  % \begin{bmatrix}
  % 0 & -v_{3} & v_{2}\\
  % v_{3} & 0 & -v_{1}\\
  % -v_{2} & v_{1} & 0
  % \end{bmatrix},
% \end{equation}
% which is related to the cross-product between two vectors as
% \begin{equation}
  % \vect{v}\times\vect{w}=\skewmat{\vect{v}}\vect{w}.
% \end{equation}

We abbreviate the commonly used trigonometric functions $\sin$, $\cos$, and
$\tan$ with their first letter and the angle they operate on as a subscript
such that $\sphi = \sin\phi$.
We use the standard basis vectors $\vect{e}_1, \vect{e}_2, \ldots, \vect{e}_N$,
where $\vect{e}_1 = \begin{bmatrix} 1 & 0 & \cdots & 0 \end{bmatrix}^\transpose$
and so forth.
